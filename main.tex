%!TEX encoding = UTF-8 Unicode
%!TEX program = xelatex

\documentclass[doctor]{hnuthesis}
\usepackage{subfigure}
\usepackage{amsmath}
\usepackage{algorithmicx,algpseudocode}
\usepackage{float}

% 学校代码
\hnucode{10532}
% 学校名称
\hnuname{湖南大学}
\enhnuname{Hunan University}
% 中图分类号
\clc{TP391}         
% 密级       
\secrettext{普通}           

% 标题
\title{第三类永动机的实现}
\entitle{The Implementation of the Third Perpetual Motion Machine}
% 作者
\author{郭靖}
\enauthor{JING Guo}
% 学号
\authorid{S18100001}      
% 学院
\college{信息科学与工程学院}
% 专业
\major{计算机科学与技术} 
\enmajor{Computer Science and Technology}
%学士学位获得学校,年份
\enbachelor{B.E.~(Hunan University)201x}
% \enmaster{M.S.~(Hunan University)2020}
\endoctor{Master of engineering}
% 研究方向
\workon{第三类永动机}
% 导师
\supervisor{洪七公\ 教授}
\ensupervisor{Professor QIGONG Hong}
% 论文提交、答辩日期
\submitdate{二〇二x年x月xx日}
\defensedate{二〇二X年x月xx日}
\endate{June, 2020}
% 答辩委员会主席
\chair{待定}

\begin{document}
% 封面、原创性声明
\maketitle

% 摘要
%中文摘要
\begin{abstract}
	连续子图匹配(Continuous Subgraph Matching,CSM)是动态图分析中的一个重要问题。该问题要求在一个动态变化的图中,寻找与查询图同构的所有子图,并根据匹配结果进行后续分析。
	由于动态图中的边和节点会实时发生变化,传统的静态子图匹配方法难以应对这些变化。

	目前,现有的CSM方法在处理大规模动态图时存在一个突出的问题:当查询图较为简单时,通常会产生大量的匹配结果。
	对于数据分析人员而言,过多的匹配子图不仅增加了计算负担,而且可能导致有价值的信息被淹没。
	此外,许多现实世界中的图是加权图,如支付网络,其中每条边的权重代表交易金额。
	在分析过程中,边的权重通常对匹配结果的优先级产生重要影响。然而,现有的CSM方法大多忽略了这一因素,未能有效地将边权引入优化策略。

	针对上述问题,本文提出了一个新的研究问题——CSM-TopK,即在动态加权图中计算查询图的前k个密度最高的匹配结果,并证明该问题为NP难问题。本文的主要贡献如下:
	\begin{enumerate}[label=(\arabic*)]  
	\item 为了有效解决CSM-TopK问题,我们首先定义了一种星形结构子图,并通过查询图匹配序列中节点的一阶邻居关系来导出该星形子图。
	\item 在星形结构子图的基础上,我们设计了两种轻量级索引,分别为全局MWstar索引和局部MWstar索引。具体而言,全局MWstar维护每个特定星形结构子图的最大权重,并在子图搜索的过程中保持不变,利用该最大权重,我们可以提前计算匹配子图的最大密度上限,从而进行剪枝,减少搜索空间,提高子图匹配的效率。
	而局部MWstar则根据每个特定数据顶点的最大权重分布动态维护索引,其最大权重的密度上限更加紧凑,因此能够更快地减少不必要的计算。
	\item 为了进一步提高性能,本文还引入了一种查询相关的图压缩技术,该技术利用标签过滤方式过滤不可能的候选点,从而生成针对此查询图的压缩图。由于压缩图的规模远小于数据图,在此基础上进行星形结构索引的初始化以及更新,显著降低了匹配子图的密度上限,从而提高了整体的时间和空间效率。
  	\end{enumerate}

	通过在五个真实世界数据集上的广泛实验,结果表明,结合图压缩技术和MWstar索引的最终方法表现最佳,在插入/删除的时间效率上,该方法相比现有对比方法至少提高了两个数量级;在索引的空间效率以及时间效率上也远超其他方法。实验结果验证了本文提出解决方案的有效性。
	\keywords{连续子图匹配;动态图;剪枝策略;TopK密度}
\end{abstract}

%英文摘要
\begin{enabstract}
	Continuous Subgraph Matching (CSM) is an important problem in dynamic graph analysis. 
	The problem requires finding all subgraphs in a dynamically changing graph that are isomorphic to a given query graph, and performing further analysis based on the matching results. 
	Since edges and vertices in a dynamic graph change in real-time, traditional static subgraph matching methods struggle to cope with these changes.

    Currently, existing CSM methods face a prominent issue when handling large-scale dynamic graphs: when the query graph is simple, it often generates a large number of matching results. 
	For data analysts, an excessive number of matching subgraphs not only increases the computational burden but may also drown out valuable information. 
	Moreover, many real-world graphs are weighted, such as payment networks, where each edge’s weight represents the transaction amount. 
	During analysis, edge weights often significantly influence the priority of matching results. However, most existing CSM methods overlook this factor and fail to effectively incorporate edge weights into the optimization strategies.

	
	To address these challenges, the paper introduces a new problem called CSM-TopK, which aims to compute the top k matching results with the highest density for a given query graph on dynamic weighted graphs and proves that this problem is NP-hard. 
	The main contributions are as follows:
	\begin{enumerate}[label=(\arabic*)]  
		\item To efficiently solve the CSM-TopK problem, we first define a star-structured subgraph, which derives the star-structured subgraph using the first-order neighborhood relationships of the vertices in the query graph’s matching order.
		\item Based on the star-structured subgraph of the query graph, we design two lightweight indexes, namely the Global MWstar Index and the Local MWstar Index. 
		Specifically, the Global MWstar maintains the maximum weight of each specific star-structured subgraph, which remains unchanged during subgraph search. 
		Using this maximum weight, we can compute the upper bound of the maximum density for the matching subgraph in advance, enabling pruning to reduce the search space and improve matching efficiency. 
		In contrast, the Local MWstar dynamically maintains the index based on the maximum weight distribution of each specific data vertex. 
		Its maximum weight density upper bound is more compact, allowing for faster reduction of unnecessary computations.
		\item To further enhance performance, we also introduce a query-dependent graph compacted technique. 
		This technique uses label filtering to remove impossible candidate vertices, yielding the candidate subgraph specifically for the query graph. 
		The size of the candidate subgraphs is significantly smaller than the original data graph. 
		The initialization and updating of the star-structured index on the candidate subgraph substantially reduce the upper bound of the matching subgraph’s density, thereby improving both time and space efficiency.
		\end{enumerate}

		Extensive experiments on five real-world datasets demonstrate that the proposed method combining graph compression techniques and MWstar indexing performs the best. 
		In terms of time efficiency for insertions/deletions, it outperforms existing comparison methods by at least two orders of magnitude. 
		It also significantly exceeds other methods in both index space efficiency and time efficiency. 
		These experimental results validate the effectiveness of the proposed solution.
	\enkeywords{Continuous Subgraph Matching; Dynamic Graph; Pruning Strategy; TopK Density}
\end{enabstract}
% 目录
\tableofcontents
% 插图附表索引
\begingroup
    \renewcommand*{\addvspace}[1]{}
        \phantomsection
        \listoffigures
        \newpage

        \phantomsection
        \listoftables
        \newpage
\endgroup

% 正文章节
\mainmatter
\chapter{绪论}
\section{研究背景与意义}
近年来,互联网实现了跨越式发展,广泛覆盖社交网络分析、生物信息学、网络安全和交通规划等多个领域。
这些实时应用会源源不断产出海量复杂的数据,这些数据极为精准地勾勒出了现实世界中不同对象间的关联关系。
这些信息将有助于人们获得重要的决策指导信息。
这些关系数据不仅有助于描述复杂的对象关系,还能为决策者提供关键的参考信息 。
%以金融领域为例,分析第三方支付平台中的账户对象及其交易行为,可有效挖掘潜在的欺诈模式;在网络安全领域,监控设备通信流量的数据有助于及时识别并预防恶意攻击。


在动态数据的分析中,连续子图匹配(Continuous Subgraph Matching,简称CSM)问题是一个重要的研究方向,主要用于在动态图数据中实时识别与给定模式相匹配的子图。
与传统的静态子图匹配问题不同,CSM 需要在数据不断变化的情况下,依然能够高效、准确地找到符合特定结构的子图。
这种动态变化的场景在许多现实世界应用中具有广泛的适用性。
在金融反欺诈中,我们希望及时发现隐藏在交易网络中的大型欺诈团伙;在网络安全领域,安全人员更关注攻击模式的演变,以便及时拦截恶意攻击。
这些应用都对动态图子图匹配的计算效率和准确性提出了更高要求,因此,如何在动态环境下高效执行 CSM,成为了一个亟待解决的挑战。

与此同时,随着数据规模的爆炸式增长,这一问题变得更加紧迫。特别是进入 5G 时代后,数据的体量、种类和变化速度远超以往。
例如,截至2024年11月13日,微信及WeChat的月活跃账户数已达到13.82亿,几乎覆盖了全国人口,并且仍在增长;
而在电商领域,2024 年“双十一”购物节的全网交易额突破 1.44 万亿元,再创新高。
这些现象表明,我们正处在一个数据规模前所未有庞大的时代,而如何高效处理这些数据,成为计算机科学领域的重要课题之一。

在如此庞大的图数据中,符合某个特定模式的匹配结果可能非常多,但在实际应用中,我们通常更关注最有价值的结果。
例如,在金融欺诈模式的挖掘中~\cite{csm-cycle-DBLP:journals/pvldb/QiuCQPZLZ18},监管机构更关心那些涉及交易金额最高的欺诈团伙,而非所有可疑交易模式;
在网络安全领域,管理员更倾向于定位受攻击最严重恶意攻击子网~\cite{traffic-graph-matching-DBLP:journals/pvldb/SongGCW14},而非每一个潜在的攻击行为。
因此,在动态图匹配时,除了考虑拓扑结构,还需要引入边权属性、设定密度限制,以筛选出更具分析价值的子图集合。
这不仅能提高匹配结果的相关性,也能帮助分析人员更快定位核心问题,从而提升整体效率。
在如此庞大的图数据中,符合某个特定模式的匹配结果可能非常多,但在实际应用中,我们通常更关注最有价值的结果。
例如,在金融欺诈模式的挖掘中~\cite{csm-cycle-DBLP:journals/pvldb/QiuCQPZLZ18},监管机构更关心那些涉及交易金额最高的欺诈团伙,而非所有可疑交易模式;
在网络安全领域,管理员更倾向于定位受攻击最严重恶意攻击子网~\cite{traffic-graph-matching-DBLP:journals/pvldb/SongGCW14},而非每一个潜在的攻击行为。
因此,在动态图匹配时,除了考虑拓扑结构,还需要引入边权属性、设定密度限制,以筛选出更具分析价值的子图集合。
这不仅能提高匹配结果的相关性,也能帮助分析人员更快定位核心问题,从而提升整体效率。

然而,目前的研究大多集中在静态图中的Top-k密度约束子图匹配问题~\cite{density-define-DBLP:journals/vldb/AngelKSSST14,dsm-noweight-Bahmani-DBLP:journals/pvldb/BahmaniKV12},这种方法构建大量的离线索引来提供权重的上界,然而其索引构建所需要的时间和空间受到了查询图中的标签大小以及数量的影响,其高昂的索引构建开销使得这些方法在动态图中不具有普遍适用性。
此外,现有动态图的工作只聚焦于所有子图匹配结果的返回,在结果数量过大的情况下,也只能随机限定返回数量的上限,缺乏对返回结果的其他条件约束~\cite{csm-sjtree-DBLP:conf/edbt/ChoudhuryHCAF15,csm-IncIsoMatch-DBLP:conf/sigmod/FanLLTWW11,dsm-noweight-Hu-DBLP:conf/cikm/HuWC17,csm-turboflux-DBLP:conf/sigmod/KimSHLHCSJ18,csm-graphflowpp-DBLP:journals/tods/MhedhbiKS21,csm-symbi-DBLP:journals/pvldb/MinPPGIH21,csm-rapidflow-DBLP:journals/pvldb/SunSHL22}。

因此,在动态图场景中,结合密度约束和拓扑结构的连续子图匹配具有重要的实际应用价值。
通过密度优先排序返回密度最高的前k个匹配子图,数据分析人员可以更快定位最具代表性和决策价值的匹配结果,从而为各类实际应用提供更有针对性的支持。
基于这一背景,本文将重点探讨如何在连续子图匹配过程中引入密度剪枝机制,以提高算法的计算效率和匹配准确性,从而更好地应对动态图环境下的子图匹配挑战。
\section{国内外研究现状}
连续子图匹配(Continuous Subgraph Matching,CSM)是图数据管理领域的一个重要的研究方向,主要关注如何在动态图中高效检测与查询模式匹配的子图。
而与本课题——密度约束下的TopK连续子图匹配相关的工作有三类:现有连续子图匹配问题,Top-k密集子图挖掘问题,以及静态Top-k子图匹配的搜索过程。
然而,现有的研究方法普遍存在某些局限性,无法完全满足动态图场景下高效匹配的需求。以下将分别讨论这三类研究的主要进展及其局限性。
\subsection{现有的连续子图匹配方案}
在现有的关于连续子图匹配(CSM)的工作\cite{csm-sjtree-DBLP:conf/edbt/ChoudhuryHCAF15,csm-IncIsoMatch-DBLP:conf/sigmod/FanLLTWW11,csm-graphflow-DBLP:conf/sigmod/KankanamgeSMCS17,csm-turboflux-DBLP:conf/sigmod/KimSHLHCSJ18,csm-graphflowpp-DBLP:journals/tods/MhedhbiKS21,csm-symbi-DBLP:journals/pvldb/MinPPGIH21,csm-rapidflow-DBLP:journals/pvldb/SunSHL22}中,根据是否依赖辅助数据结构和索引,相关方法可以分为两类。

(1) 不依赖任何辅助数据结构或索引的方法。

爱丁堡大学的樊文飞教授提出了InslsoMatch\cite{csm-IncIsoMatch-DBLP:conf/sigmod/FanLLTWW11}方法,该方法通过提取局部更新范围的子图,并在搜索过程中调用静态子图匹配算法。
然而,静态算法的查询延迟较高,因为其通常需要构建复杂的索引或遍历整个图,导致查询效率低下。
加拿大滑铁卢大学的Semih团队提出了GraphFlow\cite{csm-graphflow-DBLP:conf/sigmod/KankanamgeSMCS17},利用Worst-Case Optiomal Join来优化搜索的过程,显著减少了搜索空间。
该团队随后提出了GraphFlow+\cite{csm-graphflowpp-DBLP:journals/tods/MhedhbiKS21},该方法通过缓存GraphFlow的部分联接结果以加速后续的搜索。
这些方法均不维护任何的中间结果,并保留了完整的子图匹配结果集。

(2) 基于辅助数据结构和索引的方法

除了直接依赖增量计算的匹配方法外,另一类研究则借助辅助数据结构(如生成树、有向无环图(DAG))来加速子图匹配,以降低计算开销并提高查询效率。
例如,韩国的Kyoungmin提出TurboFlux\cite{csm-turboflux-DBLP:conf/sigmod/KimSHLHCSJ18},这是一种基于生成树索引的连续子图匹配方法。它在查询图的生成树基础上,在数据图中构建相应的辅助索引,从而提升匹配效率。
TurboFlux 的核心优势在于其增量更新能力,可以快速定位受影响的区域,避免对整个图重新计算,从而大幅降低计算成本,使其在动态图环境下具有良好的适应性。
此外,韩国科学技术院的 Wook-Shin Han 教授团队提出Symbi\cite{csm-symbi-DBLP:journals/pvldb/MinPPGIH21},该方法将查询图转化为有向无环图,并利用 DAG 对非树边的剪枝能力,在搜索过程中实现显著加速。。
另一项值得关注的研究是RapidFlow\cite{csm-rapidflow-DBLP:journals/pvldb/SunSHL22},其采用全局索引策略,根据查询图的拓扑结构构建全局索引,并在查询过程中利用该索引快速生成局部索引,极大地减少子图匹配的计算开销。
此外,复旦大学郑卫国教授团队提出了一种具有成本效益的索引方法——CaLiG\cite{csm-calig-DBLP:journals/pacmmod/YangZZY23},其核心思想是通过构建高效索引结构来减少回溯搜索的空间。
CaLiG 通过构建高效的索引结构,能够在匹配过程中快速剪枝,从而更高效地完成匹配任务。该方法在处理具有复杂拓扑结构的查询图时表现出较好的性能。

在国内,其他研究团队也在连续子图匹配领域作出了重要的贡献。例如北京大学的高军教授团队、邹磊教授团队、北京航天航空大学的马帅教授团队等,都在该领域开展了深入研究,并提出了多种优化策略。
尽管上述研究在提升匹配效率方面取得了重要进展,但它们普遍存在一个共同的局限性:大多数方法仅聚焦于搜索过程的剪枝优化,而缺乏对最终匹配结果数量的有效控制。
因此,在实际应用中,当查询图结构较简单时,匹配结果可能会过于庞大,导致后续的排序和筛选成本大幅增加。在大规模数据场景下,这种方法的局限性尤为突出,其往往只能返回匹配的总数或随机选取部分结果,而难以直接获得最优的前 $k$ 个匹配子图。
\subsection{现有的Top-k密集子图挖掘问题}
在现有的Top-k密集子图挖掘(Top K Dense Subgraph Mining)研究中,密度的定义通常根据图中节点的度数或边权值进行区分~\cite{dsm-noweight-Bahmani-DBLP:journals/pvldb/BahmaniKV12,dsm-noweight-Balalau-DBLP:conf/wsdm/BalalauBCGS15,dsm-noweight-Bonchi-DBLP:journals/corr/abs-2007-01533,dsm-noweight-Dondi-DBLP:journals/corr/abs-2002-07695,dsm-noweight-Fang-DBLP:journals/pvldb/FangYCLL19,dsm-noweight-Gabert-DBLP:conf/wsdm/GabertPC21,dsm-noweight-Hu-DBLP:conf/cikm/HuWC17,dsm-noweight-Ma-DBLP:journals/pvldb/MaCLH22,dsm-noweight-Mathieu-DBLP:journals/corr/abs-2010-07794,dsm-noweight-McGregor-DBLP:journals/corr/McGregorTVV15,dsm-noweight-Rozenshtein-DBLP:journals/tkdd/RozenshteinTG17,dsm-noweight-Saha-DBLP:journals/corr/abs-2212-08820,dsm-noweight-Tsourakakis-DBLP:conf/kdd/TsourakakisBGGT13,dsm-noweight-Valari-DBLP:conf/ssdbm/ValariKP12,dsm-noweight-Zhao-DBLP:conf/icalip/ZhaoQYB14,dsm-weight-Angel-DBLP:journals/vldb/AngelKSSST14,dsm-weight-Ma-DBLP:conf/icde/MaHWLH17,dsm-weight-Muhammad-DBLP:conf/cikm/NasirGMG17},

论是在动态还是静态场景中,这些方法的核心问题都是如何定义和计算图的“密度”,并基于这一指标寻找最密集的子图。
具体而言,密集子图的挖掘问题大致可以分为两类:一类是基于节点度数定义的密度,另一类是基于边权值定义的密度。

\subsubsection{基于节点度数的密度定义}
大多数的密集子图发现问题中密度的定义都与节点的度数有关。例如Garbert等人\cite{dsm-noweight-Gabert-DBLP:conf/wsdm/GabertPC21}提出了全动态的核算法,该算法支持小团枚举和k-core维护,并将问题分解为维护特殊的超图以及k-core,该算法的改进同样适用于trusses(高密度子图)。
Bahmani等人\cite{dsm-noweight-Bahmani-DBLP:journals/pvldb/BahmaniKV12}提出了基于Mapreduce框架的图数据流中的Top-k最密集子图发现问题。在MapReduce模型\cite{csm-mapreduce-DBLP:journals/cacm/DeanG08}下,该方法能够高效地进行并行化处理,在处理大规模数据时有显著优势。
此外,Valari等人\cite{dsm-noweight-Valari-DBLP:conf/ssdbm/ValariKP12}针对动态大图中的Top-k密集子图问题,提出了精确和近似算法的研究,精确算法基于密度的上下限来减少了精确密度计算的次数,而近似算法通过在计算速度和结果准确性之间进行权衡来加速搜索。
这些方法在动态场景下表现尤为突出,能够实现在实时更新图数据时持续提供有效的Top-k密集子图,找到根据节点度数定义的密度最大的子图。

\subsubsection{基于边权重的密度定义}
另一类研究则关注于基于边权值的密度定义。
例如,Muhammad\cite{dsm-weight-Muhammad-DBLP:conf/cikm/NasirGMG17}等人研究了在滑动窗口模型下的Top-k最密子图算法,该算法返回的结果是边权值和较大的前k个子图,且子图之间不存在相交关系。
与传统的Top-k最密子图方法不同,该模型只更新影响图的有限区域,而不是对于整个图进行重新计算。
Albert等人\cite{dsm-weight-Angel-DBLP:journals/vldb/AngelKSSST14}研究了在边权值实时更新下维护密集子图的算法,在该方法中是根据给定的密度阈值划分密集子图,随着边权重的实时更新,子图的密度会随之变化,导致子图状态发生变化,该算法提出了关于单个边权重更新所引起的密度变化幅度的理论分析。

尽管这些方法在密度阈值下的Top-k密集子图挖掘中取得了显著进展,但它们往往未对子图的拓扑结构进行约束。
这意味着,尽管子图满足密度阈值,但在结构上可能表现出较大的异质性,从而使得这些方法在连续子图匹配(CSM-TopK)问题中的应用变得困难。
在CSM-TopK问题中,我们不仅需要关注子图的密度,还需要对子图的结构进行约束,以确保结果不仅在密度上满足要求,还在拓扑结构上具有一致性。
因此,现有基于节点度数或边权重定义密度的 Top-k 密集子图挖掘方法,通常无法直接适用于 CSM-TopK 问题,亟需在子图的结构约束方面进行深入研究。

综上所述,尽管现有的Top-k密集子图发现方法在密度计算和图数据处理方面取得了诸多进展,但它们的研究重点仍然集中在密度的定义与计算,对子图的结构特性考虑不足。
在需要精确结构匹配的应用场景(如 CSM-TopK)中,现有方法往往难以满足实际需求。
因此,未来的研究应探索密度与拓扑结构相结合的策略,以提出更加适用于特定应用场景的 Top-K 密集子图挖掘方法。
%无论密度的定义是否相同,其密集子图发现问题主要聚焦于发现符合密度阈值的最密子图,但对于子图的结构没有进行任何约束,因此最终得到的密集子图在结构上各不相同,很难将这些方法应用于CSM-Topk。
\subsection{静态Top-k密度优先级机制的子图匹配}
Gupta\cite{static-topk-Gupta-DBLP:conf/icde/GuptaGYCH14}是首个提出将密度与拓扑结构相结合的研究,并将这一问题定义为在信息网络中发现前$k$个有趣子图的问题。
在其研究中,Gupta首次提出了一种基于密度的子图匹配方法,密度定义为子图中边权的总和。
为了加速寻找前$k$个最有趣的子图,他提出了通过构建两种索引来减少搜索空间的策略:图拓扑结构索引和最大元路径索引。
这种方法的核心思想是通过捕捉图中节点的拓扑关系以及节点与特定元路径之间的关系来提高匹配效率。
%在该方法中,Gupta\cite{static-topk-Gupta-DBLP:conf/icde/GuptaGYCH14}通过构建两种索引来减少子图匹配的搜索空间,从而加速前k个有趣子图的发现:图拓扑结构索引和最大元路径索引。

具体来说,图拓扑结构索引通过为数据图中的每个节点存储沿特定元路径的所有d跳邻居的数量(其中$d \in \{1,\dots,D\}$)来捕捉节点之间的拓扑关系。
这种方式可以有效地简化匹配过程,减少不必要的计算,从而加速匹配过程。
最大元路径索引则记录每个节点在d跳特定元路径下的权值和的最大值,其中元路径是由路径中的节点标签组合而成。假设B是节点的平均邻居数,元路径的最大跳数为D,则该索引的时间复杂度为$O(BD)$。
因此,索引构建的时间复杂度为$O(|V_G|BD)$,空间复杂度为$O(|V_G|TD)$,其中T是节点标签的数量。
这种方法的最大优势在于,它通过有效的索引构建,可以利用索引的最大密度进行剪枝,减少了搜索空间,提高了匹配效率。
%由于时间和空间复杂度都呈指数增长,这使得索引构建的成本极高。
然而,这种方法的主要局限性在于,随着$d$的增大,索引构建的代价会显著增加,导致其在动态场景下的应用变得困难。
尤其是在需要实时更新图数据的情况下,Gupta提出的索引结构在计算和存储上都存在指数级的增长,限制了其在动态图中的适用性。
特别是在动态图的更新过程中,随着节点和边的频繁变化,现有索引结构需要频繁更新,这会导致索引构建的计算成本和存储成本都迅速攀升,难以适应实际应用中对实时性和效率的要求。


在此基础上,Chen\cite{static-topk-Chen-DBLP:journals/ijprai/ChenLCTL18}等人对Gupta的工作进行了优化,提出了通过压缩数据图来减少枚举的范围,同时通过基于一阶邻居筛选候选节点以降低图拓扑索引的成本。
在这一方法中,Chen等人首先通过压缩数据图来减少子图匹配时的枚举范围,并进一步通过筛选一阶邻居节点,降低了图拓扑索引的计算开销。
特别是在搜索阶段,Chen\cite{static-topk-Chen-DBLP:journals/ijprai/ChenLCTL18}等人选择从候选节点数最少的节点作为起始顶点,进行双向深度优先搜索,从而提高了搜索效率。
这一策略通过减少不必要的匹配计算,进一步加速了子图匹配过程。
然而,尽管这一方法在一定程度上优化了Gupta提出的算法,但其适用性仍然受到一定限制。
特别是,该方法仅适用于特定结构的查询图——路径类型的图,对于其他更复杂的查询图类型,其效果并不理想。

此外,尽管Chen等人提出的优化方法在某些场景下取得了一定的效果,但压缩数据图的策略在动态图中的应用仍然存在问题。
在动态图中,由于数据图是实时变化,压缩数据图需要辅助数据结构来应对其变化,而此策略并没有给出在动态图中的具体应对方式。
更重要的是,Chen等人依然保留了Gupta的最大元路径索引结构,若查询图的结构复杂,其元路径的深度会增加,从而导致该方法在时间和空间复杂度上依然呈指数级增长,难以满足动态图环境下的高效性需求。

%此外,压缩数据图的策略并不适用于动态图场景,且仍然保留了Gupta提出的最大元路径索引结构,因此其时空复杂度仍然是指数级的。

综上所述,尽管Gupta提出的密度与拓扑结构融合的定义与我们的问题定义相似,但其后续的研究工作都是针对静态图。
由于这些方法具有指数级的时间与空间复杂度,并且高度依赖特定的索引结构,在动态图或复杂查询图结构下,索引的更新和存储都会面临巨大的挑战。
因此,尽管这些方法在静态图中取得了一定的成功,但它们在动态图场景下的应用存在较大的局限性,无法有效适应实时更新和动态变化的需求,因而不能直接用于本文的课题。
\section{研究内容与贡献}
现有的连续子图匹配(CSM)工作主要集中于返回所有子图匹配结果,但在结果集过大的情况下,往往只能随机返回数量上限的子图,且缺乏对结果优先级的排序和筛选机制。
与此相关的经典问题是 Top-k 密集子图挖掘问题,但该问题通常不涉及对子图结构的约束。
此外,现有的基于密度优先级排序的Top-k机制,虽然能通过大量离线索引构建来优化性能,但其索引构建的时空复杂度高,难以适应动态场景的需求。
因此,在动态场景下进行按密度优先级排序的子图匹配,具有重要的现实意义。


因此,本研究的核心目标是提出一种新的方法,解决动态场景下融合密度和拓扑结构约束的Top-k连续子图匹配问题。
具体而言,我们通过动态图模型中引入的边权值来量化密度优先级排序,并在返回的匹配结果中,除了需要同查询图形成结构上的匹配关系外,还需要满足按照密度优先级排序进入前$k$个子图($k$为指定的参数)。具体的研究内容包括以下两个方面:
因此,本研究的核心目标是提出一种新的方法,以解决动态场景下融合密度和拓扑结构约束的Top-k连续子图匹配问题。
具体而言,我们通过动态图模型中引入的边权值来量化密度优先级排序,并在返回的匹配结果时,确保它们不仅满足与查询图的结构匹配,还能按照密度优先级排序进入前$k$个子图($k$为指定的参数)。具体的研究内容包括以下两个方面:

(1)插入更新下结合密度的增量查询搜索剪枝

      在插入更新的情况下,需要额外考虑密度约束来优化搜索剪枝策略。由于引入了密度优先级机制和匹配结果数量的限制,搜索的目标范围更窄。
      若继续采用传统的CSM的搜索策略,其效率不高。
      因此,需要新的搜索策略,重点研究如何结合密度优先级顺序以及结果数限定来高效剪枝,从而提升查询性能。
      在插入更新的情况下,需要额外考虑密度约束来优化搜索剪枝策略。由于引入了密度优先级机制和匹配结果数量的限制,搜索的目标范围更窄。
      若继续采用传统的CSM的搜索策略,其效率不高。
      因此,需要新的搜索策略,重点研究如何结合密度优先级顺序以及结果数限定来高效剪枝,从而提升查询性能。

(2)删除更新下的Top-k密度子图补充策略

   在删除更新发生时,维护的前k个结果可能会出现包含已删除边的子图,这些子图将因删除操作而失效,导致结果集的大小小于$k$。
   此时,需要从数据图中补充新的子图以维持结果集的完整性。
   然而,相比于插入更新,补充操作的挑战更大,因为插入仅影响增量部分,而补充则需要在全图范围内重新搜索合适的子图。
   因此,设计高效的补充策略是删除更新场景下的关键问题。
   此时,需要从数据图中补充新的子图以维持结果集的完整性。
   然而,相比于插入更新,补充操作的挑战更大,因为插入仅影响增量部分,而补充则需要在全图范围内重新搜索合适的子图。
   因此,设计高效的补充策略是删除更新场景下的关键问题。
   综上所述,CSM-TopK问题的核心挑战在于设计一种基于密度的高效索引结构以进行剪枝,并针对删除操作设计有效的补充策略。基于上述研究内容,本研究的贡献包括:
\begin{itemize}[label={\textbullet}]
    \item \textbf{CSM-TopK问题。}本研究首次提出了CSM-TopK问题,将TopK密度约束纳入动态加权图上的连续子图匹配问题,并证明了该问题是NP难的。
    \item \textbf{轻量级索引结构(MWstar)。}本文提供了CSM-TopK 计算框架,并设计了两种轻量级索引结构,分别为全局MWstar和局部MWstar,用于加速CSM-TopK问题的计算。在边的动态插入/删除操作中,MWstar索引能够维持线性空间和常数的更新时间。
    \item \textbf{图压缩技术。}本研究采用了一种基于查询的图压缩技术,通过该技术过滤掉不可能成为匹配的候选点,从而显著缩小初始数据图的规模。在此压缩图上应用 MWstar索引,使得在时间和空间上的性能更为高效,最终形成了本文的优化方案。
    \item \textbf{广泛的实验。}本文在真实世界的数据集上进行了广泛的实验,证实了我们的解决方案在在性能上明显优于对比解决方案,至少提高了两个数量级。
  \end{itemize}
\section{论文组织架构}
全文共分为五章,具体组织结构如下:

第 1 章\ 绪论。主要介绍了密度约束下的TopK连续子图匹配(CSM-TopK)研究背景以及研究意义,CSM-TopK的国内外研究现状以及本文的研究内容。

第 2 章 \ 相关知识。介绍本文研究的相关的理论及技术,主要包括图的基本概念、子图匹配以及连续子图匹配的基本概念和应用,最后介绍本课题的概念。

第 3 章\ 基于TopK密度剪枝的连续子图匹配问题。
首先给出 CSM-TopK 问题的基本定义,然后介绍本文提出的 CSM-TopK 基础框架。
接着,阐述基于密度剪枝策略的基线方案,并提供相应的伪代码以说明其实现过程。

第 4 章\ 轻量级索引MWStar。在密度约束下的TopK连续子图匹配问题的基础框架下,本章首先介绍轻量级索引结构MWStar的设计,随后提出了两种索引策略:全局 MWStar 和局部 MWStar,并通过理论分析证明了局部 MWStar 在密度上限方面的优势。
此外,本章还提出了针对数据图的图压缩策略,详细阐述了该策略的过滤机制以及压缩后的压缩图结构。
最后介绍了如何将图压缩策略与轻量级索引 MWStar 相结合。

第 5 章\ 实验设计与结果分析
本章基于第三和第四章提出的基线方法以及优化剪枝策略,在五个真实数据集上进行了大量实验,旨在验证和比较本文提出方法与其他对比方案在时空效率上的表现。实验结果表明,本文所提出的算法在效率上显著优于现有的解决方案。

结论。总结本文研究得出的结论以及说明本文工作的不足之处,并介绍了未来可进一步开展的研究工作。

\chapter{相关知识}

\section{连续子图匹配(CSM)的定义与概念}
\subsection{图的基本定义与概念}
图是一种比较复杂的数据结构,它研究数据元素之间的多对多的关系。它是由节点和边构成的数据结构,用于
表示任意两个元素之间的关系。图(Graph)是计算机科学中表示实体间关系的重要数据结构,定义为二元组
$G=(V,E)$,其中V是节点的结合,E是边的集合,即:
\begin{align*}
  V &= \{\ v_1, v_2, \dots, v_n\ \},\\
  E &\subseteq V \times V
\end{align*}

根据边的特性,图有多种的划分方式,根据边的方向性可分为有向图(图\ref{fig:example_noweight})与无向图(图\ref{fig:example_weight})。有向图中的边是有明确的方向的,每条边表示一种特定的方向,表示从起点指向终点的一条路径。如图\ref{fig:example_noweight}中所示$<v_1',v_2'>$,表示一条有向边,$v_1'$是起点,$v_2'$是终点。
而无向图表示边是没有方向性的,即两个相连的顶点是可以互相抵达的,因此无向图又可以称之为特殊的有向图,因为每条边都代表着一个双向可达关系。如图\ref{fig:example_weight}中所示$<v_1,v_2>$,表示一条无向边,其可以认为是由$<v_1,v_2>$和$<v_2,v_1>$两条有向边组成。
根据边上的权值,又可以分为有权图(图\ref{fig:example_weight})和无权图(图\ref{fig:example_noweight})。
无权图中的边只表示两个顶点之间的可达关系,没有携带额外的信息;但有权图中,其边带有权重,并且这些权重具有不同的含义,例如距离、费用、强度等信息。

\begin{figure}[h!]
    \def\wscorevone{0.48}
    \centering
        \begin{subfigure}[t]{\wscorevone\linewidth}
            \centering
            \resizebox{\linewidth}{!}
            {
                \includegraphics{\figs youxiangwuquan.pdf}
            }
            \caption{有向无权图}
            \label{fig:example_noweight}
        \end{subfigure}
        \hfill
        \begin{subfigure}[t]{\wscorevone\linewidth}
            \centering
            \resizebox{\linewidth}{!}
            {
                \includegraphics{\figs wuxiangyouquan.pdf}
            }
            \caption{无向有权图}
            \label{fig:example_weight}
        \end{subfigure}
        \label{fig:definition}
        \caption{图的常见分类}
    \end{figure}

图数据结构能够使用非常简洁的形式来表达复杂的数据。利用图这种数据结构,它能够很好的表达陈述句中的主谓宾,即以主语和宾语为对象,谓语作为两个对象之间的关联关系,这保证了它对于形式多样的复杂数据有非常强的表达能力。
具体而言,生活中的各式各样的信息都能够通过主谓宾的形式分解并构建图模型,更重要的是,生活中大多数构建的图模型的边都具有权值,表示特定的含义。例如在金融支付应用中,“用户A向用户B转账100”这一信息中,用户A和用户B构成两个顶点,“转账”就是两个顶点之间的关联边,“100”就是这条关联边上的权值,代表的是这次转账的金额。
综上所述,图模型在生活中的应用非常广泛,它能够清晰展示对象之间的关联关系,并且能够帮助数据分析人员更好的分析和挖掘数据。本文中主要以无向有权图作为分析对象,并且分析的算法同样适用于有向图。

\subsection{子图匹配的概念与应用}
子图匹配问题是图论中的经典问题之一。给定一个查询图和一个数据图,子图匹配的任务是找出数据图G中所有能够与查询图Q完全匹配的子图。具体而言,子图匹配涉及了以下的一些关键概念:
\begin{figure}[h!]
    \centering
    \resizebox{0.8\linewidth}{!}{
        \includegraphics{\figs e_subgraph_matching.pdf}
    }
    \caption{子图匹配示例}
    \label{fig:example_subgraph_matching}
\end{figure}
\begin{itemize}
    \item 子图匹配,也称为子图同构(Subgraph Isomorphism):即给定一个查询图Q以及一个数据图G,Q子图同构于G中的一个子图g,当且仅当存在一个双射函数,对于Q中的每一个节点,在g中都能到找到与之对应的节点。
    对于Q中的每一条边,在g中都能找到与其对应的边。
    即查询图Q和子图g之间存在一个一一对应的节点和边的映射关系,换句话说就是查询图的结构能够完全嵌入到数据图中。
    如图\ref{fig:example_subgraph_matching}a为查询图Q,图\ref{fig:example_subgraph_matching}b为数据图G。数据图中的子图$<v_1,v_2,v_3>$是查询图$<u_1,u_2,u_3>$的一个子图匹配结果,图中虚线表示他们的映射关系。
    \item 子图匹配任务就需要在数据图中找到所有的子图集合,集合中的所有子图都能与查询图能够构成子图同构的关系。
    \item 完全匹配(Full Match):完全匹配其实就是一次完整的子图同构,查询图中的所有节点和边都在数据图的某个子图中找到一一对应的映射。
    \item 部分匹配(Partial Match):部分匹配是指查询图中的一部分节点和边在数据图中找到了对应的关系,但不要求查询图的所有节点都能够找到映射。
\end{itemize}

子图匹配相关的经典算法主要分为两类,一类是基于深度一类为基于深度搜索加回溯的方式(Backtracking Search)\cite{sm-ullmann-DBLP:journals/jacm/Ullmann76},一类为基于广度优先的Multi-way Join方法\cite{sm-bfs-DBLP:conf/focs/AtseriasGM08}。

基于深度搜索加回溯的方式: 给一个查询图Q,首先定义节点被匹配的顺序,如$<u_1,u_2,u_3>$,即按照$u_1$,$u_2$,$u_3$的顺序进行匹配,如果当前状态匹配不了,则回溯;如果要找全部的解集,也得做回溯。其优点是可避免产生大量的中间结果,因采用深度优先,仅有递归调用栈的空间,没有什么中间结果。其缺点是难以并行执行,会有大量的递归开销。

基于广度优先的Multi-way Join方法:对于宽度优先的算法,实际关系数据库每次的Join就是宽度优先。其实就是不同候选解之间的Join操作,例如对于$<u_1,u_2,u_3>$的匹配顺序,每个节点在宽度优先搜索的时候都有对应的候选解,然后利用边的相邻关系做join操作,最终得到子图匹配的结果集合。
可以看到,对于所有的节点都需要求候选集合做join操作,因此基于这种方式比较容易做并行操作。本文算法框架中的Worst Case Optiomal Join\cite{sm-bfs-DBLP:conf/focs/AtseriasGM08}就属于第二种方式。

子图匹配问题在模式识别、社交网络分析、网络安全等方面都具有重要的应用价值。在生物信息学中,子图匹配用于识别分子结构中的相似模式;在社交网络中,子图匹配用于发现用户之间的潜在关系或者行为模式,在网络安全中,子图匹配用于检测恶意活动的特征或攻击模式。
但是由于子图匹配的判定是NP完全的,列举所有的子图匹配出现的位置是NP难的。这意味着子图匹配的计算难度会随着图的规模的增加而迅速增加。特别是在大规模的图数据集上,子图匹配问题需要采用高效的算法来剪枝。

\subsection{连续子图匹配的概念与应用}
连续子图匹配(CSM,Continuous Subgraph Matching)是在子图匹配的基础上,研究如何在不断变化的动态图中,保持子图匹配的有效性和准确性。与静态子图匹配不同,连续子图匹配主要涉及以下特点:
\begin{itemize}
   \item 连续子图匹配:在动态图中,图的结构随着时间不断变化,节点和边可能会被增加、删除。因此,连续子图匹配需要考虑如何在这种变化中有效地维护子图匹配结果,避免每次变化都从头开始计算。
   \item 增量更新:对于动态图,子图匹配不仅仅是对静态图进行一次匹配操作,而是需要根据图的增量更新,实时地更新匹配结果。增量更新的目标是尽量减少每次更新计算的代价。
   \item 匹配结果的更新与维护:连续子图匹配的核心问题之一是如何有效地维护和更新匹配结果。随着边的添加或删除,部分匹配可能会失效,需要重新计算。而另一方面,有些匹配可能依然有效,可以通过增量更新来保存计算结果。
\end{itemize}

因为在现实生活中,大部分数据图都不是一尘不变的,因此连续子图匹配被广泛应用于如社交网络分析、交通网络、金融欺诈检测等领域。社交网络中的用户关系是动态变化的,对社交网络中的用户子图进行连续匹配,可以识别潜在的社交圈子、群体或行为模式;
在交通网络中,路况和道路连接经常发生变化,实时监控交通状态并进行子图匹配,能够帮助分析交通流量、预测交通拥堵等;在金融网络中,交易关系和账户之间的连接不断变化,检测异常的金融行为或交易模式需要利用连续子图匹配技术。


\section{密度约束下TopK子图连续子图匹配的概念}
\begin{figure}[h!]
    \def\wscorevone{0.49}
    \centering
        \begin{subfigure}[t]{\wscorevone\linewidth}
            \centering
            \resizebox{\linewidth}{!}
            {
                \includegraphics{\figs e_attack_pattern.pdf}
            }
            \caption{子网攻击模式~\cite{static-topk-Gupta-DBLP:conf/icde/GuptaGYCH14}}
            \label{fig:example_attack_pattern}
        \end{subfigure}
        \hfill
        \begin{subfigure}[t]{\wscorevone\linewidth}
            \centering
            \resizebox{\linewidth}{!}
            {
                \includegraphics{\figs e_traffic_jam.pdf}
            }
            \caption{交通拥堵模式~\cite{traffic-graph-matching-DBLP:journals/pvldb/SongGCW14}}
            \label{fig:example_traffic_jam}
        \end{subfigure}
        \label{fig:definition}
        \caption{CSM-TopK示例}
    \end{figure}
本文研究的主要问题是在连续子图匹配问题的基础上,利用TopK密度剪枝策略,旨在有效获取前K个具有最高密度的子图匹配结果。
2014年,Manish Gupta等人首次提出了子图匹配中密度优先级的概念\cite{static-topk-Gupta-DBLP:conf/icde/GuptaGYCH14}。他们将子图的密度定义为其边权之和,即边权总和越大,子图的密度优先级越高。
然而,该研究主要聚焦于静态图中的密度优先级,且提出的静态图索引无法有效应用于动态场景。
至今,尚未有研究关注动态场景中TopK密度优先级机制对子图匹配结果的影响。

因此,本文提出并研究一个新问题,即在动态场景下如何实时计算并获取前K个具有最高密度的子图匹配结果。
这一问题在现实应用中具有广泛的应用前景。
例如,在通信网络中,不同攻击模式可能具有不同的分析优先级,尽管它们可能具有相同的结构模式。
图\ref{fig:example_attack_pattern}展示了通信网络中的攻击模式\cite{static-topk-Gupta-DBLP:conf/icde/GuptaGYCH14},其中高数据传输速率的模式应当具有更高的响应优先级。快速识别这些高数据传输速率的模式有助于网络管理员及时找到受损最严重的子网络。
同样,图\ref{fig:example_traffic_jam}展示了一种典型的交通拥堵模式\cite{traffic-graph-matching-DBLP:journals/pvldb/SongGCW14},在道路网络中,交通流量较高的模式通常意味着拥堵程度较高,交管系统需要迅速采取应对措施。上述两种模式的实际应用场景均为动态变化的,因此,基于动态图中的密度约束进行TopK连续子图匹配,其应用价值远高于静态场景中的相同问题。

\section{本章小结}
本章介绍了图的基本概念、子图匹配和连续子图匹配(CSM)。
首先,回顾了图的定义与分类,重点讨论了无向有权图,并介绍了图模型在实际应用中的广泛应用。
接着,阐述了子图匹配的基本概念,并介绍了基于深度优先搜索、回溯以及广度优先的多向联接方法等经典算法。
最后,本章探讨了连续子图匹配(CSM)的定义和基本概念,并介绍了其在社交网络、交通网络等领域的应用。
基于这些基础,本章提出了一个新的研究问题——在动态场景下如何实时计算并获取前K个具有最高密度的子图匹配结果。
\begin{summary}
	近年来,随着大规模图数据在各个领域的广泛应用,图计算技术逐渐成为研究的热点,尤其是在智能交通、社交网络、金融风控等领域中,如何高效处理图数据和解决实际问题的挑战变得愈发重要。
	子图匹配作为图计算中的关键问题,尤其是在复杂图结构中进行快速的子图匹配计算并获取所有的子图匹配结果,已成为研究的热点。
	由于动态图相比静态图具有更高的应用价值,因此连续子图匹配(CSM)问题更是学术界和工业的关注重点。

	尽管国内外已有多个优秀团队在连续子图匹配问题上进行深入研究,提出了多种优化策略并取得了显著成果,
	但现有的CSM方法普遍忽略了一个重要问题:在大规模数据图下,子图匹配的结果数目庞大,密度优先级机制在筛选有价值的子图匹配结果中起到至关重要的作用。
	此外,尽管已有研究在静态场景下提出了利用密度优先级筛选TopK子图的方法,但由于其索引结构的构建时间和空间复杂度呈指数级增长,离线构建时间长,无法有效应用于CSM问题。
	
	针对以上问题,本文提出了一种新的子图匹配问题——CSM-TopK(密度约束下的TopK连续子图匹配问题),并设计了一种基于密度剪枝的高效算法。本文的研究成果主要集中在以下几个方面:
		
		(1) 本文首先通过形式化定义子图匹配、密度等核心概念,建立CSM-TopK的基础计算框架,首次提出了在子图匹配算法的递归搜索过程中与密度优先级机制相结合的基线方法。提出了基于第$k$个子图匹配结果的密度上限,不再维护第$k$个之外的子图匹配结果。通过引入密度优先级机制,可以筛选出更有价值的匹配结果供数据分析人员进一步分析。

		(2) 为了更高效的解决CSM-TopK问题,本文设计了一种轻量级的星形索引结构。基于该索引结构扩展出两种关键索引——全局MWstar和局部MWstar索引,分别通过维护全局和局部匹配的密度上限,有效减少不满足密度约束的搜索空间,从而优化计算过程。
		首先,全局MWstar索引为子图匹配的递归搜索过程提供了一种粗粒度的密度上限$gBound$,显著减少了无效的搜索空间;其次,局部MWstar索引在全局索引的基础上进一步缩小候选集合,构建了更为紧凑的密度上限$lBound$,可以更早的发现无意义的搜索路径,减少部分子图匹配的递归深度。
		更重要的是,我们构建的全局和局部的星形索引能够保持常数的时间复杂度和线型的空间复杂度,既有效减少了计算时间,也降低了空间消耗,确保了在大规模图数据中的高效性。

		(3) 为了进一步提升算法的效率,本文引入了图压缩技术,通过NLF标签过滤策略构建了规模更小的压缩图,显著降低了计算和存储的负担。
		此外,压缩图在压缩原数据图的大小规模的基础上,继续维护MWstar索引,从而实现了比基线方法更严格的密度上限。该算法在处理动态数据更新时,能够保持较低的时间复杂度和空间复杂度,适应了大规模图数据中的复杂需求。
		
		(4) 本文所提出的算法在五个真实的数据集上进行多组循环测试,验证了所提出的密度剪枝算法的有效性。
		实验结果表明,我们的最终算法——基于压缩图上的MWstar索引剪枝算法,相较于其他已有的解决方案,能够显著提高查询效率,性能提升达到2至4个数量级。
		随着数据集规模的增大,其算法的性能提升更为显著。
		%此外,我们还对比了我们的方案中与其他方案的索引的构建时间以及空间开销,实验证明我们的索引的轻量级。
		此外,我们还对比了本文提出的四种优化算法之间的性能差异,验证了所提索引的轻量级特性。与基线方法相比,图压缩技术和MWstar索引剪枝策略在算法性能上提升了1至2个数量级。
		因此,我们可以得出结论,我们的算法在处理大规模图数据时,能够有效节约计算时间和空间资源。
	
	本文重点研究密度约束下的TopK连续子图匹配问题,旨在图数据中实时高效的查询匹配查询图的前k个密度最大的匹配结果。在研究的过程中,本文考虑到了动态场景下更新的复杂程度,基于动态场景提出了适应于动态场景的索引结构,大大提升了算法的时空效率。
	尽管本文在CSM-TopK问题的研究中取得了初步成果,但仍有许多方面可以进一步优化。未来的研究方向可以从以下几个方面进行扩展:

		(1)子图匹配的多样性研究:
		当前的CSM-TopK算法可能返回彼此之间重叠程度比较高的匹配结果,
		但在实际应用中,如何保证返回结果的多样性,是一个值得进一步研究的方向。
		未来的研究可以引入多样性约束,探索如何在密度约束下返回重叠程度更低的前$k$个匹配结果,以提高算法的实用性,尤其在社交网络分析和推荐系统中,能够更好地满足用户的个性化需求。

		(2)更多约束条件的引入:本文研究的主要集中于子图的密度优先级约束,但在实际应用中,可能还需考虑时间成本、空间成本等其他约束条件。
		未来的研究可以结合用户的个性化需求,将这些约束条件融入密度的计算公式中,通过加权融合不同约束条件的密度公式,从而进一步提升匹配结果的准确性和优化性。

		(3)硬件加速与算法优化:本文提出的优化算法都是基于软件层面的优化,而软件与硬件的协同优化更能够推动技术的进步与发展,因而在未来的研究中可以考虑将算法与现代硬件加速相结合。
		例如,利用图形处理单元(GPU)或现场可编程逻辑门阵列(FPGA)等硬件设备的并行计算能力,将子图匹配的搜索过程并行化处理,从根本上提升计算速度和算法效率。特别是在面对大规模的图数据时,硬件加速能够显著改善处理能力。

		(4)密度上限的进一步优化:在大规模稠密数据图中,一个查询图可能会产生亿万级别的子图匹配结果,如何设计高效的索引结构以得到更严格的密度上限,成为了优化CSM-TopK问题的关键。
		因此,未来研究可以在现有轻量级索引结构的基础上,进一步缩小匹配序列中的候选点的范围,以获得比当前方法更加严格的密度上限,提前剪枝无用扩展过程,减少了递归的深度,从而提高查询效率和可扩展性。

	综上所述,本文提出的CSM-TopK问题及其基于MWstar的密度剪枝算法,在子图匹配中取得了显著进展,尤其在动态大规模图数据处理中展现了较高的效率和可行性。
	尽管目前仍面临一些挑战,未来的研究可以从多样性研究、约束条件引入、硬件加速与算法优化、密度上限优化等多个方向进行深入探索,以促进该算法在实际应用中的广泛落地与发展。
\end{summary}

\bibliography{references}

\appendix
\chapter{读学位期间所发表的学术论文}

\begin{enumerate}[label={[\arabic*]},leftmargin=*,align=left]
    \item \textbf{Chuchu Gao}, Youhuan Li, Zhibang Yang, Xu Zhou. 
    \textsf{CSM-TopK: Continuous Subgraph Matching with TopK Density Constraints}. 
    In \textit{Proceedings of the 40th IEEE International Conference on Data Engineering (ICDE)}, 
    Utrecht, The Netherlands, May 13-16, 2024, pp. 3084-3097. IEEE, 2024. 
    DOI: \href{https://doi.org/10.1109/ICDE60146.2024.00239}{10.1109/ICDE60146.2024.00239}
    
    \item (专利)\textbf{周旭},\textbf{高楚楚},杨志邦,李友焕,杨圣洪,肖国庆,李肯立. 
    \textsf{一种基于动态加权图的Top-k密集子图匹配方法和系统}. 
    国家发明专利,申请号 202311708146.7
\end{enumerate}

\chapter{读学位期间所参加的科研项目}

\begin{enumerate}
    \item 行业知识图谱自动构建与生命周期管理技术,科大讯飞股份有限公司,项目批准号: 2021YFF0901002,研究年限:2021年12月-2024年11月。
\end{enumerate}


\backmatter
\begin{acknowledgements}
	不知不觉中,在湖南大学计算机科学与技术专业攻读硕士的岁月也即将画上句号。回首这三年,我的心态发生了很大的变化。
	研一刚入学时,我的心中充斥着焦虑和后悔。由于研一后互联网的就业步入了寒冬,看着本科人均大厂的同学,我心中有悔。
	同时,面对着未知的研究生生涯,我总是产生一阵又一阵的焦虑情绪,我害怕研究生三年虚度光阴,我害怕浑浑噩噩的度过而一事无成。
	但是,我遇到了很多很美好的人,他们治愈、关怀、鼓励着我;渐渐地,我变得越来越珍惜这段校园时光,三年一晃而过,而现在我的心中只有着感恩和不舍。
	在此,我希望以我最真挚的笔触,向一路给予我帮助的大家表达谢意。

	师者之光。首先,我要重心的感谢我的两位导师周旭老师和李友焕老师。很幸运,我的三年有两位这么优秀的老师共同指导。
	周旭老师给人的感觉就是平易近人、和蔼可亲,不管是学业上还是工作上,都给予着我很大的帮助。
	她教会了我“努力比天赋更加重要”,让我明白了,每日勤勤恳恳的努力,终将会有回报。
	而李友焕老师,则可以说是我科研以及人生道路上的引路人,不仅是教会了我科研的技巧,也教会了我很多为人处事之道。
	从本科毕设开始,李老师就一直亲力亲为,远程和我讨论课题,一谈就是一个小时。硕士期间,从科研选题的确定到具体算法的实现,
	以及论文的撰写,都离不开李老师的悉心指导。从每周多次的学术课题讨论,深夜中发消息解答我的疑问,科研遇到瓶颈时的鼓励以及帮助,
	论文中每句话的逐句斟酌,无不让我感觉科研的严谨和温度。抛开科研,在就业指导方面,李老师也像一束光一样照亮了我前方的迷惘。他会花时间一句句修改我的简历,
	会时常关心我的就业进展,也会关心我的心情和状态。最终,我能拿到不错的工作,这也离不开李老师的一直以来对我的帮助。
	总之,非常感谢两位老师在科研学习、项目实验中的倾囊相授,愿您们学术长青,桃李满天。
	
	同窗之谊。非常感谢一路上陪伴共同学习生活的418和547的小伙伴们,你们的陪伴让我觉得科研枯燥的日子中也有了一些乐趣和珍贵。
	感谢我的好同门们,和你们互相的倾诉、聊天,分享着生活中的趣事,让我觉得研究生的科研之路都不再是孤身奋战,而是互助友爱。
	这三年来,大家每天在实验室一起学习、干饭、爬山、玩桌游、探索了湖大周围好多好吃的小店。
	非常感谢研究生三年有你们陪我一起疯狂,希望现在短暂的离别,只是为了未来我们更好的相聚。
	
	同室之情。非常感谢本科605宿舍的小伙伴们,我们的同宿舍的情谊一直延续至今。感谢你们在我压力大的时候给予我的鼓励,
	在我委屈的时候也一直站在我的角度上安慰我。虽然我们大家因为读研和工作各奔东西,但是大家每天都会在群上分享自己每天琐事,
	就好像回到了宿舍半夜我们仍然叽叽喳喳,熄灯长谈的时候。希望你们未来都能在各自的领域里闪闪发光。
	
	相知相守。在此特别感谢我的男朋友。异地给我们的感情带来的重大的挑战,但是你总是在无数个节假日和周末义无反顾的奔向我。
	每当我遇到心情不好的时候,总是把自己的坏情绪第一时间的反馈给你。无数的日夜,你总是不厌其烦的听我诉说着生活中的喜怒哀乐。
	祝愿你在未来的日子里学术坦途,而我们继续相伴前行。

	父母之恩。在此要感谢我的家人们,他们成为了我坚强的后盾,让我能心无旁骛的做我想做的事情。虽然他们不懂我的研究生要做什么,
	但是也总是会默默倾听我的牢骚,偷偷了解我的研究方向,特别是我的弟弟妹妹,在我心情不好的时候,总是第一时间“使相”,用力搞笑逗我开心。
	希望你们都能平安健安,希望以后的我能成为你们的后盾。

	不负韶华。最后,我要感谢一直没有放弃的自己。感谢自己的三年没有虚度光阴,感谢无数个挑灯夜战的日子,感谢无数次想要放弃却又继续坚持的时刻。
	“宝剑锋从磨砺出,梅花香自苦寒”,愿自己能成为那一把宝剑,那一缕寒梅,不惧磨砺,不畏严寒。

	麓山巍巍,湘水汤汤。行文最后,谨谢母校湖南大学,我将时刻谨记“实事求是,敢为人先”的校训,
	未来我将以务实创新的态度投身技术领域,不负湖大所育。

\end{acknowledgements}


\end{document}
