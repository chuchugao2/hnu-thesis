\begin{summary}
	近年来,随着大规模图数据在各个领域的广泛应用,图计算技术逐渐成为研究的热点,尤其是在智能交通、社交网络、金融风控等领域中,如何高效处理图数据和解决实际问题的挑战变得愈发重要。
	子图匹配作为图计算中的关键问题,尤其是在复杂图结构中进行快速的子图匹配计算并获取所有的子图匹配结果,已成为研究的热点。
	由于动态图相比静态图具有更高的应用价值,因此连续子图匹配(CSM)问题更是学术界和工业的关注重点。

	尽管国内外已有多个优秀团队在连续子图匹配问题上进行深入研究,提出了多种优化策略并取得了显著成果,
	但现有的CSM方法普遍忽略了一个重要问题:在大规模数据图下,子图匹配的结果数目庞大,密度优先级机制在筛选有价值的子图匹配结果中起到至关重要的作用。
	此外,尽管已有研究在静态场景下提出了利用密度优先级筛选TopK子图的方法,但由于其索引结构的构建时间和空间复杂度呈指数级增长,离线构建时间长,无法有效应用于CSM问题。
	
	针对以上问题,本文提出了一种新的子图匹配问题——CSM-TopK(密度约束下的TopK连续子图匹配问题),并设计了一种基于密度剪枝的高效算法。本文的研究成果主要集中在以下几个方面:
		
		(1) 本文首先通过形式化定义子图匹配、密度等核心概念,建立CSM-TopK的基础计算框架,首次提出了在子图匹配算法的递归搜索过程中与密度优先级机制相结合的基线方法。提出了基于第$k$个子图匹配结果的密度上限,不再维护第$k$个之外的子图匹配结果。通过引入密度优先级机制,可以筛选出更有价值的匹配结果供数据分析人员进一步分析。

		(2) 为了更高效的解决CSM-TopK问题,本文设计了一种轻量级的星形索引结构。基于该索引结构扩展出两种关键索引——全局MWstar和局部MWstar索引,分别通过维护全局和局部匹配的密度上限,有效减少不满足密度约束的搜索空间,从而优化计算过程。
		首先,全局MWstar索引为子图匹配的递归搜索过程提供了一种粗粒度的密度上限$gBound$,显著减少了无效的搜索空间;其次,局部MWstar索引在全局索引的基础上进一步缩小候选集合,构建了更为紧凑的密度上限$lBound$,可以更早的发现无意义的搜索路径,减少部分子图匹配的递归深度。
		更重要的是,我们构建的全局和局部的星形索引能够保持常数的时间复杂度和线型的空间复杂度,既有效减少了计算时间,也降低了空间消耗,确保了在大规模图数据中的高效性。

		(3) 为了进一步提升算法的效率,本文引入了图压缩技术,通过NLF标签过滤策略构建了规模更小的压缩图,显著降低了计算和存储的负担。
		此外,压缩图在压缩原数据图的大小规模的基础上,继续维护MWstar索引,从而实现了比基线方法更严格的密度上限。该算法在处理动态数据更新时,能够保持较低的时间复杂度和空间复杂度,适应了大规模图数据中的复杂需求。
		
		(4) 本文所提出的算法在五个真实的数据集上进行多组循环测试,验证了所提出的密度剪枝算法的有效性。
		实验结果表明,我们的最终算法——基于压缩图上的MWstar索引剪枝算法,相较于其他已有的解决方案,能够显著提高查询效率,性能提升达到2至4个数量级。
		随着数据集规模的增大,其算法的性能提升更为显著。
		%此外,我们还对比了我们的方案中与其他方案的索引的构建时间以及空间开销,实验证明我们的索引的轻量级。
		此外,我们还对比了本文提出的四种优化算法之间的性能差异,验证了所提索引的轻量级特性。与基线方法相比,图压缩技术和MWstar索引剪枝策略在算法性能上提升了1至2个数量级。
		因此,我们可以得出结论,我们的算法在处理大规模图数据时,能够有效节约计算时间和空间资源。
	
	本文重点研究密度约束下的TopK连续子图匹配问题,旨在图数据中实时高效的查询匹配查询图的前k个密度最大的匹配结果。在研究的过程中,本文考虑到了动态场景下更新的复杂程度,基于动态场景提出了适应于动态场景的索引结构,大大提升了算法的时空效率。
	尽管本文在CSM-TopK问题的研究中取得了初步成果,但仍有许多方面可以进一步优化。未来的研究方向可以从以下几个方面进行扩展:

		(1)子图匹配的多样性研究:
		当前的CSM-TopK算法可能返回彼此之间重叠程度比较高的匹配结果,
		但在实际应用中,如何保证返回结果的多样性,是一个值得进一步研究的方向。
		未来的研究可以引入多样性约束,探索如何在密度约束下返回重叠程度更低的前$k$个匹配结果,以提高算法的实用性,尤其在社交网络分析和推荐系统中,能够更好地满足用户的个性化需求。

		(2)更多约束条件的引入:本文研究的主要集中于子图的密度优先级约束,但在实际应用中,可能还需考虑时间成本、空间成本等其他约束条件。
		未来的研究可以结合用户的个性化需求,将这些约束条件融入密度的计算公式中,通过加权融合不同约束条件的密度公式,从而进一步提升匹配结果的准确性和优化性。

		(3)硬件加速与算法优化:本文提出的优化算法都是基于软件层面的优化,而软件与硬件的协同优化更能够推动技术的进步与发展,因而在未来的研究中可以考虑将算法与现代硬件加速相结合。
		例如,利用图形处理单元(GPU)或现场可编程逻辑门阵列(FPGA)等硬件设备的并行计算能力,将子图匹配的搜索过程并行化处理,从根本上提升计算速度和算法效率。特别是在面对大规模的图数据时,硬件加速能够显著改善处理能力。

		(4)密度上限的进一步优化:在大规模稠密数据图中,一个查询图可能会产生亿万级别的子图匹配结果,如何设计高效的索引结构以得到更严格的密度上限,成为了优化CSM-TopK问题的关键。
		因此,未来研究可以在现有轻量级索引结构的基础上,进一步缩小匹配序列中的候选点的范围,以获得比当前方法更加严格的密度上限,提前剪枝无用扩展过程,减少了递归的深度,从而提高查询效率和可扩展性。

	综上所述,本文提出的CSM-TopK问题及其基于MWstar的密度剪枝算法,在子图匹配中取得了显著进展,尤其在动态大规模图数据处理中展现了较高的效率和可行性。
	尽管目前仍面临一些挑战,未来的研究可以从多样性研究、约束条件引入、硬件加速与算法优化、密度上限优化等多个方向进行深入探索,以促进该算法在实际应用中的广泛落地与发展。
\end{summary}
