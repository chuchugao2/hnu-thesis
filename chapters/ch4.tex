\chapter{基于压缩图的轻量级索引MWStar}
\label{ch4}
\section{MWSTAR:CSM-TOPK的轻量级索引}
本章讨论提出的全局和局部MWstar索引。
首先,在第\ref{mwstar:star_subquery}节中,我们将讨论星形子结构和部分匹配。
接着,在第\ref{mwstar:global}节中,我们将介绍第一个全局MWstar版本,它维护每个特定星形子结构的所有部分匹配的最大权重。
在第\ref{mwstar:local}节中,我们将把这个初步版本优化为局部版本,以实现更好的基于密度的剪枝能力。
最后,我们结合全局和局部MWstar,构建最终的密度上限函数。

\begin{figure}[h!]
\def\wscorevone{0.48}
\centering
\begin{subfigure}[t]{\wscorevone\linewidth}
\centering
\resizebox{\linewidth}{!}
{
\includegraphics{\csmfolder e_qstar.pdf}
}
\caption{Qstar示例}
\label{fig:qstar}
\end{subfigure}
\begin{subfigure}[t]{\wscorevone\linewidth}
\centering
\resizebox{\linewidth}{!}
{
\includegraphics{\csmfolder e_dstar.pdf}
}
\caption{Dstar示例}
\label{fig:dstar}
\end{subfigure}
\caption{MWstar在查询图和数据图中的示例}
\label{fig:exp:mwstar}
\end{figure}

\subsection{星形结构子图与部分匹配}
\label{mwstar:star_subquery}

\begin{definition}[星形结构子图]\label{def:star-subquery}
对于查询$Q$和匹配顺序$\Phi=\{u_1, u_2, \cdots, u_{|Q|} \}$,我们为每个查询顶点$u_i$定义其在$\Phi$下的星形结构子图$Qstar_{\Phi}(u_i)$,
包括顶点$LN(u_i) \cup {u_i}$以及连接$u_i$与其左邻居的边,即:$V_{Qstar_{\Phi}(u_i)} = LN(u_i) \cup \{u_i\}$,$E_{Qstar_{\Phi}(u_i)} = \{(u_i, u_j)\ |\ u_j\in LN(u_i)\}$。
当上下文清晰时,我们可以简写为$Qstar(i)$表示相应的子图。
\end{definition}

图\ref{fig:qstar}展示了运行示例中,匹配顺序为\{$u_1$, $u_2$, $u_3$, $u_4$\}时的$Qstar(3)$和$Qstar(4)$。

考虑一个部分匹配$g^i=\{v_1, v_2, \cdots, v_i\}$,它匹配查询$q^i=\{u_1, u_2, \cdots, u_{i}\}$在$\Phi$下的情况。
对于查询顶点$u_{i^\prime}$($1 \leq i^\prime \leq i$),
假设$Qstar(i^\prime) = \{u_{x_1}, u_{x_2}, \cdots, u_{x_j}\}$($j = |LN(u_{i^\prime})|$)。
对于$g^i$的子图$g^\prime = {v_{x_1}, v_{x_2}, \cdots, v_{x_j}}$(即$g^\prime \subseteq g^i$),如果$g^\prime$匹配$Qstar(i^\prime)$,且每个$v_{x_{j^\prime}}$映射到$u_{x_{j^\prime}}$($1 \leq i^\prime \leq i$),则我们称$g^\prime$为$g^i$中在$u_{i^\prime}$下的星形部分匹配,记作$Dstar_{\Phi}(g^i, i^\prime)$。
当上下文清晰时,我们也可以简写为$Dstar(g^i, i^\prime)$。
图\ref{fig:dstar}展示了匹配$g={v_1, v_2, v_8, v_{1001}}$时的$Dstar(g, 3)$和$Dstar(g, 4)$。

\subsection{全局MWstar索引}
\label{mwstar:global}

接下来我们讨论全局MWstar索引。
对于$Q$的一个子查询$q^i$,在$\Phi$下,$q^i$可以被分解为一系列的星型子图$Qstar(i^\prime)$($1 \leq i^\prime \leq i$),即:

\[
	q^i = Qstar(1)\cup Qstar(2) \cdots \cup Qstar(i)
\]

实际上,对于任何两个查询顶点$u_{i_1}$和$u_{i_2}$($i_1 \neq i_2$),根据星形子图的定义,有$E_{Qstar(i_1)} \cap E_{Qstar(i_2)} = \emptyset$。
通过这种方式,对于匹配$q^i$的部分匹配$g^i$,我们可以相应地将$g^i$分解为:

\begin{equation} \label{equation:dstar-decompose}
    g^i = Dstar(g^i, 1)\cup Dstar(g^i, 2) \cdots \cup Dstar(g^i, i)
\end{equation}

其中$Dstar(g^i, i^\prime)$匹配$Qstar(i^\prime)$($1 \leq i^\prime \leq i$),并且对于$1 \leq i_1 < i_2 \leq i$,$E_{Dstar(g^i, i_1)} \cap E_{Dstar(g^i, i_2)} = \emptyset$。

通过对$g^i$的分解,我们得到了如下公式:
\[
	den(g^i) = \sum\nolimits_{1\leq i^\prime\leq i} \left(den(Dstar(g^i,i^\prime))\right)
\] 

我们使用$den(Qstar(j))$来表示$Qstar(j)$的最大匹配权重,即:
\[
	den(Qstar(j)) = \max\nolimits_{g^\prime \in A_{G_t, Qstar(j)}} \left(den(g^\prime)\right)
\]


因此,我们有以下定理,用于设计基于全局MWstar的剪枝策略。
\begin{theorem} \label{theorem:global-mwstar}
假设$g^i$是$\Phi$下与$q^i$匹配的部分匹配,并且$g_{max}$是包含$g^i$的最大密度的完全匹配。那么,我们可以得出结论:
\[
	den(g_{max}) \leq den(g^i) + \sum\nolimits_{j=i+1}^{|V_Q|} \left(den(Qstar(j))\right)
\]

\end{theorem}

\begin{proof}
我们将$g_{max}$分解为:
\[
	g_{max} =   \bigcup\nolimits_{j=1}^{|V_Q|} Dstar(g_{max}, j)
\]
然后:
\begin{equation*}
    \begin{aligned}
      den(g_{max}) & =  \sum\nolimits_{j=1}^{|V_Q|} \left( den(Dstar(g_{max}, j)) \right)  \\
        & =  den(g^i) + \sum\nolimits_{j=i+1}^{|V_Q|} \left( den(Dstar(g_{max}, j)) \right) \\
        & \leq den(g^i) + \sum\nolimits_{j=i+1}^{|V_Q|} \left( den(Qstar(j)) \right) 
    \end{aligned}
  \end{equation*}
  \end{proof}

接下来我们给出相应密度的上界,以及利用密度上界进行递归搜索的过程。


\begin{algorithm}[h!]
    \small
    \caption{带$gBound$的递归搜索}
    \label{alg:find-dense-matches-gBound}
    \SetKwBlock{iFunc}{Procedure}{End Procedure}
    \iFunc(FindDenseMatches-Global{(}$G_t, Q, \Phi, g^i, A, k${)})
    {
    设$g_{min}$为$A$中权重最小的匹配 \
    \If{$|g^i|=|V_Q|$}{
    \If{$|A|<k \lor den(g_{min}) < den(g^i)$ } {
    将$g^i$添加到$A$中. \\
    从$A$中移除$g_{min}$ if $|A|>k$\\
    }
    \Return
    }
    \If{$|A|=k \land gBound(g^i)\leq den(g_{min})$}{\label{code:B-basic-begin}
    \Return \label{code:B-basic-end}
    }
    执行算法\ref{alg:find-dense-matches}中第\ref{code:base-extension:begin}至第\ref{code:base-extension:end}行相同的操作,扩展$g^i$为一系列的$g^{i+1}$以进行进一步搜索 \\
    \Return
    }
\end{algorithm}

\textbf{密度的上界。} 根据定理 \ref{theorem:global-mwstar},对于每一个部分匹配 $g^i$,我们可以得到包含 $g^i$ 的完整匹配的密度上界,记作 $gBound_{G_t, Q, \Phi}(g^i)$,它是对应星形子图下部分匹配的最大密度:
  \begin{equation*}
	\begin{multlined}
	 gBound_{G_t, Q, \Phi}(g^i) =  \\ den(g^i) + \sum\nolimits_{i+1\leq j\leq |V_Q|} \left(den(Qstar(j))\right)
	\end{multlined}
  \end{equation*}
  在上下文明确时,我们也可以简写为 $gBound(g^i)$。
  
  \textbf{基于上界的递归搜索。} 在我们的搜索过程中,对于每个部分匹配 $g^i$,如果 $gBound(g^i)$ 无法在已存在的 $G_t$ 匹配的密度排名中位列前 $k$,我们可以安全地剪枝掉 $g^i$(参见算法 \ref{alg:find-dense-matches-gBound}  Lines \ref{code:B-basic-begin}-\ref{code:B-basic-end})。
  
  对于每一种匹配顺序 $\Phi$,我们可以 $Qstar(i)$($1 \leq i \leq |V_Q|$)下维护相应的星形结构子图的最大密度,这正是全局 MWstar。显然,全局 MWstar 的空间开销是 $O(|E_Q|*|V_Q|)$,这是可以忽略不计的。
  \begin{algorithm}[H]
    \scriptsize
    \caption{全局 MWstar 维护}
    \label{alg:base-mwstar:update}
    \KwIn{ $G_t$, $Q$ 和 $o_t={op, v_1, v_2}$}
    \KwOut{更新后的全局 MWstar}
    \If{$op = +$}{
      \ForEach{$(u_{i_1},u_{i_2})\in E_{Q}$ 匹配 $(v_1, v_2)$} {
          \ForEach{匹配顺序 $\Phi$}{
              \If{$u_{i_1}\in LN(u_{i_2})$}{
                  UpdateDensity($v_{2}$, $u_{i_2}$,$\Phi$) \\
              }
              \uElse { 
                  UpdateDensity($v_{1}$, $u_{i_1}$,$\Phi$) \\
              }
          }
      }
  } % If  
  \If{$op=-$}{
          设 $g^\prime$ 表示 $A_{G_t, Qstar(i)}$ 中标记为最密集的部分匹配(即 $den(g)=den(Qstar(i))$) \label{code:record-denest} \\
          \If{$(v_1,v_2)\notin E_{g^\prime}$}{ \label{code:not-hit-return:begin}
              \Return \label{code:not-hit-return:end}
          }   
         \ForEach{$v\in V_{G_t}$}{
                 \ForEach{$u\in V_Q$}{
                     \If{$L(v) = L(u)$}{
                         \ForEach{匹配顺序 $\Phi$}{
                             updateDensity($v$, $u$, $\Phi$)
                         }
                     }
                 }
         }
       }
  
  \If{$op = +$}{
      \ForEach{$(u_{i_1},u_{i_2})\in E_{Q}$ 匹配 $(v_1, v_2)$} {
          \ForEach{匹配顺序 $\Phi$}{
              \If{$u_{i_1}\in LN(u_{i_2})$}{
                  UpdateDensity($v_{2}$, $u_{i_2}$,$\Phi$) \\
              }
              \uElse { 
                  UpdateDensity($v_{1}$, $u_{i_1}$,$\Phi$) \\
              }
          }
      }
  } % If  
  
  \SetKwFunction{FupdateMWstar}{updateDensity}
  \SetKwProg{Fn}{Function}{:}{\KwRet}
  \Fn{\FupdateMWstar{$u_{i}$,$v_{j}$,$\Phi$}}{
       /* 为每个 $v\in V_{G_t}$ 保持其邻接节点和对应边权重的排序是很简单的 */   \label{code:basic-den-update:begin}  \label{code:basic-sorted:begin}\\
       假设 $LN(u_i) =$ \{$u_{i_1}$, $u_{i_2}$, $\cdots$, $u_{i_x}$\} \\
       遍历 $N_{G_t}(v_j)$ 一次,计算出 $x$ 个不同的顶点 {$v_{j_1}$, $v_{j_2}$, $\cdots, v_{j_x}$}
       其中 $L(v_{j_{x^\prime}}) = L(u_{i_{x^\prime}})$ ($1\leq x^\prime\leq x$)   \label{code:basic-sorted:end} \\
       \If{这 $x$ 个顶点存在}{
       令 $g^\prime$ 表示由边从 $v_j$ 到 \{$v_{j_1}$, $v_{j_2}$, $\cdots, v_{j_x}$\} 组成的子图 \\
       \If{$den(g^\prime)>den(Qstar(i))$}{
       $den(Qstar(u_i))$ = $den(g^\prime)$ \\
       将 $g^\prime$ 标记为 $Qstar(i)$ 的最密匹配
       }
       }
       \uElse{
       $den(Qstar(u_i)) = 0$; \label{code:basic-den-update:end}
       }
       }
\end{algorithm}
  
     算法 \ref{alg:base-mwstar:update} 通过更新 $o_t=(op, v_1, v_2)$ 来维护全局 MWstar,其中 $(v_1, v_2)$ 匹配 $(u_{i_1}, u_{i_2})$。
     
     如果 $o_t$ 是插入操作,
     那么对于每个匹配顺序 $\Phi$,如果 $u_{i_1}$ $\in LN(u_{i_2})$(我们省略对称情况的讨论),
     我们可以计算所有与 $Qstar(i_2)$匹配的所有新的星形结构的部分匹配,其中 $(v_1, v_2)$ 匹配 $(u_{i_1}, u_{i_2})$。
     且如果 $g^\prime$ 是其中最密的部分匹配,当 $den(g^\prime) > den(Qstar(i))$ 时,我们更新 $den(Qstar(i))$。
     
     如果 $N_{G_t}(v_2)$ 按邻接标签和对应边权重排序的话(见算法 \ref{alg:base-mwstar:update} 中 Lines\ref{code:basic-sorted:begin}-\ref{code:basic-sorted:end}),
     则使用$v_2$ 匹配 $u_{i_2}$来计算这样的 $g^\prime$,只需要 $O(|N_{G_t}(v_2)|)$ 的时间。
     由于存在 $|E_Q|$ 种匹配顺序,边插入时全局 MWstar 的维护开销为 $O(|E_Q|*d_t)$,其中 $d_t$ 是 $G_t$ 的平均度数。
     
     如果 $o_t$ 是删除操作,
     则有可能删除的 $(v_1, v_2)$ 会导致 $den(Qstar(i))$ 过期。
     一种简单的策略是为每次删除操作重新计算 $den(Qstar(i))$,然而这种方法效率很低,因为这会对整个 $G_t$ 进行扫描,开销为 $O(|V_{G_t}|d_t)$,即 $O(E_{G_t})$,这与数据边的数量成线性关系。
     
     实际上,我们可以始终记录 $A_{G_t, Qstar(i)}$ 中的部分匹配 $g^\prime$,它的密度为$den(Qstar(i))$(见算法 \ref{alg:base-mwstar:update} Line \ref{code:record-denest})。
     因此,当 $(v_1, v_2) \notin E_{g^\prime}$ 时,我们可以避免重新计算 $den(Qstar(i))$ 的高开销(见算法 \ref{alg:base-mwstar:update}  Lines \ref{code:not-hit-return:begin}-\ref{code:not-hit-return:end})。
     通常,重新计算的概率是 $|E_{g^\prime}|/|E_{G_t}|$,即 $|LN(u_{i_2})|/|E_{G_t}|$。
     因此,重新计算的平均开销是 $O(|E_{G_t}|)*(|LN(u_{i_2})|/|E_{G_t}|)$,即 $O(|LN(u_{i_2})|)$。
     由于存在 $|E_Q|$ 种匹配顺序,每次删除操作的平均开销为 $O(|E_Q|*|LN(u_{i_2})|)$。
     
     我们可以看到,我们的全局 MWstar以恒定的空间和几乎恒定的更新时间为代价,为每个部分匹配 $g^i$ 提供了 $gBound(g^i)$ 作为上界,以此为基础进行剪枝,减少了不必要的搜索,从而大大提高了搜索效率。
     这个MWstar是我们方案的一个简单版本,我们将在下面的\ref{mwstar:local}节中将这个简单的MWstar优化为局部版本,在此章节我们设计了一个面向CSM-TopK的搜索,其密度上限lBound比gBound更紧凑。

\subsection{局部MWstar索引}
\label{mwstar:local}
我们设计了一种局部MWstar(Local MWstar),它可以与全局 MWstar 结合使用,从而构建一个更紧致的上界。
我们用 $lBound(g^i)$ 表示部分匹配 $g^i$ 的新上界,该上界比全局 MWstar 的上界更加紧密。
接下来,我们首先介绍局部 MWstar 索引,并在此基础上提出一种面向 CSM-TopK 查询的搜索策略,该策略利用 $lBound$ 函数进行优化。

局部 MWstar 是全局 MWstar 的扩展。
在全局 MWstar 中,对于每种匹配顺序 $\Phi$,我们维护不同子查询 $Qstar(i)$($1\leq i\leq |V_Q|$)下星形结构部分匹配的最大权重。
而在局部 MWstar 中,我们针对每个特定的数据顶点维护类似的最大权重分布。

具体而言,给定匹配顺序 $\Phi$ 和子查询 $Qstar(i)$,对于每个数据图中的顶点 $v$,我们维护所有匹配 $Qstar(i)$ 并且 $v$ 映射到 $u_i$ 的部分匹配 $g^\prime$ 的最大密度。用 $den(Qstar(i), v)$ 表示该最大密度。
例如,在图 \ref{fig:csm-topk} 中的 运行示例 里,考虑 $v_5$ 和相应的子查询 $Qstar(3)$(如图 \ref{fig:qstar} 所示)。存在两个子图,匹配 $Qstar(3)=\{u_1, u_2, u_3\}$,其中 $v_5$ 映射到 $u_3$:

     (1)$g_1^\prime=\{v_1, v_2, v_5\}$,密度 $den(g_1^\prime) = 40$

     (2)$g_2^\prime=\{v_1, v_3, v_5\}$,密度 $den(g_2^\prime) = 50$。

由于 $den(g_1^\prime)= 40 < den(g_2^\prime)=50$,因此我们设定 $den(Qstar(3), v_5) = 50$。

此外,对于具有标签 $L(u_i)$的顶点集合 $V^\prime$,我们可以使用$den(Qstar(i), V^\prime)$来表示任何$v\in V^\prime$的最大$den(Qstar(i), v)$,即:
\begin{equation}\label{equation:den-qstar-v}
	den(Qstar(i), V^\prime) = \max\nolimits_{v\in V^\prime}\left(den(Qstar(i), v)\right)
\end{equation}


\textbf{基于 $lBound$ 的 CSM-TopK 搜索策略}

接下来我们用lBound来讨论我们面向CSM-TopK的搜索策略,在我们的搜索策略中,$lBound$ 由局部 MWstar 和全局 MWstar 共同决定。
对于每个部分匹配 $g^i$,我们始终为每个尚未匹配的查询顶点维护一个局部候选集(Local Candidate Set)。


\begin{definition} [局部候选集] \label{def:local-candidate-set}
    %Consider a graph $G_t$, query $Q$, and 
    假设部分匹配 $g^i=\{v_1, v_2, \dots, v_i\}$ 匹配 $q^i=\{u_1, u_2, \dots, u_i\}$。
    %under $\Phi=$ \{$u_1, u_2$, $\cdots, u_{|V_Q|}$\} ($i<|V_Q|$).
    设 $f$ 表示  $V_{q^i}$  到 $V_{g^i}$ 之间的双射映射,即 $f(u_{i^\prime})$ $=v_{i^\prime}$ ($1\leq i^\prime\leq i$)。
    则每个尚未匹配的查询顶点 $u_j \in V_Q\setminus V_{q^i}$(即 $j>i$)的 局部候选集 $C_{g^i}(u_j)$ 定义如下:
    \begin{equation} \label{equation:local-candidate-set}
    \begin{multlined}
    C_{g^i}(u_j) = \bigcap\nolimits_{u_{i^\prime}\in LN(u_j)\cap V_{q^i}} \left(N_{G_t}^{L(u_j)}\left(f(u_{i^\prime}\right)\right) \setminus V_{g^i}
    \end{multlined}
    \end{equation}
    其中,$N_{G_t}^{L(u_j)}(f(u_{i^\prime}))$ 表示数据图中 $f(u_{i^\prime})$(即 $v_{i^\prime}$)的所有邻居集合,且这些邻居的标签必须为 $L(u_j)$。
\end{definition}


根据定义 \ref{def:local-candidate-set},我们可以看到$C_{g^i}(u_{i+1})$ 恰好是查询顶点 $u_{i+1}$ 最大的可行候选集,它将用于将 $g^i$ 扩展为一系列新的 $g^{i+1}$,即:

$g^{i+1}\in$ $g^i\times C_{g^i}(u_{i+1})$

此外,我们提出以下引理 \ref{lemma:local-candidate-set},它对于设计CSM-TopK 搜索策略至关重要。

\begin{lemma} \label{lemma:local-candidate-set}
    假设 $g^i=$ \{$v_1, v_2$, $\cdots, v_i$\} 是匹配 $q^i$的部分匹配.
    若 $g=\{v_1, v_2, \dots, v_{|V_Q|}\}$ 是包含 $g^i$ 的完整匹配,则对于 $V_g\setminus V_{g^i}$ 中的每个 $v_j$,若 $v_j$ 对应查询顶点 $u_j$,那么必然有:
       $v_j\in$ $C_{g^i}(u_j)$.
    \end{lemma}

\begin{proof}
        实际上,对于两个数据顶点 $v_{i^\prime}$ 和 $v_j$, 其中 $v_{i^\prime}\in$ $V_{g^i}$ ,  $v_{i^\prime}$ 匹配$u_{i^\prime}$ 
            以及 $v_{j}\in$ $V_{g}\setminus V_{g^i}$ , $v_{j}$ 匹配 $u_j$ ($i^\prime\leq i<j$), 
            如果 $(u_{i^\prime}, u_j)\in E_Q$ (即, $u_j\in$ $RN(u_{i^\prime})$), 则 $v_j$ $\in N_{G_t}^{L(u_j)}(v_{i^\prime})$, 即 $v_j$ 必须是 $v_{i^\prime}$ 的邻居,标签为$L(u_j)$. 
            因此,根据公式\ref{equation:local-candidate-set},$v_j$ 将是$C_{g^i}(u_j)$中的顶点。
\end{proof}


在定义 \ref{def:local-candidate-set} 和引理 \ref{lemma:local-candidate-set} 的基础上,在我们的递归搜索过程中,对于每一个部分匹配 $g^i$,我们可以为每个尚未匹配的查询顶点不断维护一个局部候选集,用于加速 CSM-TopK 搜索。
一种可能的新的剪枝策略是,当局部候选集为空时,则当前部分匹配不可能扩展为完整匹配,
因此,我们可以安全地剪枝当前的部分匹配上的搜索分支。
更重要的是,局部候选集可以用来计算一个新的界限 $lBound$,这个界限比 $gBound$ 更加紧凑。
对于每一个包含部分匹配 $g^i$ 的完整匹配 $g=\{v_1, v_2, \cdots, v_{|V_Q|}\}$,其中 $v_j$($i < j$)必须在 $C_{g^i}(u_j)$ 中,因此我们可以得出以下结论:
\begin{equation*}
    \begin{aligned}
      den(Dstar(g^\prime, j)) & \leq \max\nolimits_{v_{j^\prime}\in C_{g^i}(u_j)}\left(den(Qstar(j), v_{j^\prime})\right) \\
        & \leq den(Qstar(j), C_{g^i}(u_j))
    \end{aligned}
\end{equation*}


其中 $Dstar(g^\prime, j)$ 是 $g$ 的一个子图 $g^\prime$,其中 $g^\prime$ 匹配 $Qstar(j)$(见公式 \ref{equation:dstar-decompose}),$den(Qstar(j), v_{j^\prime})$ 表示 $Qstar(j)$ 与 $v_{j^\prime}$ 匹配时的最大密度(见公式 \ref{equation:den-qstar-v})。
\inull{
\begin{theorem} \label{theorem:local-bound}
考虑图 $G_t$、查询 $Q$ 以及相应的匹配顺序 $\Phi$。
假设 $g^i$ 是在 $\Phi$ 下与 $q^i$ 匹配的部分匹配,
并且 $g_{max}$ 是 $g^i$ 可以扩展成的最大密度的完整匹配。
则可以得出结论:

den(g_{max}) \leq den(g^i) + \sum\nolimits_{j=i+1}^{|V_Q|} \left(den(Qstar(j), C_{g^i}(u_j))\right)

\end{theorem}
} 
因此,如果 $g_{max}$ 是包含 $g^i$ 的完整密度最大的匹配,那么:
\[
	den(g_{max}) \leq den(g^i) + \sum\nolimits_{j=i+1}^{|V_Q|} \left(den(Qstar(j), C_{g^i}(u_j))\right)
\]

同时,对于每一个部分匹配 $g^i$,我们可以计算出它可以扩展为的一个完整匹配的密度上界。
我们使用 $lBound_{G_t, Q, \Phi}(g^i)$ 来表示这个界限,即:
\begin{equation*}
    \begin{multlined}
    lBound_{G_t, Q, \Phi}(g^i) \\ = den(g^i) + \sum\nolimits_{j=i+1}^{|V_Q|} \left(den(Qstar(j), C_{g^i}(u_j))\right)
    \end{multlined}
    \end{equation*}
为了简便起见,当上下文明确时,我们将其表示为 $lBound(g^i)$。
需要注意的是,如果 $C_{g^i}(u_j)$ 被标记为未定义,则在计算 $lBound(g^i)$ 时,我们将 $den(Qstar(j), C_{g^i}(u_j))$ 替换为 $den(Qstar(j))$(来自全局的 MWstar)。

通常,$lBound(g^i)$ 比 $gBound(g^i)$ 要紧凑得多。
实际上,我们知道 $C_{g^i}(u_j)$ 是通过几个邻居集的交集形成的(见定义 \ref{def:local-candidate-set}),因此 $|C_{g^i}(u_j)|$ 通常不会大于平均度,而平均度远小于 $|V_{G_t}|$。
此外,在现实世界的大型图中,$|V_{G_t}|$ 通常比 $G_t$ 的平均度大一千倍以上。
因此,$den(Qstar(j), C_{g^i}(u_j))$ 通常远小于 $den(Qstar(j), V)) = den(Qstar(j))$,从而 $lBound(g^i)$ 通常比 $gBound(g^i)$ 紧凑得多。

\begin{figure}[h!]
\centering
\resizebox{0.9\linewidth}{!}{
\includegraphics{\csmfolder searching_space_example.pdf}
}
\caption{基于 MWstar 的剪枝}
\label{fig:exp:mwstar:pruning}
\end{figure}

考虑图 \ref{fig:csm-topk} 中的运行示例。
我们在图 \ref{fig:exp:mwstar:pruning} 中展示了在更新 $o_t=<+,v_1,v_2>$ ($k=3$)时的 CSM-TopK 搜索树。
搜索从 $g^2=\{v_1, v_2\}$ 开始,匹配查询 $q^2=\{u_1, u_2\}$,其中匹配顺序为 $\Phi=\{u_1, u_2, u_3, u_4\}$。
\begin{itemize}
\item
\textcircled{1} 如果没有 MWstar 索引,则我们计算 $C_{g^2}(u_3)=\{v_5, v_6, v_7, v_8\}$,并将 $g^2$ 扩展为四个新的搜索分支。
之后,对于每一个 $g^3 \in g^2 \times C_{g^2}(u_3)$,可行候选集 $C_{g^3}(u_4)$ 将是 $\{v_{1001}, \cdots, v_{2000}\}$。
因此,每个 $g^3$ 将扩展为 $1000$ 个新的搜索分支。
如果没有 MWstar 时,基于 $g^2$ 的搜索树中可能会有超过 $4000$ 个节点。
\item
\textcircled{2} 如果仅使用全局 MWstar 索引,则对于 $q^3$ 下的四个分支,即 $g_1^3=\{v_1, v_2, v_5\}$、$g_2^3=\{v_1, v_2, v_6\}$、$g_3^3=\{v_1, v_2, v_7\}$、$g_4^3=\{v_1, v_2, v_8\}$,
我们有 $gBound(g_3^3)=gBound(g_4^3)=50+10+10+den(Qstar(4))=130$,这个值小于 $den(g_{min})=140$,其中 $g_{min}=\{v_1, v_3, v_6, v_9\}$,如图 \ref{fig:csm-topk}c 所示。
因此,我们将立即剪枝 $g_3^3$ 和 $g_4^3$。
$g_1^3$ 和 $g_2^3$ 不会被剪枝,因为 $gBound(g_1^3) = gBound(g_2^3) = 50+20+20+den(Qstar(4)) = 150 > 140$。
因此,使用全局 MWstar 时,搜索空间的大小接近 $2000$。
\item
\textcircled{3} 如果同时应用全局和局部 MWstar 索引,在我们从 $g^2$ 开始进行搜索时,
我们将发现 $C_{g^2}(u_3)=\{v_5, v_6, v_7, v_8\}$,并且 $den(Qstar(3), C_{g^2}(u_3)) = 50$。
同样,$C_{g^2}(u_4)=\{v_{1001}, \cdots, v_{2000}\}$,并且 $den(Qstar(4), C_{g^2}(u_4)) = 20$。
因此,$lBound(g^2)=den(g^2)+50+50+20=120<140$,于是 $g^2$ 会被剪枝。
总的搜索空间可以忽略不计。
我们可以看到,我们基于MWstar的修剪策略可以显著减少搜索空间,避免对不够密集的匹配进行浪费的计算。
\end{itemize}

因此,在我们的递归搜索过程中,对于每一个部分匹配 $g^i$,我们可以为每个未匹配的查询顶点 $u_j$(即 $u_j \in V_Q \setminus V_{q^i}$)维护一个局部候选集。
这些候选集将与我们的 MWstars 一起使用,以便尽早修剪掉那些不太可能的部分匹配。算法 \ref{alg:localmwstar:framework} 中提供了这种面向CSM-TopK的伪代码。

具体而言,对于一个部分匹配 $g^2 = \{v_1, v_2\}$,我们根据公式 \ref{equation:local-candidate-set} (算法 \ref{alg:localmwstar:framework} 中的第 \ref{code:init-lcs:begin}-\ref{code:init-lcs:end} 行)初始化每个未匹配查询顶点的局部候选集(即 $u_3$,$u_4$,$\cdots$,$u_{|V_Q|}$)。
如果某个未匹配的查询顶点 $u_j$ 在 $V_{g^2}$ 中没有左邻居(即 $\{u_1, u_2\}$),我们将相应的 $C_{g^2}(u_j)$ 标记为未定义。
如前所述,如果 $C_{g^i}(u_j)$ 被标记为未定义,我们在计算 $lBound(g^i)$ 时将 $den(Qstar(j), C_{g^i}(u_j))$ 替换为 $den(Qstar(j))$(来自全局 MWstar)。
然后,我们可以从 $g^2$ 开始递归地进行 面向CSM-TopK 的搜索。

\begin{algorithm}[H]
    \small
    \caption{基于全局和局部 MWstars 的 CSM-TopK 搜索}
    \label{alg:localmwstar:framework}
    \KwIn{一个动态图 $\iG = (G_0, \Delta)$,查询图$Q$ 和 参数$k$}
    \KwOut{对于每个 $G_t$,报告 $A^k_t = {g_1, g_2, \ldots, g_k}$}
    % \tcc{离线预处理}
    BuildMatchingOrder($Q$, $u_1$, $u_2$) 对于每个 $(u_1, u_2) \in Q$ \\
    BuildMWstar($G_0$, $Q$, $\Phi$) 对每个 $\Phi$  \\
    计算 $A_0^k$ 在 $G_0$ 上的查询 $Q$ \\
    % \tcc{在线预处理}
    \ForEach{$o_t = (op, v_1, v_2) \in \Delta$}{
    应用 $o_t$ 到 $G_{t-1}$ 形成 $G_t$,并更新 MWstar \\
    初始化 $A^k_{t}$ 为 $A^k_{t-1}$ \\
    \If{$op = +$}{
    \ForEach{$(u_1, u_2) \in E_{Q_t}$ 匹配 $(v_1, v_2)$} {\label{local:code:baseline:begin}
    令 $g^2 = {v_1, v_2}$ \\
    设 $v_1$,$v_2$ 为已访问\\
    令 $\Phi$ 为匹配顺序,并从 $u_1$,$u_2$ 开始 \\
    \ForEach{$u_j \in V_Q \setminus V_{q^2}$}{ \label{code:init-lcs:begin}
    根据定义 \ref{def:local-candidate-set} 初始化 $C_{g^2}(u_j)$ \label{code:init-lcs:end}
    }
    FindDenseMatches-Both$(G_t, Q, \Phi, g^2, A_t^k, k)$ \label{local:code:baseline:end} \
    /* 见算法 \ref{alg:find-dense-matches-formal} 获取详细过程 */
    }
    }
    \If{$op = -$}{
    执行与算法 \ref{alg:baseline:framework} 中第 \ref{code:base-del:begin}-\ref{code:base-del:end} 行对称的操作
    }
    }
    \end{algorithm}

对于每个部分匹配 $g^i$,如果满足以下任意条件之一,我们可以安全地剪枝 $g^i$:
\begin{itemize}
\item \textcircled{1} 有一个局部候选集为空(算法 \ref{alg:find-dense-matches-formal} 中的第 \ref{code:lcs-empty:begin}-\ref{code:lcs-empty:end} 行),
\item \textcircled{2} $lBound(g^i)$ 不够大(算法 \ref{alg:find-dense-matches-formal} 中的第 \ref{code:B-formal:begin}-\ref{code:B-formal:end} 行)。
\end{itemize}
如果 $g^i$ 没有被丢弃,那么我们可以利用已经计算好的可行候选集 $C_{g^i}(u_{i+1})$ 将 $g^i$ 扩展为若干个 $g^{i+1}$,继续递归搜索。

对于每个 $g^{i+1} = g^i \cup {v_{i+1}}$,其中 $v_{i+1} \in C_{g^i}(u_{i+1})$,我们将按以下方式更新每个 $u_j$($j > i+1$)的局部候选集:
如果 $u_j \notin RN(u_{i+1})$,我们无需执行任何操作;否则,如果 $u_j \in RN(u_{i+1})$(算法 \ref{alg:find-dense-matches-formal} 中的第 \ref{code:formal-each-rn:begin}-\ref{code:formal-each-rn:end} 行):
\begin{itemize}
\item 如果 $u_j$ 的局部候选集未定义,则我们使用基于访问性的过滤将 $C_{g^{i+1}}(u_j)$ 设置为 $N_{G_t}^{L(u_j)}(v_{i+1})$,即 $v_{i+1}$ 的邻居的标签都为 $L(u_j)$。
\item 如果 $u_j$ 的局部候选集已存在,则根据定义 \ref{def:local-candidate-set},我们使用基于访问性的过滤,将 $C_{g^{i+1}}(u_j)$ 设置为 $C_{g^i}(u_j) \cap N_{G_t}^{L(u_j)}(v_{i+1})$。
\end{itemize}

\begin{algorithm}[H]
    \scriptsize
    \caption{利用 $lBound$ 进行递归搜索}
    \label{alg:find-dense-matches-formal}
    \SetKwBlock{iFunc}{Procedure}{End Procedure}
    \iFunc(FindDenseMatches-Both{(}$G_t, Q, \Phi, g^i, A, k${)})
    {
    令 $g_{min}$ 为 $A$ 中密度最小的匹配 \\
    \If{$|g^i|=|V_Q|$}{
    \If{$|A|<k \lor den(g_{min}) < den(g^i)$ } {
    将 $g^i$ 添加到 $A$ 中 。\\
    如果 $|A|>k$,则从 $A$ 中移除 $g_{min}$\
    \Return
    }
    }
    初始化 $lBound(g^i)$ 为 $den(g^i)$\
    \ForEach{$u_{j} \in V_Q \setminus V_{q^i}$}{ \label{code:lcs-empty:begin}
    \If{$C_{g^i}(u_j) = \emptyset$}{
    \Return   \label{code:lcs-empty:end}
    }
    \If{$C_{g^i}(u_j)$ 未定义}{
    /* 来自全局 MWstar 的密度 */ \\
    $lBound(g^i) += den(Qstar(j))$
    }
    \Else{
                    /* 来自局部 MWstar 的密度 */  \\
    $lBound(g^i) += den(Qstar(j), C_{g^i}(u_j))$
    }
    
    }
    
    \If{$|A|=k \land lBound(g^i) \leq den(g_{min})$}{ \label{code:B-formal:begin}
        \Return \label{code:B-formal:end}
    }
    \ForEach{$v_{i+1} \in C_{g^i}(u_{i+1})$}{ \label{code:formal-extension:begin}
    将 $v_{i+1}$ 标记为已访问,并令 $g^{i+1} = g^i \cup {v_{i+1}}$ \\
    \ForEach{$u_{j} \in RN(u_{i+1})$}{ \label{code:formal-each-rn:begin}
    \If{$C_{g^{i}}(u_{j})$ 未定义}{
        $C_{g^{i+1}}(u_{j}) = N_{G_t}^{L(u_{j})}(v_{i+1})$ 
    }
    \Else{
        $C_{g^{i+1}}(u_{j}) = C_{g^{i}}(u_{j}) \cap N_{G_t}^{L(u_{j})}(v_{i+1})$ \label{code:formal-each-rn:end}
    }
    $C_{g^{i}}(u_{j}) =$ $(C_{g^{i}}(u_{j})).filterVisited()$
    }  
    FindDenseMatches-Both{(}$G_t, Q, \Phi, g^{i+1}, A, k${)} \\ 
    将 $v_{i+1}$ 标记为未访问 \label{code:formal-extension:end}  \\
    }
    
    \Return 
    
    }
    \end{algorithm}



一共有 $|E_Q|$ 个匹配顺序,每个数据顶点 $v_i$ 在某一匹配顺序下有 $|V_Q|$ 个密度(即每个 $u_j$ 的 $den(Qstar(j, v_i))$)。因此,空间复杂度显然是 $O(|E_Q|*|V_{G_t}|*|V_Q|)$,这是与数据图的顶点数量成线性关系的。
实际上,局部 MWstar 的维护也是高效的。考虑一个更新 $o_t = (op, v_1, v_2)$,其中 $(v_1, v_2)$ 匹配 $(u_{i_1}, u_{i_2})$。
如果 $o_t$ 是插入操作,即 $op = +$,对于每个匹配顺序 $\Phi$,假设 $u_{i_1} \in LN(u_{i_2})$(我们省略对称情况的讨论),我们可以在 $O(d_t)$ 时间内更新 $den(Qstar(i_2), v_2)$(算法 \ref{alg:base-mwstar:update} 中第 \ref{code:basic-den-update:begin}-\ref{code:basic-den-update:end} 行),其中 $d_t$ 是 $G_t$ 的平均度数。
由于有 $|E_Q|$ 个匹配顺序,边插入时局部 MWstar 的维护需要 $O(|E_Q| * d_t)$ 平均时间。

如果 $o_t$ 是删除操作,即 $op = -$,对于每个匹配顺序 $\Phi$,如果 $u_{i_1} \in LN(u_{i_2})$(对称情况的讨论省略),我们可以在 $O(d_t)$ 时间内更新 $den(Qstar(i_2), v_2)$。
因此,删除操作的时间复杂度是 $O(|E_Q| * d_t)$,几乎是常数时间。

\section{图压缩技术}
\label{mwstar:compact-graph}

在本节中,我们将针对每个$G_t$的CSM-TopK搜索转化为在压缩图$\cG_t$和新查询$\cQ$上的更高效的搜索。
$\cQ$与$Q$拥有相同的顶点集和边集,唯一的不同在于新的标签函数$\cL_{\cQ}$(当上下文明确时简写为$\cL$),其中$\cQ$中的每个查询顶点都具有新的唯一标签。
在不失一般性的情况下,我们假设$\cL(u_j) = l_j^{new}$。
因此,$\cQ = {V_Q, E_Q, \cL}$。

$\cG_0$是基于$G_0$构建的,而对于$t>0$,$\cG_t$是通过一系列基于$o_t$的调整从$\cG_{t-1}$转换而来。
我们在$\cG_t$上进行的$\cQ$搜索可以更加高效地返回正确的$A_{G_t, Q}^k$,且所需的空间开销、搜索时间以及维护开销都大大降低。

\subsection{压缩图构建}
\label{s-sec:compact-graph-construction}

接下来我们讨论在$G_t$和$Q$上的压缩图$\cG_t$。

我们用$\tau$表示$\cG_t$中的一个数据顶点,用$\cg$表示$\cG_t$中的一个子图。
每个$\tau \in V_{\cG_t}$对应着$V_{G_t}$中的一个数据顶点和$V_Q$中的一个查询顶点之间的匹配关系。
我们使用$\langle v_i, u_j \rangle$来表示一个匹配关系,其中$v_i$匹配$u_j$。
如果$v_i$通过了$u_j$的邻域标签频率过滤(NLF)~\cite{sgi-turbo-iso-DBLP:conf/sigmod/HanLL13},我们说$v_i$是$u_j$的NLF候选(NLF-candidate),即:
$v_i$匹配$u_j$,并且对于任何标签$l$(在$L_Q$下,而不是$\cL_{\cQ}$),有$|N_{G_t}^l(v_i)| \geq |N_Q^l(u_j)|$。

每个顶点$\tau \in V_{\cG_t}$恰好对应一个唯一的匹配关系$\langle v_i, u_j \rangle$,其中$v_i$是$u_j$的NLF候选。
我们将$\cL(\tau)$设置为$u_j$在$\cQ$中的标签,即$l_j^{new}$。
因此,我们可以按以下方式构建$V_{\cG_t}$:首先将$V_{\cG_t}$初始化为空集,并枚举$v_i$和$u_j$的所有匹配对。
如果$v_i$是$u_j$的NLF候选,则在$V_{\cG_t}$中创建一个新的顶点$\tau_{i^\prime}$,标签为$\cL(\tau_{i^\prime}) = l_j^{new}$。

$E_{\cG_t}$是基于$V_{\cG_t}$构建的。
考虑两个顶点$\tau_1$和$\tau_2$,它们对应的匹配关系分别为$\langle v_{i_1}, u_{j_1} \rangle$和$\langle v_{i_2}, u_{j_2} \rangle$。
当且仅当$(v_{i_1}, v_{i_2}) \in E_{G_t}$且$(u_{j_1}, u_{j_2}) \in E_Q$,则在$E_{\cG_t}$中存在一条边$(\tau_1, \tau_2)$。
显然,$E_{\cG_t}$中的边$(\tau_1, \tau_2)$表示数据边$(v_{i_1}, v_{i_2})$与查询边$(u_{j_1}, u_{j_2})$之间的匹配关系。
同时,将边$(\tau_1, \tau_2)$的权重设置为边$(v_1, v_2)$的权重,即$W_{v_1, v_2}$。

图\ref{fig:compact}中展示了图\ref{fig:csm-topk}a中的运行示例对应的压缩图$\cG_t$和新查询$\cQ$。
我们可以看到,$\tau_1$对应着$\langle v_1, u_1 \rangle$,因为$v_1$是$u_1$的NLF候选。
同时,$V_{\cG_t}$中没有与$v_{10}$或$v_{11}$相关的顶点,因为它们都不是任何查询顶点的NLF候选。

\begin{figure}[h!]
\def\wscorevone{0.62}
\centering
\begin{subfigure}[t]{\wscorevone\linewidth}
\centering
\resizebox{\linewidth}{!}
{
\includegraphics{\csmfolder data_compact_graph.pdf}
}
\caption{压缩图$\cG_t$}
\label{fig:data_compact_graph}
\end{subfigure}
\begin{subfigure}[t]{0.35\linewidth}
\centering
\resizebox{\linewidth}{!}
{
\includegraphics{\csmfolder query_compact_graph.pdf}
}
\caption{新查询图$\cQ$}
\label{fig:query_compact_graph}
\end{subfigure}
\caption{压缩图和新查询示例}
\label{fig:compact}
\end{figure}


\subsection{压缩图计算}
接下来我们讨论如何基于我们在$\cG_t$上的CSM-TopK搜索来计算目标密集答案$A_{G_t, Q}^k$。

我们假设$\cG_t$上的MWstar已经构建完成。
对于一个子图$\cg = \{\tau_1, \tau_2, \dots, \tau_{|V_{\cQ}|}\}$,我们假设$\tau_x$对应匹配关系$\langle v_{i_x}, u_{j_x} \rangle$($1 \leq x \leq |V_{\cQ}|$)。
我们可以很容易地理解,如果$g = \{v_{i_1}, v_{i_2}, \dots, v_{i_{|V_Q|}}\}$是$Q$在$G_t$上的匹配,而$\Phi = \{u_{j_1}, u_{j_2}, \dots, u_{j_{|V_Q|}}\}$是$Q$的查询顶点,那么$\cg$一定是$\cQ$在$\cG_t$上的匹配。
然而,反过来并不成立,也就是说,当$\cg$是$\cQ$在$\cG_t$上的匹配时,对应的$g$可能不是$Q$的匹配。
具体来说,对于子图$V_{\cg}$中的两个不同的顶点$\tau_x$和$\tau_y$,它们对应的数据顶点$v_{i_x}$和$v_{i_y}$可能是相同的(即$v_{i_x} = v_{i_y}$),这会将同构映射$f(u_{j_{x^\prime}}) = v_{i_{x^\prime}}$($1 \leq x^\prime \leq |V_Q|$), 从同构降级为同态~\cite{homomorphism-DBLP:conf/soda/CyganFGKMPS16}。

因此,我们需要一种有效的策略来保证搜索仅返回其对应的$g$与$Q$在$G_t$中完全同构的$\cg$。
实际上,我们可以通过扩展$V_{\cG_t}$中数据顶点的访问状态到对应的$V_{G_t}$中的数据顶点来轻松实现这一点。
具体来说,对于$V_{G_t}$中的每个数据顶点$v_{i_x}$,我们可以在$V_{\cG_t}$中维护一个顶点计数,该计数表示在$V_{\cG_t}$,$v_{i_x}$与某些查询顶点之间的匹配关系。
我们称这个计数为$v_{i_x}$的\emph{访问计数}。
这样,对于每个顶点$\tau_x$,如果它对应的匹配关系是$\langle v_{i_x}, u_{j_x} \rangle$,当我们需要标记$\tau_x$为已访问时,我们会同时增加$v_{i_x}$的访问计数;
而当我们将$\tau_x$重置为未访问时,我们减少$v_{i_x}$的访问计数。
当我们对候选集进行访问检查时(例如公式 \ref{equation:wcoj-intersection} 和 \ref{equation:local-candidate-set}),我们会过滤掉那些访问计数大于 0 的 $\tau_x$。
显然,访问扩展是非常简单的,并且对搜索的额外时间消耗可以忽略不计。
\begin{theorem} \label{theorem:reduce-correctness}
如果 $A_{\cG_t, \cQ}^k$ 包含通过我们的CSM-TopK搜索(算法 \ref{alg:localmwstar:framework})在 $\cG_t$ 和 $\cQ$ 上得到的前 $k$ 个答案,并且应用了访问扩展,
那么对于 $A_{\cG_t, \cQ}^k$ 中的每个子图 $\cg=$ \{$\tau_1$, $\tau_2$, $\cdots$, $\tau_{|V_{\cQ}|}$\},
其中 $\tau_{x}$ 对应于匹配关系 $\langle v_{i_x}$, $u_{j_x}\rangle$($1\leq x\leq |V_{\cQ}|$),
$g=$ \{$v_{i_1}$, $v_{i_2}$, $\cdots$, $v_{i_{|V_Q|}}$\} 必须是 $A_{G_t, Q}^k$ 中的一个匹配,且满足 $den(g) = den(\cg)$。
\end{theorem}
\begin{proof}
设 $f$ 表示从 $V_{\cQ}$ 到 $V_{\cg}$ 的双向映射,即 $f(u_x) = \tau_x$ ($1\leq x\leq |V_Q|$)。
我们可以构建一个从 $V_{\cg}$ 到集合 \{$v_{i_1}$, $v_{i_2}$, $\cdots$, $v_{i_{|V_Q|}}$\} 的映射 $f^{\prime}$,其中 $f^{\prime}(\tau_x) = v_{i_x}$。
由于我们在搜索 $\cg$ 时应用了访问扩展,对于 $1\leq x<y\leq |V_Q|$,$v_{i_x} \neq v_{i_y}$, $f^{\prime}$ 也是双射的。
因此,设 $f^{new} = f \cdot f^\prime$,那么 $f^{new}$ 将是 $V_Q$ 和满足子图匹配约束的集合 \{$v_{i_1}$, $v_{i_2}$, $\cdots$, $v_{i_{|V_Q|}}$\} 之间的双射。
\end{proof}
定理 \ref{theorem:reduce-correctness} 保证了我们在压缩图上的搜索的正确性。

例如,图 \ref{fig:compact} 中的 $\cG_t$ 和 $\cQ$ 对应的 $A_{\cG_t, \cQ}^3$ 包含 $\cg_1=$\{$\tau_1$, $\tau_3$, $\tau_4$, $\tau_9$\},$\cg_2=$\{$\tau_1$, $\tau_3$, $\tau_5$, $\tau_9$\} 和 $\cg_3=$\{$\tau_1$, $\tau_3$, $\tau_6$, $\tau_9$\},我们可以看到对应的 $g_1$、$g_2$ 和 $g_3$ 正好构成图 \ref{fig:csm-topk} 中的 $A_{G_t, Q}^3$。

\subsection{压缩图维护}
接下来我们讨论如何根据每个更新 $o_t$($t>0$)将 $\cG_{t-1}$ 调整为 $\cG_t$。
对于压缩图上的 MWstars 维护,过程与我们之前的讨论完全相同。
假设我们已经在 $\cG_{t-1}$ 上构建了全局和局部的 MWstars。
考虑一个更新 $o_t=$ $(op, v_{i_1}, v_{i_2})$,其中 $v_{i_1}$ 和 $v_{i_2}$ 分别与 $u_{j_1}$ 和 $u_{j_2}$ 匹配。
%
我们首先集中讨论 $E_{\cG_{t-1}}$ 上的边插入/删除(由 $o_t$ 引起的)。这些边的更新将 $\cG_{t-1}$ 转换为 $\cG_t$。

\subsubsection{插入}
如果 $o_t$ 是一个插入操作,
我们首先检查 $v_{i_1}$ (或 $v_{i_2}$)是否是$G_t$上$u_{j_1}$(或 $u_{j_2}$)的 NLF 候选,然后通过以下步骤对 $E_{\cG_{t-1}}$ 进行调整:
\begin{itemize}
\item \textcircled{1} 我们在$G_{t-1}$上应用$o_t$形成 $G_t$,然后初始化 $\cG_t = \cG_{t-1}$。
\item \textcircled{2} 如果 $v_{i_1}$ 不是 $u_{j_1}$ 的 NLF 候选者,且 $v_{i_2}$ 也不是 $u_{j_2}$ 的 NLF 候选者,则$o_t$在$\cG_{t-1}$上不会有任何更新,我们可以直接终止当前更新。
\item \textcircled{3} 如果 $v_{i_1}$(对称情况是 $v_{i_2}$)是 $u_{j_1}$ 的一个新的 NLF 候选者,
即在 $V_{\cG_{t-1}}$ 中没有与 $\langle v_{i_1}, u_{j_1} \rangle$ 相对应的顶点,
那么我们将在 $\cG_t$ 中为 $\langle v_{i_1}, u_{j_1} \rangle$创建一个标签为$i_1$的新顶点 $\tau_1$。
新顶点 $\tau_1$ 的加入可能会导致新的边加入到 $\cG_t$。
我们可以遍历 $v_{i_1}$ 在 $G_t$ 中的邻居,并为每个 $v_{i_3}\in N_{G_t}(v_{i_1})$,
计算所有匹配对 $\langle v_{i_3}, u_{j_3} \rangle$,使得 $u_{j_3}$ 是 $u_{j_1}$ 的邻居,并且 $v_{i_3}$ 是 $u_{j_3}$ 的 NLF 候选者。
这样,我们可以将每条边 $(\tau_1, \tau_3)$ 添加到 $\cG_t$,其中 $\tau_3$ 是对应于 $\langle v_{i_3}, u_{j_3} \rangle$ 的顶点。
\item \textcircled{4} 如果 $v_{i_1}$ 和 $v_{i_2}$ 分别是 $u_{j_1}$ 和 $u_{j_2}$ 的 NLF 候选者,并且它们都不是新的,那么我们将把边 $(\tau_1, \tau_2)$ 添加到 $\cG_t$,其中 $\tau_1$ 和 $\tau_2$ 分别对应于 $\langle v_{i_1}, u_{j_1} \rangle$ 和 $\langle v_{i_2}, u_{j_2} \rangle$两个顶点。
\end{itemize}

很容易理解上述维护步骤的正确性。

在将这些边插入到 $E_{\cG_t}$ 后,我们可以更新 $\cG_t$ 的 MWstars。

由于更新 $o_t$ 可能会导致 $\cG_t$ 中的多条新边。
具体地,设 $E_1$ 和 $E_2$ 分别表示与 $\tau_1$ 和 $\tau_2$ 相邻的两组新插入的边,
其中 $\tau_1$(或 $\tau_2$)是 $V_{\cG_t}$ 中对应于 $\langle v_{i_1}, u_{j_1} \rangle$(或 $\langle v_{i_2}, u_{j_2} \rangle$)的顶点
\footnote{如果 $\tau_1$(或 $\tau_2$)不存在,我们设定 $E_1$(或 $E_2$)为 $\emptyset$。}。

一种直接的方法是从每个新插入的边开始在 $E_1 \cup E_2$ 中进行搜索。
然而,正如定理 \ref{theorem:only-one-edge-to-search} 所指出的,这种朴素策略会导致大量冗余计算。

\begin{theorem} \label{theorem:only-one-edge-to-search}
    考虑更新 $o_t = (op, v_{i_1}, v_{i_2})$,其中 $v_{i_1}$ 和 $v_{i_2}$ 分别与 $u_{j_1}$ 和 $u_{j_2}$ 匹配。
    对于每个新的匹配 $\cg \in A_{\cG_t, \cQ} \setminus A_{\cG_{t-1}, \cQ}$,$\cg$ 必须包含新插入的边 $(\tau_1, \tau_2)$,其中 $\tau_1$ 对应于匹配关系 $\langle v_{i_1}, u_{j_1} \rangle$,而 $\tau_2$ 对应于匹配关系 $\langle v_{i_2}, u_{j_2} \rangle$。
\end{theorem}

\begin{proof}
    我们通过矛盾法来证明这一点。
    假设 $\cg$ 是一个新匹配,其中 $(\tau_1, \tau_2) \notin E_{\cg}$。
    为了不失一般性,我们假设 $\cg$ 是由新插入的边 $(\tau_1, \tau_3)$(其中 $\tau_3 \neq \tau_2$)引起的,其中 $\tau_3$ 对应于匹配关系 $\langle v_{i_3}, u_{i_3} \rangle$。
    由于在应用 $o_t$ 之前,边 $(v_{i_1}, v_{i_3})$ 已经存在于 $E_{G_{t-1}}$ 中,且根据我们的更新规则,
    $(\tau_1, \tau_3)$ 只有在 $\tau_1$ 是一个新创建的顶点时才会成为新插入的边。
    然而,设 $g$ 是 $\cg$ 在 $G_t$ 中的对应匹配,那么 $v_{i_1}$ 必须在 $V_g$ 中而 $v_{i_2}$ 不在,这意味着即使没有 $v_{i_2}$,$v_{i_1}$ 仍然是 $u_{j_1}$ 的 NLF 候选者。
    换句话说,$\tau_1$ 并不是一个新创建的顶点,这与假设矛盾。
    因此,证明完成。
\end{proof}

因此,我们只需要从边 $(\tau_1, \tau_2) \in E_1 \cap E_2$ 开始进行搜索,并且可以剪枝从$(E_1 \cup E_2) \setminus (E_1 \cap E_2)$中新插入的边开始的搜索。

\subsubsection{删除操作}
如果 $o_t$ 是一个删除操作,我们按以下步骤进行维护:
\begin{itemize}
\item \textcircled{1} 我们在 $G_{t-1}$ 上应用 $o_t$,形成 $G_t$,然后初始化 $\cG_t = \cG_{t-1}$;
\item \textcircled{2} 如果在 $\cG_{t-1}$ 中,$v_{i_1}$ 既不是 $u_{j_1}$ 的 NLF 候选者,$v_{i_2}$ 也不是 $u_{j_2}$ 的 NLF 候选者,那么我们无需做任何操作,因为 $\cG_{t-1}$ 上不会发生更新;
\item \textcircled{3} 如果某个顶点 $\tau$ 在删除边 $(v_{i_1}, v_{i_2})$ 后失效,我们需要从 $\cG_t$ 中删除 $\tau$,并移除与 $\tau$ 相邻的边。
\end{itemize}

对于每个删除的边,我们将相应地更新 MWstars。根据第 \ref{ch3:base-framework} 节中的框架,如果之前排名前 $k$ 的答案已经过期,无论 $\cG_{t-1}$ 中删除了多少条边,我们只需要计算一次补充的答案。并且我们的重新计算是在压缩图 $\cG_t$ 上进行的,而不是在 $G_t$ 上。
\section{本章小结}
本章针对动态图中TopK连续子图匹配问题,提出了一种基于星形结构密度索引的高效解决方案。
首先,设计了全局MWstar索引,通过维护星形子结构的最大权重,提供了部分匹配的粗粒度密度上界 $gBound$。
其次,创新性地提出了局部MWstar索引,结合候选集的交运算缩小候选集合的大小,从而构建了更加紧凑的密度上界 $lBound$,显著减少了无效搜索空间。
更为重要的是,该索引方法具有轻量级特点,能够在常数时间内完成更新,并保持线性空间复杂度。
为进一步提升系统性能,本章还提出了动态压缩图技术。
通过NLF过滤策略,构建了规模更小的压缩图,并在该压缩图中维护了由星形结构索引构建的局部MWstar索引,使其更加紧凑。
这样不仅保证了搜索结果的正确性,还显著提高了子图匹配的搜索效率。
