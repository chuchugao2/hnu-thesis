\begin{acknowledgements}
	不知不觉中,在湖南大学计算机科学与技术专业攻读硕士的岁月也即将画上句号。回首这三年,我的心态发生了很大的变化。
	研一刚入学时,我的心中充斥着焦虑和后悔。由于研一后互联网的就业步入了寒冬,看着本科人均大厂的同学,我心中有悔。
	同时,面对着未知的研究生生涯,我总是产生一阵又一阵的焦虑情绪,我害怕研究生三年虚度光阴,我害怕浑浑噩噩的度过而一事无成。
	但是,我遇到了很多很美好的人,他们治愈、关怀、鼓励着我;渐渐地,我变得越来越珍惜这段校园时光,三年一晃而过,而现在我的心中只有着感恩和不舍。
	在此,我希望以我最真挚的笔触,向一路给予我帮助的大家表达谢意。

	师者之光。首先,我要重心的感谢我的两位导师周旭老师和李友焕老师。很幸运,我的三年有两位这么优秀的老师共同指导。
	周旭老师给人的感觉就是平易近人、和蔼可亲,不管是学业上还是工作上,都给予着我很大的帮助。
	她教会了我“努力比天赋更加重要”,让我明白了,每日勤勤恳恳的努力,终将会有回报。
	而李友焕老师,则可以说是我科研以及人生道路上的引路人,不仅是教会了我科研的技巧,也教会了我很多为人处事之道。
	从本科毕设开始,李老师就一直亲力亲为,远程和我讨论课题,一谈就是一个小时。硕士期间,从科研选题的确定到具体算法的实现,
	以及论文的撰写,都离不开李老师的悉心指导。从每周多次的学术课题讨论,深夜中发消息解答我的疑问,科研遇到瓶颈时的鼓励以及帮助,
	论文中每句话的逐句斟酌,无不让我感觉科研的严谨和温度。抛开科研,在就业指导方面,李老师也像一束光一样照亮了我前方的迷惘。他会花时间一句句修改我的简历,
	会时常关心我的就业进展,也会关心我的心情和状态。最终,我能拿到不错的工作,这也离不开李老师的一直以来对我的帮助。
	总之,非常感谢两位老师在科研学习、项目实验中的倾囊相授,愿您们学术长青,桃李满天。
	
	同窗之谊。非常感谢一路上陪伴共同学习生活的418和547的小伙伴们,你们的陪伴让我觉得科研枯燥的日子中也有了一些乐趣和珍贵。
	感谢我的好同门们,和你们互相的倾诉、聊天,分享着生活中的趣事,让我觉得研究生的科研之路都不再是孤身奋战,而是互助友爱。
	这三年来,大家每天在实验室一起学习、干饭、爬山、玩桌游、探索了湖大周围好多好吃的小店。
	非常感谢研究生三年有你们陪我一起疯狂,希望现在短暂的离别,只是为了未来我们更好的相聚。
	
	同室之情。非常感谢本科605宿舍的小伙伴们,我们的同宿舍的情谊一直延续至今。感谢你们在我压力大的时候给予我的鼓励,
	在我委屈的时候也一直站在我的角度上安慰我。虽然我们大家因为读研和工作各奔东西,但是大家每天都会在群上分享自己每天琐事,
	就好像回到了宿舍半夜我们仍然叽叽喳喳,熄灯长谈的时候。希望你们未来都能在各自的领域里闪闪发光。
	
	相知相守。在此特别感谢我的男朋友。异地给我们的感情带来的重大的挑战,但是你总是在无数个节假日和周末义无反顾的奔向我。
	每当我遇到心情不好的时候,总是把自己的坏情绪第一时间的反馈给你。无数的日夜,你总是不厌其烦的听我诉说着生活中的喜怒哀乐。
	祝愿你在未来的日子里学术坦途,而我们继续相伴前行。

	父母之恩。在此要感谢我的家人们,他们成为了我坚强的后盾,让我能心无旁骛的做我想做的事情。虽然他们不懂我的研究生要做什么,
	但是也总是会默默倾听我的牢骚,偷偷了解我的研究方向,特别是我的弟弟妹妹,在我心情不好的时候,总是第一时间“使相”,用力搞笑逗我开心。
	希望你们都能平安健安,希望以后的我能成为你们的后盾。

	不负韶华。最后,我要感谢一直没有放弃的自己。感谢自己的三年没有虚度光阴,感谢无数个挑灯夜战的日子,感谢无数次想要放弃却又继续坚持的时刻。
	“宝剑锋从磨砺出,梅花香自苦寒”,愿自己能成为那一把宝剑,那一缕寒梅,不惧磨砺,不畏严寒。

	麓山巍巍,湘水汤汤。行文最后,谨谢母校湖南大学,我将时刻谨记“实事求是,敢为人先”的校训,
	未来我将以务实创新的态度投身技术领域,不负湖大所育。

\end{acknowledgements}
