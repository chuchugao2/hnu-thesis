\chapter{相关理论与技术基础}
本章节主要阐述了图、子图匹配、连续子图匹配、密度约束下TopK连续子图匹配的概念与应用以及与本课题相关的技术工作。
首先介绍图的基本定义,包括不同类型的图结构(如有向图、无向图、加权图和无权图)。接着阐述子图匹配、连续子图匹配、密度约束下TopK连续子图匹配的关键概念以及在现实场景中的应用。
最后我们介绍了子图匹配以及静态TopK密度优先级机制的子图匹配的相关工作,为本文算法设计提供基础支持。
\section{图的基本定义}
图是一种非常重要的数据结构,用于呈现数据元素之间的多对多的复杂关系。
它由一组节点(顶点)和一组边组成,其中每条边 连接两个节点,以描述它们之间的关联关系。
在计算机科学中,图被广泛应用于表示实体间的结构化关系,其数学定义为一个二元组,
$G=(V,E)$,其中$V$是节点的结合,$E$是边的集合,即:
\begin{equation}
    \begin{split}
        V &= \{ v_1, v_2, \dots, v_n \}, \\
        E &\subseteq V \times V
    \end{split}
\end{equation}

根据边的特性,图可以根据不同的划分标准进行分类。首先,根据边的方向性可分为有向图(图\ref{fig:example_noweight})与无向图(图\ref{fig:example_weight})。有向图中的边具有明确的方向的,每条边表示一种特定的方向,表示从起点指向终点的有向关系。如图\ref{fig:example_noweight}中所示$<v_1',v_2'>$,表示一条有向边,$v_1'$是起点,$v_2'$是终点,表示从节点 $v_1'$ 指向节点 $v_2'$ 的有向边。
与之相对,无向图的边则没有方向性,即两个相连的节点可以互相访问,因此无向图又可以称之为特殊的有向图,因为每条边都代表着一个双向可达关系。如图\ref{fig:example_weight}中所示$<v_1,v_2>$,表示一条无向边,其可以视为由两条有向边$<v_1,v_2>$和$<v_2,v_1>$组成。
此外,根据边上是否附带权重,图还可以进一步分为有权图(图\ref{fig:example_weight})和无权图(图\ref{fig:example_noweight})。
无权图中的边仅表示两个节点间的可达关系,不携带额外的数值信息;而在有权图中,边的权重承载了附加信息,如距离、费用、强度等指标。

\begin{figure}[h!]
    \def\wscorevone{0.48}
    \centering
        \begin{subfigure}[t]{\wscorevone\linewidth}
            \centering
            \resizebox{\linewidth}{!}
            {
                \includegraphics{\figs youxiangwuquan.pdf}
            }
            \caption{有向无权图}
            \label{fig:example_noweight}
        \end{subfigure}
        \hfill
        \begin{subfigure}[t]{\wscorevone\linewidth}
            \centering
            \resizebox{\linewidth}{!}
            {
                \includegraphics{\figs wuxiangyouquan.pdf}
            }
            \caption{无向有权图}
            \label{fig:example_weight}
        \end{subfigure}
        \label{fig:definition}
        \caption{图的常见分类}
    \end{figure}

图(Graph)这种数据结构能够使用非常简洁的形式来表达复杂的数据。
具体而言,图结构擅长表达形式多样的数据结构,特别是在分析具有明确主谓宾结构的句子时,其能够通过节点表示实体、边表示关系,进而清晰展现实体间的关联关系。
%利用图这种数据结构,它能够很好的表达陈述句中的主谓宾,即以主语和宾语为对象,谓语作为两个对象之间的关联关系,这保证了它对于形式多样的复杂数据有非常强的表达能力。
现实世界中的各类信息往往可以拆解为主谓宾结构,进而构建相应的图模型。更重要的是,大多数实际应用中构建的图模型不仅关注节点之间的关系,还会为边赋予权值,以表示特定的含义。
以金融支付场景为例,“用户A向用户B转账100元”可以建模为一个图结构,其中“用户A”和“用户B”分别表示图中的两个节点,而“转账”则表示它们之间的边,边上的权重值“100”则代表了转账的金额。
通过这种方式,图模型能够清晰地表示对象之间的关联,并为数据分析提供结构化支持,帮助研究人员挖掘潜在的规律和模式。
% 因此,图模型在现实世界中具有广泛的应用,能够清晰展示对象之间的关联关系,并帮助数据分析人员深入理解和挖掘数据背后的潜在规律。
在本文中,我们主要以无向有权图作为主要分析对象,并且所提出的算法也同样适用于有向图,以期为更复杂的图结构分析提供借鉴。

\section{子图匹配的概念与应用}
\label{section:subgraph-match}
子图匹配问题是图论中的一个经典问题\cite{sm-ullmann-DBLP:journals/jacm/Ullmann76}。给定一个查询图(Query Graph)和一个数据图(Data Graph),子图匹配的任务是寻找数据图$G$中所有能够与查询图$Q$完全匹配的子图。
具体而言,子图匹配涉及以下几个关键概念:

    \textbullet~\textbf{子图同构(Subgraph Isomorphism)}:即给定一个查询图$Q$以及一个数据图$G$,查询图 $Q$ 与数据图 $G$ 中的某个子图 $g$ 是同构的,当且仅当存在一个双向映射,使得查询图中的每个节点都能够与数据图中的一个唯一节点一一对应,并且查询图中的每一条边在数据图中也有对应的边。
    %$Q$子图同构于$G$中的一个子图g,当且仅当存在一个双射函数,对于Q中的每一个节点,在g中都能到找到与之对应的节点。对于Q中的每一条边,在g中都能找到与其对应的边。即查询图Q和子图g之间存在一个一一对应的节点和边的映射关系,换句话说就是查询图的结构能够完全嵌入到数据图中。
    例如,如图\ref{fig:example_subgraph_matching}所示,左侧为查询图$Q$,右侧为数据图$G$。
    在数据图$G$中,子图$<v_1,v_2,v_3>$与查询图$<u_1,u_2,u_3>$同构,图中虚线表示他们的映射关系。
    \begin{figure}[h!]
        \centering
        \resizebox{0.8\linewidth}{!}{
            \includegraphics{\figs e_subgraph_matching.pdf}
        }
        \caption{子图同构示例}
        \label{fig:example_subgraph_matching}
    \end{figure}

    \textbullet~ \textbf{子图匹配(Subgraph Matching)}:子图匹配的目标是在数据图中找到所有能与查询图完全匹配的子图集合。这个集合中的每一个子图都可以与查询图形成子图同构关系。

    \textbullet~ \textbf{完全匹配(Full Match)}:完全匹配其实就是一次完整的子图同构,查询图中的所有节点和边都在数据图的某个子图中找到一一对应的映射。

    \textbullet~ \textbf{部分匹配(Partial Match)}:部分匹配是指查询图中的一部分节点和边在数据图中找到了对应的关系,但不要求查询图的所有节点都能够找到映射。


子图匹配问题的经典算法主要可分为两类:一类基于深度优先搜索( Depth-First Search,DFS)加回溯的方式\cite{sm-ullmann-DBLP:journals/jacm/Ullmann76},另一类基于广度优先搜索(Breadth-First Search,BFS)及多路联接(Multi-way Join)的方式\cite{sm-bfs-DBLP:conf/focs/AtseriasGM08}。

基于深度优先搜索和回溯的方式: 该算法首先定义查询图$Q$中节点的匹配顺序,如$<u_1, u_2, u_3>$,然后通过深度优先搜索(DFS)逐步匹配数据图中的相应节点。
如果某一步匹配失败,算法会回溯并尝试其他可能的匹配路径。为了找到所有可能的匹配结果,必须进行完整的回溯搜索。
该算法的优点在于能够有效减少中间结果的存储,因为深度优先搜索仅依赖递归调用栈,无需显式存储大量中间匹配状态,从而降低了内存消耗。
并且,由于回溯仅在搜索路径失效时进行,可以通过路径剪枝技术降低计算量。
然而,该方法也存在一定的局限性:首先DFS 本质上是串行执行的,难以高效并行化,且在回溯过程中会产生较大的递归开销。
其次,在大规模图数据中,可能的匹配路径数量指数级增长,导致计算时间大幅增加,影响算法的实际可用性。

基于广度优先的多路联接方法: 广度优先搜索(BFS)算法在子图匹配中的应用与关系数据库中的Join操作相似。
首先为查询图$Q$中的每个节点计算候选解集(即数据图中所有可能匹配的节点集合),然后通过边的相邻关系执行多路联接操作,最终得到符合条件的子图匹配结果。
与深度优先搜索相比,广度优先搜索的优势在于其具有更好的并行性,因为每个节点的候选解在搜索过程中是独立计算的,且候选解集较为稳定。
因此,BFS特别适合在多核或分布式环境中进行并行计算。
例如,本文所提出的框架中的WCOJ算法\cite{sm-bfs-DBLP:conf/focs/AtseriasGM08}采用了这种广度优先搜索与多路联接结合的策略,能够在理论上保证最坏情况下的最优性能。

子图匹配在多个领域具有广泛的应用价值,尤其在处理复杂结构和大规模图数据时,能够提供重要的分析工具。
在生物信息学领域,子图匹配被广泛应用于分子结构的分析与比较\cite{窦建凯2019单图中的近似频繁子图挖掘算法}。通过识别分子结构中的相似模式,研究人员能够发现潜在的药物靶点或生物标志物,在药物研发过程中发挥重要作用。
此外,在蛋白质结构分析和基因表达数据的处理过程中\cite{wang2023zerobind,biology-proteins-DBLP:journals/nar/XenariosSDHKE03,uniprot2017uniprot},子图匹配有助于揭示生物体内的复杂关系和相互作用。
在社交网络分析中,子图匹配同样至关重要。它能够识别社交网络中的互动模式和社区结构,用于分析社交行为、信息传播路径以及群体演化\cite{wang2023mago}。
同时,通过识别用户交互模式中的特定子图结构,可有效检测定向广告投放异常、垃圾邮件传播路径及欺诈行为网络\cite{boshmaf2011socialbot,jiang2012isolating,wang2010don}。
%在个性化推荐和用户行为预测等应用中,子图匹配可用于快速发现具有相似结构或行为的用户群体,从而提升推荐系统的精准度。
网络安全领域也高度依赖子图匹配算法。例如,在入侵检测系统(IDS)中,子图匹配可用于识别恶意行为特征或攻击模式\cite{yuan2023motif}。
通过与已知攻击图谱进行匹配,系统能够迅速检测潜在的安全威胁。此外,子图匹配还能帮助识别异常行为,提高系统在应对零日攻击和复杂攻击时的防御能力。

尽管子图匹配在诸多领域展现出巨大潜力,但由于其判定问题属于 NP难问题,列举所有子图匹配位置的计算复杂度随着图规模的增加而迅速上升,导致在大规模数据处理中面临严峻挑战。
为此,研究人员提出了剪枝优化、索引预处理、并行计算等方法,以提升子图匹配算法的效率,使其更适用于实际应用场景。

首先,对深度优先搜索加回溯的算法而言,由于其逐步深入的搜索方式,回溯过程中可能会产生大量的冗余计算。
为了解决这一问题,剪枝技术被广泛应用。
剪枝技术能够通过提前排除不可能的匹配,从而减少不必要的计算。
已有的一些研究工作如GraphQL\cite{sm-GraphQL-DBLP:series/ads/HeS10},SPath\cite{sm-spath-DBLP:journals/pvldb/ZhaoH10},
都通过构建索引结构和一些剪枝策略对候选集合进行剪枝。
例如,GraphQL\cite{sm-GraphQL-DBLP:series/ads/HeS10}结合标签匹配和拓扑结构,提出基于伪子图同构的过滤框架,实现高效剪枝。
SPath\cite{sm-spath-DBLP:journals/pvldb/ZhaoH10}则是通过路径分解和邻居信息索引来快速计算子图匹配结果。

除了剪枝策略,预处理同样是解决子图匹配效率问题的有效优化手段。 
预处理算法的核心,是在子图匹配算法执行枚举操作前,增设一个预处理环节。
在此环节,通过降低每个查询顶点候选集的规模,获取更为精确的统计数据,进而优化查询匹配的先后顺序。
具体来说,在预处理阶段,会为每个查询顶点构建一个完备的候选顶点集合。后续在进行子图匹配时,算法将从这些候选集合中挑选匹配节点。
已有的一些工作如TurboIso\cite{sm-turbo-iso-DBLP:conf/sigmod/HanLL13},CECI\cite{sm-ceci-DBLP:conf/sigmod/BhattaraiLH19},DP-iso\cite{sm-dp-iso-DBLP:conf/sigmod/HanKGPH19}均采用预处理框架来减少子图匹配过程中的搜索空间。

此外,针对大规模图数据,许多基于并行和分布式计算的优化方法也得到了广泛应用。
由于基于广度优先的子图匹配算法天然具有较好的并行性,研究者们提出了基于多核处理器和GPU加速的子图匹配算法。
现阶段,Stwing\cite{sm-stwing-DBLP:journals/pvldb/SunWWSL12},PSgL\cite{sm-psgl-DBLP:conf/sigmod/ShaoCCMYX14}在应对子图匹配问题时,均运用了并行处理策略。
这些方法通过将搜索任务分配到多个处理单元并行执行,有效地减少了整体计算时间。
例如,研究者们利用图的稀疏性和局部性,设计了基于图分割的并行算法,即将大规模数据图划分为多个子图,在各个子图内独立进行匹配操作,并最终合并匹配结果。
这类方法不仅能够显著提升匹配速度,还能够扩展子图匹配算法的适用范围,使其能够高效处理更大规模的图数据。


\section{连续子图匹配的概念与应用}
\label{csm-concept}
连续子图匹配(Continuous Subgraph Matching,CSM)是指在图的结构发生动态变化时,如何有效地持续计算并维护子图匹配的结果\cite{wang2023survey}。
传统的子图匹配主要针对静态图,在图的结构保持不变的情况下,可以通过一次性计算获得匹配结果。
然而,在现实应用场景中,图往往是动态变化的,即节点和边可能不断新增或删除,这使得静态子图匹配方法无法满足实时性需求。
因此,CSM 需要在图结构变化时,高效更新匹配结果,确保其实时性和有效性。
与静态子图匹配不同,CSM主要涉及以下几个核心特点:

\textbullet~ \textbf{连续子图匹配}:在动态图中,图的结构随着时间不断变化,节点和边可能会被增加或删除。因此,连续子图匹配需要在这些变化中,维持已有子图匹配结果的有效性,避免每次变化都从头开始重新计算匹配结果。
  % \item \textbf{增量更新}:在动态图中,子图匹配不仅是对静态图进行一次性的匹配操作,而是需要根据图的增量变化实时更新匹配结果。增量更新的核心思想是通过最小化每次更新的计算代价,仅针对图的变化部分进行局部调整,而非重新计算所有匹配。通过增量更新,可以大幅度减少计算开销,提升匹配算法的效率。

  \textbullet~ \textbf{匹配结果的维护}:连续子图匹配的核心问题之一是如何高效地维护结果。在动态图中,随着节点和边的增加或删除,某个匹配结果会新增,而某些原有的匹配结果可能会失效,导致匹配结果需要重新调整。此时,需要高效的结果维护策略,实时维护当前所有的子图匹配结果,显著提高匹配的响应速度。

在现实生活中,大部分数据图并非静态不变,而是随着时间不断演化。因此,连续子图匹配具有广泛的应用前景。

例如,在交通网络中,道路连接状态、交通流量、施工信息、事故情况等因素会引起图结构的频繁变化\cite{traffic-graph-matching-DBLP:journals/pvldb/SongGCW14}。CSM 技术为大规模路网的实时拥堵管控提供了有效的技术手段。现代城市交通图通常包含数万个交叉路口和路段,其动态特性表现为:当某一节点发生拥堵时,拥堵状态会以波纹状向周边扩散,最终在图结构中形成特定的“拥堵子图”模式。例如,一种典型的连续拥堵模式由三个相邻交叉口构成,特征是:相邻路段的车速同步下降至某一阈值以下,排队长度持续超限且未见缓解趋势。通过在全路网范围内持续匹配此类模式,交通管理系统能够精准锁定正在演化的拥堵核心区,从而实施精细化控制策略,如动态调整信号灯周期,实现“绿波带”控制,或就近调度交警与清障车辆前往关键节点干预。相比传统依赖人工监测和区域宏观判断的方法,基于连续子图匹配的方式在拥堵初期便可实现自动识别与快速响应,显著提升了监控的准确性与处理效率。

在金融网络中,金融机构之间的资金流动和账户间的交易关系高度动态化\cite{csm-timing-DBLP:conf/icde/Li0O019}。CSM 技术在金融风控体系中扮演着核心角色,尤其在打击欺诈交易与识别异常资金流向方面展现出独特优势。现代金融系统每秒处理数以百万计的交易请求,构建出一个实时变化的资金流动图。传统的静态图分析方法难以捕捉如洗钱、闪电贷套利等依赖复杂结构设计的风险模式。例如,洗钱团伙可能构建出一个多层壳公司参与的闭环交易结构(如 $A \rightarrow B \rightarrow C \rightarrow D \rightarrow A$),或者在极短时间内完成的一系列高频转账操作。借助连续子图匹配,系统能够在图结构发生更新时实时维护索引并扫描与预设高风险模式相符的路径。某跨国银行的实际应用案例显示:当检测到“同一 IP 地址控制多个账户在一秒内形成循环转账”的风险模式时,系统可在 50 毫秒内完成子图识别并冻结相关账户,响应速度相比基于批处理的传统方法提升了三个数量级。

在通信网络安全领域,攻击行为往往会在系统日志或网络通信图中以特定的结构模式留下痕迹,CSM 成为动态威胁检测的重要分析基础\cite{static-topk-Gupta-DBLP:conf/icde/GuptaGYCH14}。以高级持续性威胁(Advanced Persistent Threat, APT)为例,攻击者通常采用“低频、慢速、小流量”的隐匿策略,长时间潜伏于目标网络之中。其攻击路径在日志或访问图中呈现为缓慢演进的访问子图。安全团队可预先定义常见的横向移动模式(如“域控制器 → 文件服务器 → 数据库”),系统通过 CSM 在实时更新的网络访问图中持续扫描符合该拓扑模式的子图。当系统发现某一子图结构高度匹配预定义攻击路径,即可判断存在潜在入侵行为,及时触发告警并采取自动化应对策略,有效防止攻击扩大化。

因此,CSM的研究不仅能够在学术领域提供新的视角,也能够在诸如网络安全、交通监控、金融反欺诈等实际应用中发挥巨大的作用。

\section{密度约束下TopK连续子图匹配的概念与应用}
在子图匹配的研究中,如何在大量候选匹配中快速选择最具代表性的子图是一个重要问题。
传统的子图匹配方法大多关注如何找到所有可能的匹配,而在实际应用中,我们往往更关注某些特定的子图,特别是那些密度较高、结构更为紧密的子图。
且在2014年,Manish Gupta等人首次提出了子图匹配中密度优先级的概念\cite{static-topk-Gupta-DBLP:conf/icde/GuptaGYCH14},其基于子图的密度对匹配结果进行排序,选择密度最高的前$k$个匹配。
然而,该研究主要聚焦于静态图中的密度优先级,对于动态图中的应用并未进行深入探讨。并且提出的静态图索引无法有效应用于动态场景。
至今,尚未有研究关注动态场景中,TopK密度优先级机制对子图匹配结果的影响。
因此,在动态场景下引入密度约束并结合TopK匹配策略成为了一个重要的研究方向。

\begin{figure}[h!]
    \def\wscorevone{0.49}
    \centering
        \begin{subfigure}[t]{\wscorevone\linewidth}
            \centering
            \resizebox{\linewidth}{!}
            {
                \includegraphics{\figs e_attack_pattern.pdf}
            }
            \caption{子网攻击模式~\cite{static-topk-Gupta-DBLP:conf/icde/GuptaGYCH14}}
            \label{fig:example_attack_pattern}
        \end{subfigure}
        \hfill
        \begin{subfigure}[t]{\wscorevone\linewidth}
            \centering
            \resizebox{\linewidth}{!}
            {
                \includegraphics{\figs e_traffic_jam.pdf}
            }
            \caption{交通拥堵模式~\cite{traffic-graph-matching-DBLP:journals/pvldb/SongGCW14}}
            \label{fig:example_traffic_jam}
        \end{subfigure}
        \label{fig:definition}
        \caption{CSM-TopK示例}
    \end{figure}

本文提出了一个新问题——\textbf{密度约束下TopK连续子图匹配},即在\ref{csm-concept}节中定义的连续子图匹配概念的基础上,引入密度优先级机制(见\ref{ch3:definition}节),旨在有效获取前k个具有最高密度的子图匹配结果。
这一问题在实际应用领域具有广阔的应用空间。

例如,在通信网络中,给定固定的网络攻击模式可能具有不同的分析优先级。
图\ref{fig:example_attack_pattern}展示了通信网络中的攻击模式\cite{static-topk-Gupta-DBLP:conf/icde/GuptaGYCH14},其中高数据传输速率的攻击模式应当具有更高的响应优先级。
快速识别这些高数据传输速率的模式,有助于网络管理员及时找到受损最严重的子网络。
同样,图\ref{fig:example_traffic_jam}展示了一种典型的交通拥堵模式\cite{traffic-graph-matching-DBLP:journals/pvldb/SongGCW14},在道路网络中,交通流量较高的模式通常意味着拥堵程度较高,交管系统需要迅速采取应对措施。
快速识别交通流量较高的前$k$个匹配模式,可以让交管系统优先处理堵塞严重的区域,从而快速缓解交通拥堵情况。
上述两种模式均涉及到动态变化的场景,因此,基于动态图中的密度约束进行TopK连续子图匹配,其应用价值远高于静态场景中的相同问题,也远高于动态场景下的子图匹配问题。

\section{子图匹配相关工作}
\subsection{静态子图匹配技术}
在第\ref{section:subgraph-match}节中,我们已经介绍了子图匹配问题的经典算法的常见分类方式及其在图计算领域中的重要作用。本节将重点介绍静态图上的子图匹配技术。静态子图匹配是指在图结构不发生变化的前提下,从数据图中寻找与查询图结构等价(即满足同构或同构变体关系)的子图。围绕该问题,学术界提出了众多经典算法,涵盖从早期的暴力回溯搜索方法到近年来基于索引和优化剪枝的高效算法。代表性方法包括:Ullmann  \cite{sm-ullmann-DBLP:journals/jacm/Ullmann76}、VF2  \cite{sm-vf2-2004}、QuickSI \cite{sm-quicksi-shang2008taming}、TurboIso \cite{sm-turbo-iso-DBLP:conf/sigmod/HanLL13} 以及 CPI \cite{sm-CPI-bi2016efficient} 等。

(1) Ullmann算法

Ullmann算法\cite{sm-ullmann-DBLP:journals/jacm/Ullmann76}由 Julius R. Ullmann 于 1976 年提出,是最早系统地解决子图同构问题的方法之一。该算法采用回溯搜索结合剪枝策略来探索数据图与查询图之间所有可能的同构映射。其核心思想是构建一个候选匹配矩阵 $M$,其中元素 $M(i,j)$ 表示查询图节点 $u_i$ 与数据图节点 $v_j$ 是否可能匹配。在初始化阶段,如果 $v_j$ 的度不小于 $u_i$ 的度,则 $M(i,j) = 1$,否则置为 0,以此实现初步的过滤。

在候选矩阵初始化后,算法通过深度优先的回溯方式逐步尝试为每个查询图节点分配一个数据图中的候选节点。每完成一步匹配后,候选矩阵将被更新,以排除不满足一致性约束的映射。其中,一致性约束要求:若 $u_i$ 被映射到 $v_j$,则对于 $u_i$ 的任一邻居 $u_k$,其匹配的目标节点 $v_l$ 必须满足 $(v_j, v_l)$ 在数据图中存在相应边。为提高效率,Ullmann 算法进一步引入了剪枝策略,如度过滤和连通性过滤,用以尽早剪除不可能成功的匹配路径。

尽管 Ullmann 算法结构清晰、易于实现,并适用于带标签或有向图的匹配问题,但其在面对大规模图时的性能瓶颈较为明显。其最坏时间复杂度为 $O(n! \cdot m^n)$,其中 $n$ 和 $m$ 分别为查询图和数据图的节点数,因此在实际应用中难以满足效率要求。

(2) VF2算法

为了提高效率,VF2算法应运而生,VF2算法最初由L. P. Cordella等人在2001年提出\cite{sm-vf2-2004},后经过多次改进和优化,已成为一种高效的子图同构算法,利用状态空间搜索和启发式规则来减少搜索空间。VF2 算法通过维护两个已匹配节点集合$M_1$和$M_2$ ,分别对应查询图和数据图中的节点,逐步扩展这些匹配集合,同时使用一组可行性条件来确保扩展的有效性。VF2 算法的状态表示包括已匹配节点对集合M、待扩展节点集合$T_1$ (查询图中)和$T_2$(数据图中)以及访问标记数组,用于记录节点是否已被访问。在匹配过程中,算法会检查多种可行性条件,包括结构可行性、标签一致性和语义可行性。结构可行性要求已匹配节点之间的连接关系必须保持一致;标签一致性要求匹配的节点和边必须具有相同的标签(如果图是带标签的);语义可行性则允许加入额外的约束条件,如度约束。


为了进一步优化匹配过程,VF2 算法采用了特定的状态扩展规则。算法优先选择度数较大的节点进行匹配,这样可以更早地发现不可行的匹配,同时利用历史匹配信息来指导当前的匹配决策。与 Ullmann 算法不同,VF2 算法采用了更高效的状态表示和剪枝策略,在实际应用中效率要比 Ullmann 算法高很多。

尽管 VF2 算法在效率上有显著提升,但其算法实现相对复杂。而且对于高度对称的图,由于可能的匹配数量仍然很大,性能提升有限。VF2 算法的时间复杂度在最坏情况下仍然是指数级的,但在实际应用中表现出了更好的性能,能够处理中等规模的图数据,并且支持增量式匹配,可以提前终止以找到第一个匹配。

(3)QuickSI和TurboIso

随着图数据规模的不断增大,传统算法面临着巨大的挑战,QuickSI算法\cite{sm-quicksi-shang2008taming}便是为应对这一挑战而提出的。QuickSI 算法采用两阶段方法,即过滤阶段和验证阶段。在过滤阶段,使用基于路径的索引快速排除不可能包含匹配子图的区域;在验证阶段,对过滤后的候选区域进行精确的子图同构检查。该算法结合了过滤-验证框架和多种优化策略,能够高效处理大规模图数据,但其路径索引的构建需要额外的内存空间,对于高度密集的图,过滤效果可能会受到一定影响。

TurboIso算法\cite{sm-turbo-iso-DBLP:conf/sigmod/HanLL13} 进一步优化了大规模图数据的子图匹配效率,它通过预处理阶段对查询图和数据图进行结构分析,提取有用的特征;在候选生成阶段,利用节点标签和结构特征快速生成候选匹配;在验证优化阶段,采用增量式验证和剪枝策略减少不必要的计算。TurboIso 算法采用了双向扩展、动态排序和内存优化等多种优化技术,在实际应用中比 VF2 和 QuickSI 快几倍到几十倍,但其预处理阶段需要额外的计算开销,对于高度对称的查询图,性能提升有限。

(4)CPI算法

CPI\cite{sm-CPI-bi2016efficient}算法进一步引入了 Core-Forest-Leaf(CFL)分解框架,通过将查询图分解为核心(Core)、森林(Forest)和叶子(Leaf)三部分,实现笛卡尔积的延迟计算,从而减少冗余匹配。具体来说,核心结构是查询图的 2-core 子图,即通过迭代删除度为1的顶点后剩余的最大子图,包含所有非树边;森林结构是核心之外的树状子图,由连接核心的树边构成;叶子结构则是森林中度为1的顶点。

CFL 分解的实现步骤包括:首先进行核心 - 森林分解,通过迭代删除度为 1 的顶点得到核心集,剩余部分为森林集;然后进行森林 - 叶子分解,将森林集进一步分解为森林集(度≥2 的节点)和叶子集(度 = 1 的节点);最后优化匹配顺序,遵循先核心、再森林、最后叶子的顺序。这种分解方式使得 CPI 算法能够更好地处理复杂查询图,通过提前验证非树边约束,减少无效匹配。

在匹配过程中,CPI 算法结合 CFL 分解,基于规范路径索引快速过滤不可能的匹配,然后对候选区域进行精确验证。该算法的路径规范表示确保同构的路径具有相同的表示,从而提高匹配效率。同时,通过将叶子节点按标签分组,利用笛卡尔积延迟策略避免冗余组合,进一步优化了匹配过程。

然而,CPI 算法的路径规范表示的计算可能比较复杂,对于无标签或标签稀疏的图,优势不明显。尽管如此,在处理带标签的大规模图时,CPI 算法结合 CFL 分解能够显著提高子图匹配的效率,其也为未来的子图匹配算法提供了优化思路。

% 综上所述,静态子图匹配技术涵盖了从早期的暴力回溯方法(如 Ullmann)到近年来融合启发式搜索、剪枝优化及离线索引的高效算法(如 TurboIso 和 CPI)。后续研究也在不断探索更高效的索引构建方法、更智能的匹配顺序选择策略,以及适应动态图场景的子图匹配技术。
\subsection{动态子图匹配技术}
动态子图匹配作为图数据管理中的关键技术,广泛应用于社交网络监控、网络安全检测、金融欺诈识别等实时分析场景。相比静态子图匹配,动态匹配需在图结构频繁变化的情况下高效维护匹配结果,因而在算法设计上面临更高挑战。近年来,诸如 Graphflow\cite{csm-graphflow-DBLP:conf/sigmod/KankanamgeSMCS17}、RapidFlow\cite{csm-rapidflow-DBLP:journals/pvldb/SunSHL22}和 CaLiG\cite{csm-calig-DBLP:journals/pacmmod/YangZZY23}等一系列代表性算法相继提出,推动了该领域的发展。以下将重点介绍这三种算法,并将其纳入本研究的实验对比方法中。

(1) GraphFlow

Graphflow\cite{csm-graphflow-DBLP:conf/sigmod/KankanamgeSMCS17} 是一种面向动态图的子图匹配系统,其核心设计围绕 “从更新边出发,增量扩展匹配” 的思路。当图中出现边插入或删除时,算法以更新边为起点,通过递归扩展部分匹配来寻找完整子图同构。

Graphflow 将子图匹配建模为多表连接问题,采用 WCOJ 理论来指导查询执行计划。具体地,系统对查询图的顶点按照稀有度或度数进行排序,形成顶点序 $\phi$,随后按照 $\phi$ 顺序进行多路交集扩展。此策略可显著降低连接爆炸带来的计算冗余,并在理论上保证接近线性的最坏执行时间。

为适应动态图更新,Graphflow 引入了 Delta 通用连接(elta Generic Join,$\Delta$-GJ)机制,将图更新转化为一系列增量子图查询(Delta Subgraph Queries,DSQ)。每个 DSQ 仅覆盖查询图中受当前更新影响的部分结构。通过维护这些 DSQ 的中间候选集合,当边更新发生时,仅需局部更新受影响的 DSQ,无需全局重扫描匹配空间,从而大幅降低计算代价。

(2) Rapidflow 

RapidFlow 针对传统 CSM算法在每次更新中产生冗余中间结果、重复搜索及查询图自同构引发的重复匹配问题,提出了两项关键技术以提升效率:查询缩减(Query Reduction)与对偶匹配(Symmetric Matching)。

在查询缩减技术中,RapidFlow 首先将 CSM 问题转化为仅对局部受影响区域进行处理的批量子图匹配(Batch Subgraph Matching, BSM)任务。当一条边 $e$ 被插入或删除时,系统从数据图中提取受影响区域并构建局部索引 $A$,仅针对约简子图 $Q_R$(删除与 $e$ 相关顶点后的查询图)进行匹配。通过映射还原技术,最终还原完整匹配结果。这种转化利用 BSM 领域的成熟技术(如高效匹配顺序优化),避免从更新边强制开始匹配的局限性。

与此同时,RapidFlow 引入了 对偶匹配技术,将查询边分组为 “自同构集”。对于同一集合中的边,只需计算其中一条边的增量匹配,其余边的匹配可通过顶点置换生成。例如,若查询图存在自同构映射 $M_Q$ 使边 $e_1$映射到 $e_2$,计算 $e_1$的匹配后,$e_2$的结果可通过 $M_Q$ 直接转换,避免重复搜索。实验显示,该技术在含自同构的查询中可减少 50\% 以上冗余计算。

(3) CaLiG

CaLiG(Candidate Lighting Graph)专为流图场景设计,通过索引优化和匹配流程重构降低回溯开销。

CaLiG 的核心思想是构建一个轻量的候选图索引结构,用以捕捉潜在匹配点的局部拓扑信息。其将匹配对 (u, v) 组织为有向图,用 “照明状态”(ON/OFF)标记 v 是否为 u 的有效候选。插入边时,通过评估邻接节点列表的注入匹配(Injective Matching)动态更新状态,利用双向传播机制剪枝无效候选。例如,若 (u, v) 的邻接匹配对无法形成有效子图同构,则标记为 OFF 并触发级联更新,减少后续回溯次数。

此外,CaLiG引入了核-壳搜索框架,将查询顶点分解为 “核顶点” 和 “壳顶点”。核顶点通过回溯确定匹配,壳顶点则利用核匹配的结果直接通过笛卡尔积生成匹配,避免重复扩展。例如,核顶点的匹配确定后,壳顶点的候选集可通过邻接关系快速过滤,无需逐层回溯。

\section{静态TopK密度优先级机制的子图匹配技术}
静态场景下的两种TopK密度优先级机制的子图匹配工作是第一次将子图密度和拓扑结构结合的工作,主要包括:基于拓扑索引与最大元路径权重的KiSD算法和基于基于等价顶点压缩与邻域过滤的PBSM算法。这两项工作不仅在静态图中实现了高效的top-K匹配能力,也为本文后续在动态图场景中的研究提供了理论与技术基础。
\subsection{基于拓扑索引与最大元路径权重的KiSD算法}
Gupta\cite{static-topk-Gupta-DBLP:conf/icde/GuptaGYCH14}等人在2014年提出的KiSD算法,针对异构信息网络中的 Top-K 有趣子图发现问题,通过离线索引构建与在线查询优化结合的方式,解决传统 “匹配后排序”(RAM)方法在大规模图数据中效率低下的问题。该算法核心包括以下两个阶段:

\textcircled{1}~离线索引构建

KiSD 算法的核心之一是离线阶段构建高效的索引结构,以支持后续的快速查询。该阶段主要包含两种索引:拓扑索引(Topology Index)和最大元路径权重索引(Maximum Metapath Weight Index, MMW Index)。

首先,拓扑索引用于存储每个节点在不同跳数范围内沿特定元路径所能访问到的邻居节点类型分布。具体来说,元路径是一种描述节点标签序列的模式,例如从类型 $A$ 到类型 $B$ 再到类型 $A$ 的路径序列($A \rightarrow B \rightarrow A$)。拓扑索引记录了节点 $v$ 在 $d$ 步跳数内,沿该元路径能够到达的各类邻居节点的数量。例如,若节点 $v$ 在 $d=2$ 跳范围内,沿元路径 $(B,A)$ 有 2 个类型为 $A$ 的邻居,则索引项表示为 Topology$(v, (B,A))=2$。该索引通过广度优先搜索遍历构建,能够快速反映节点的局部结构特征。

接着,最大元路径权重索引(MMW Index)与拓扑索引结构类似,但索引内容不同。它记录了节点沿元路径的最大边权重和,如果节点 $v$ 沿元路径 $(B,A)$ 的所有 $d$ 步路径中,最大边权重和为 5,则 MMW$(v, (B,A))=5$。该索引为后续的在线查询阶段提供了权重上界估算,从而实现高效的剪枝。

\textcircled{2}~ 在线查询处理

算法首先利用拓扑索引对候选节点进行过滤。具体来说,对于查询图中的节点 $u$,如果其在 $d$ 步内某条元路径邻居的数量多于数据图节点 $v$ 对应元路径邻居的数量,则节点 $v$ 无法匹配 $u$,直接剪枝剔除,减少搜索空间。随后采用“排序即匹配”(RWM)策略,维护一个大小为 $k$ 的堆,动态存储当前找到的最优匹配集合。算法按照边权重从大到小遍历边列表,从高权重边开始扩展候选匹配,借助MMW索引计算尚未匹配边的最大权重上界。如果该上界低于堆中当前最小匹配权重,则终止该候选的扩展,否则继续深度优先搜索,直到找到 top-k个匹配。这样,通过离线索引与在线剪枝相结合,KiSD 有效提升了在异构信息网络中 top-k有趣子图发现的效率和精度。

\subsection{基于等价顶点压缩与邻域过滤的PBSM算法}
Chen\cite{static-topk-Chen-DBLP:journals/ijprai/ChenLCTL18}等人 2018 年提出的 基于剪枝的子图匹配算法(Pruned-Based Subgraph Matching,PBSM),针对KiSD算法在大规模图中索引开销大、候选验证冗余的问题,从图压缩、邻域过滤和匹配顺序优化三方面改进,但其适用场景更偏向路径型查询。

首先,PBSM采用了一种无损图压缩策略,通过等价顶点定义及压缩限制策略来缩减图的规模。等价顶点被定义为:若两顶点标签相同、彼此无边相连且邻居类型及数量完全一致,假设数据图$G$中的节点$v_3$与$v_4$符合等价顶点条件,则可以将其压缩为单一节点,并保留原边中最大的权重,同时维护映射关系以保证匹配的正确性。压缩限制策略要求所有等价顶点之间不应存在边连接,以避免因压缩导致的路径信息丢失,即如果在数据图 $G$中$v_3$ 与 $v_4$ 之间存在边,压缩会造成匹配遗漏。

其次,在候选生成阶段,PBSM采用邻域结构过滤方法,仅基于查询节点的直接邻居标签进行过滤。例如,对于查询节点 $u$ 需要有两个类型为 $A$ 的邻居,则数据图节点 $v$ 必须至少具备两个类型为 $A$ 的邻居才能成为候选。该方法的时间复杂度为 $O(DT^{D+1})$,相较于KiSD中的最大元路径权重索引(MMW Index),具有更轻量的计算开销,但过滤精度相对略低。

最后,在匹配阶段,PBSM引入匹配顺序优化策略,以提升匹配效率。具体做法是优先选择候选集规模最小的查询节点作为匹配起点,沿路径向右递归直至终点,再反向递归回起点,最终获得路径型查询的匹配解。该策略依然采用了KiSD中“排序即匹配”(RWM)的权重上界估计思想,但结合压缩图中最大边权重,加快了上界计算过程。不过,由于压缩和邻域过滤的设计偏向路径结构,PBSM在非路径型查询图上的性能提升有限。



\section{本章小结}
本章介绍了图、子图匹配、连续子图匹配及密度约束下 Top-K 连续子图匹配的基本概念、应用背景和相关技术工作。
首先,回顾了图的定义与分类,重点讨论无向有权图及其在实际中的广泛应用。
随后,阐述了子图匹配的基本概念,介绍了回溯法、多向连接等典型算法及其优化策略,并说明其在多个实际场景中的应用。接着,引入连续子图匹配(CSM)的定义和应用场景,进一步提出了本文研究问题——基于密度约束的 Top-K 连续子图匹配,并强调其实际意义。
最后,回顾了与本研究相关的子图匹配代表性工作,为后续算法设计提供理论支撑。

