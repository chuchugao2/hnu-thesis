\chapter{相关知识}

\section{连续子图匹配(CSM)的定义与概念}
\subsection{图的基本定义与概念}
图是一种非常重要的数据结构,用于表述数据元素之间的多对多的关系。
它由一组节点(顶点)和一组边构成,其中每一条边连接两个节点,表示它们之间的关系。
在计算机科学中,图被广泛应用于表示实体间的结构化关系,其数学定义为一个二元组,
$G=(V,E)$,其中$V$是节点的结合,$E$是边的集合,即:
\begin{align*}
  V &= \{\ v_1, v_2, \dots, v_n\ \},\\
  E &\subseteq V \times V
\end{align*}

根据边的特性,图可以根据不同的划分标准进行分类。首先,根据边的方向性可分为有向图(图\ref{fig:example_noweight})与无向图(图\ref{fig:example_weight})。有向图中的边具有明确的方向的,每条边表示一种特定的方向,表示从起点指向终点的有向关系。如图\ref{fig:example_noweight}中所示$<v_1',v_2'>$,表示一条有向边,$v_1'$是起点,$v_2'$是终点,表示从节点 $v_1'$ 指向节点 $v_2'$ 的有向边。
与之相对,无向图的边则没有方向性,即两个相连的节点可以互相访问,因此无向图又可以称之为特殊的有向图,因为每条边都代表着一个双向可达关系。如图\ref{fig:example_weight}中所示$<v_1,v_2>$,表示一条无向边,其可以视为由两条有向边$<v_1,v_2>$和$<v_2,v_1>$组成。
此外,根据边上是否附带权重,图还可以进一步分为有权图(图\ref{fig:example_weight})和无权图(图\ref{fig:example_noweight})。
无权图中的边仅表示两个节点间的可达关系,不携带额外的数值信息;而在有权图中,边的权重承载了附加信息,如距离、费用、强度等指标。

\begin{figure}[h!]
    \def\wscorevone{0.48}
    \centering
        \begin{subfigure}[t]{\wscorevone\linewidth}
            \centering
            \resizebox{\linewidth}{!}
            {
                \includegraphics{\figs youxiangwuquan.pdf}
            }
            \caption{有向无权图}
            \label{fig:example_noweight}
        \end{subfigure}
        \hfill
        \begin{subfigure}[t]{\wscorevone\linewidth}
            \centering
            \resizebox{\linewidth}{!}
            {
                \includegraphics{\figs wuxiangyouquan.pdf}
            }
            \caption{无向有权图}
            \label{fig:example_weight}
        \end{subfigure}
        \label{fig:definition}
        \caption{图的常见分类}
    \end{figure}

图(Graph)这种数据结构能够使用非常简洁的形式来表达复杂的数据。
具体而言,图结构擅长表达形式多样的数据结构,特别是在分析具有明确主谓宾结构的句子时,其能够通过节点表示实体、边表示关系,进而清晰展现实体间的关联关系。
%利用图这种数据结构,它能够很好的表达陈述句中的主谓宾,即以主语和宾语为对象,谓语作为两个对象之间的关联关系,这保证了它对于形式多样的复杂数据有非常强的表达能力。
生活中的各式各样的信息都能够通过主谓宾的形式分解并构建图模型,更重要的是,生活中大多数构建的图模型的边都具有权值,表示特定的含义。
例如在金融支付应用中,“用户A向用户B转账100”这一场景中,其中“用户A”和“用户B”分别表示图中的两个节点,而“转账”则表示它们之间的边,边上的权重值“100”则代表了转账的金额。
因此,图模型在现实世界中具有广泛的应用,能够清晰展示对象之间的关联关系,并帮助数据分析人员深入理解和挖掘数据背后的潜在规律。
在本文中,我们主要以无向有权图作为主要分析对象,并且所提出的算法也同样适用于有向图,以期为更复杂的图结构分析提供借鉴。

\subsection{子图匹配的概念与应用}
子图匹配问题是图论中的一个经典且重要的问题。给定一个查询图(Query Graph)和一个数据图(Data Graph),子图匹配的任务是寻找数据图$G$中所有能够与查询图$Q$完全匹配的子图。
具体而言,子图匹配涉及以下几个关键概念:
\begin{figure}[h!]
    \centering
    \resizebox{0.8\linewidth}{!}{
        \includegraphics{\figs e_subgraph_matching.pdf}
    }
    \caption{子图匹配示例}
    \label{fig:example_subgraph_matching}
\end{figure}
\begin{itemize}
    \item \textbf{子图匹配},也称为子图同构(Subgraph Isomorphism):即给定一个查询图$Q$以及一个数据图$G$,查询图 $Q$ 与数据图 $G$ 中的某个子图 $g$ 是同构的,当且仅当存在一个双射映射,使得查询图中的每个节点都能够与数据图中的一个唯一节点一一对应,并且查询图中的每一条边在数据图中也有对应的边。
    %$Q$子图同构于$G$中的一个子图g,当且仅当存在一个双射函数,对于Q中的每一个节点,在g中都能到找到与之对应的节点。对于Q中的每一条边,在g中都能找到与其对应的边。即查询图Q和子图g之间存在一个一一对应的节点和边的映射关系,换句话说就是查询图的结构能够完全嵌入到数据图中。
    例如,图\ref{fig:example_subgraph_matching}a为查询图$Q$,图\ref{fig:example_subgraph_matching}b为数据图$G$。
    在数据图$G$中,子图$<v_1,v_2,v_3>$是查询图$<u_1,u_2,u_3>$的一个子图匹配结果,图中虚线表示他们的映射关系。
    \item \textbf{子图匹配任务}:子图匹配的目标是在数据图中找到所有能与查询图完全匹配的子图集合。这个集合中的每一个子图都可以与查询图形成子图同构关系。
    \item \textbf{完全匹配(Full Match)}:完全匹配其实就是一次完整的子图同构,查询图中的所有节点和边都在数据图的某个子图中找到一一对应的映射。
    \item \textbf{部分匹配(Partial Match)}:部分匹配是指查询图中的一部分节点和边在数据图中找到了对应的关系,但不要求查询图的所有节点都能够找到映射。
\end{itemize}

子图匹配问题的经典算法主要可分为两类:一类基于深度优先搜索(DFS)加回溯的方式\cite{sm-ullmann-DBLP:journals/jacm/Ullmann76},另一类基于广度优先搜索(BFS)及多路联接(Multi-way Join)的方式\cite{sm-bfs-DBLP:conf/focs/AtseriasGM08}。

基于深度优先搜索加回溯的方式: 给一个查询图$Q$,首先定义节点被匹配的顺序,如$<u_1,u_2,u_3>$,即按照$u_1$,$u_2$,$u_3$的顺序进行节点匹配,如果当前匹配无法继续,则回溯;若需要找全部的解集,则必须进行回溯。
该算法的优点在于能够避免产生大量中间结果,因为采用深度优先搜索(DFS)时,只有递归调用栈的空间,且没有显式的中间结果存储。
然而,该方法的缺点是并行性较差,且回溯过程中会产生较大的递归开销。

基于广度优先的多路联接方法:广度优先搜索算法的基本思想与关系数据库中的 Join 操作类似,就是不同候选解之间的Join操作。
在匹配过程中,首先为每个节点计算出候选解集,然后通过边的相邻关系执行Join操作,最终得到符合条件的子图匹配结果集。
由于此算法采用宽度优先搜索(BFS),每个节点在搜索过程中都有对应的候选解,因此非常适合并行处理。本文所提出的框架中的 Worst Case Optimal Join 算法\cite{sm-bfs-DBLP:conf/focs/AtseriasGM08} 即属于这种方法。


子图匹配问题在多个领域具有广泛的应用价值。具体来说,在生物信息学中,子图匹配用于识别分子结构中的相似模式,帮助研究人员发现潜在的药物靶点或生物标志物;
在社交网络分析中,子图匹配用于发现用户之间的潜在关系或者行为模式,辅助分析社交关系中的传播路径、社区发现等问题;在网络安全领域中,子图匹配用于检测恶意活动的特征或攻击模式,通过匹配已知的攻击图谱来识别潜在的网络安全威胁。

尽管子图匹配在实际应用中非常重要,但由于其判定是 NP 完全问题,列举所有的子图匹配位置是 NP 难的,这意味着子图匹配的计算复杂度会随着图规模的增大而迅速上升。
因此,在处理大规模图数据时,必须采用高效的算法,通过剪枝和优化策略降低计算复杂度。

\subsection{连续子图匹配的概念与应用}
连续子图匹配(CSM,Continuous Subgraph Matching)是在子图匹配的基础上,研究如何在动态变化的图中,持续保持子图匹配的有效性和准确性。与静态子图匹配不同,连续子图匹配主要涉及以下特点:
\begin{itemize}
   \item \textbf{连续子图匹配}:在动态图中,图的结构随着时间不断变化,节点和边可能会被增加或删除。因此,连续子图匹配需要在这些变化中,维持已有子图匹配结果的有效性,避免每次变化都从头开始重新计算匹配结果。
   \item \textbf{增量更新}:在动态图中,子图匹配不仅是对静态图进行一次性的匹配操作,而是需要根据图的增量变化实时更新匹配结果。增量更新的目标是通过最小化每次更新的计算代价,提升匹配算法的效率。
   \item \textbf{匹配结果的更新与维护}:连续子图匹配的核心问题之一是如何高效地维护和更新匹配结果。随着边的增加或删除,部分匹配可能会失效,此时需要重新计算匹配。但另一方面,有些匹配可能依然有效,可以通过增量更新来保留这些计算结果,避免不必要的重复计算。
\end{itemize}

在现实生活中,大部分数据图并非静态不变,因此连续子图匹配具有广泛的应用前景。例如,社交网络中的用户关系是动态变化的,对社交网络中的用户子图进行连续匹配,可以识别潜在的社交圈子、群体或行为模式;
在交通网络中,路况和道路连接经常发生变化,实时监控交通状态并进行子图匹配,能够帮助分析交通流量、预测交通拥堵等;在金融网络中,交易关系和账户之间的连接不断变化,检测异常的金融行为或交易模式需要利用连续子图匹配技术。


\section{密度约束下TopK连续子图匹配的概念}
\begin{figure}[h!]
    \def\wscorevone{0.49}
    \centering
        \begin{subfigure}[t]{\wscorevone\linewidth}
            \centering
            \resizebox{\linewidth}{!}
            {
                \includegraphics{\figs e_attack_pattern.pdf}
            }
            \caption{子网攻击模式~\cite{static-topk-Gupta-DBLP:conf/icde/GuptaGYCH14}}
            \label{fig:example_attack_pattern}
        \end{subfigure}
        \hfill
        \begin{subfigure}[t]{\wscorevone\linewidth}
            \centering
            \resizebox{\linewidth}{!}
            {
                \includegraphics{\figs e_traffic_jam.pdf}
            }
            \caption{交通拥堵模式~\cite{traffic-graph-matching-DBLP:journals/pvldb/SongGCW14}}
            \label{fig:example_traffic_jam}
        \end{subfigure}
        \label{fig:definition}
        \caption{CSM-TopK示例}
    \end{figure}
本文研究的主要问题是在连续子图匹配问题的基础上,结合TopK密度剪枝策略,旨在有效获取前k个具有最高密度的子图匹配结果。
2014年,Manish Gupta等人首次提出了子图匹配中密度优先级的概念\cite{static-topk-Gupta-DBLP:conf/icde/GuptaGYCH14}。其中,子图的密度定义为其边权之和,即边权总和越大,子图的密度优先级越高。
然而,该研究主要聚焦于静态图中的密度优先级,并且提出的静态图索引无法有效应用于动态场景。
至今,尚未有研究关注动态场景中,TopK密度优先级机制对子图匹配结果的影响。

因此,本文提出并研究一个新问题,即在动态场景下,如何实时计算并获取前k个具有最高密度的子图匹配结果。
这个问题在实际应用中具有广泛的应用前景。
例如,在通信网络中,不同的攻击模式可能具有不同的分析优先级,尽管它们可能具有相同的结构模式。
图\ref{fig:example_attack_pattern}展示了通信网络中的攻击模式\cite{static-topk-Gupta-DBLP:conf/icde/GuptaGYCH14},其中高数据传输速率的攻击模式应当具有更高的响应优先级。快速识别这些高数据传输速率的模式,有助于网络管理员及时找到受损最严重的子网络。
同样,图\ref{fig:example_traffic_jam}展示了一种典型的交通拥堵模式\cite{traffic-graph-matching-DBLP:journals/pvldb/SongGCW14},在道路网络中,交通流量较高的模式通常意味着拥堵程度较高,交管系统需要迅速采取应对措施。上述两种模式均涉及到动态变化的场景,因此,基于动态图中的密度约束进行TopK连续子图匹配,其应用价值远高于静态场景中的相同问题。

\section{本章小结}
本章介绍了图的基本概念、子图匹配和连续子图匹配(CSM)。
首先,回顾了图的定义与分类,重点讨论了无向有权图,并介绍了图模型在实际应用中的广泛应用。
接着,阐述了子图匹配的基本概念,并介绍了基于深度优先搜索的回溯方法以及广度优先的多向联接方法等经典算法。
最后,本章探讨了连续子图匹配(CSM)的定义和基本概念,并介绍了其在社交网络、交通网络等领域的应用。
基于这些基础,本章提出了一个新的研究问题——在动态场景下如何实时计算并获取前K个具有最高密度的子图匹配结果。