\chapter{相关知识}

\section{连续子图匹配(CSM)的定义与概念}
本小节主要介绍连续子图匹配(CSM)的基本概念及其应用。
首先,我们将介绍图的基本定义,包括不同类型的图结构(如有向图、无向图、加权图和无权图)
接着,我们探讨子图匹配的核心概念,即在数据图中查找与查询图匹配的子图,并分析不同的匹配方式(如完全匹配与部分匹配)。
随后,我们介绍两种经典的子图匹配算法,并总结了其常见的优化策略。
最后,我们讨论连续子图匹配(CSM)的概念,并结合实际场景,分析其在金融风控、网络安全、社交网络分析等领域的广泛应用。
\subsection{图的基本定义与概念}
图是一种非常重要的数据结构,用于呈现数据元素之间的多对多的复杂关系。
它由一组节点(顶点)和一组边组成,其中每条边 连接两个节点,以描述它们之间的关联关系。
在计算机科学中,图被广泛应用于表示实体间的结构化关系,其数学定义为一个二元组,
$G=(V,E)$,其中$V$是节点的结合,$E$是边的集合,即:
\begin{align*}
  V &= \{\ v_1, v_2, \dots, v_n\ \},\\
  E &\subseteq V \times V
\end{align*}

根据边的特性,图可以根据不同的划分标准进行分类。首先,根据边的方向性可分为有向图(图\ref{fig:example_noweight})与无向图(图\ref{fig:example_weight})。有向图中的边具有明确的方向的,每条边表示一种特定的方向,表示从起点指向终点的有向关系。如图\ref{fig:example_noweight}中所示$<v_1',v_2'>$,表示一条有向边,$v_1'$是起点,$v_2'$是终点,表示从节点 $v_1'$ 指向节点 $v_2'$ 的有向边。
与之相对,无向图的边则没有方向性,即两个相连的节点可以互相访问,因此无向图又可以称之为特殊的有向图,因为每条边都代表着一个双向可达关系。如图\ref{fig:example_weight}中所示$<v_1,v_2>$,表示一条无向边,其可以视为由两条有向边$<v_1,v_2>$和$<v_2,v_1>$组成。
此外,根据边上是否附带权重,图还可以进一步分为有权图(图\ref{fig:example_weight})和无权图(图\ref{fig:example_noweight})。
无权图中的边仅表示两个节点间的可达关系,不携带额外的数值信息;而在有权图中,边的权重承载了附加信息,如距离、费用、强度等指标。

\begin{figure}[h!]
    \def\wscorevone{0.48}
    \centering
        \begin{subfigure}[t]{\wscorevone\linewidth}
            \centering
            \resizebox{\linewidth}{!}
            {
                \includegraphics{\figs youxiangwuquan.pdf}
            }
            \caption{有向无权图}
            \label{fig:example_noweight}
        \end{subfigure}
        \hfill
        \begin{subfigure}[t]{\wscorevone\linewidth}
            \centering
            \resizebox{\linewidth}{!}
            {
                \includegraphics{\figs wuxiangyouquan.pdf}
            }
            \caption{无向有权图}
            \label{fig:example_weight}
        \end{subfigure}
        \label{fig:definition}
        \caption{图的常见分类}
    \end{figure}

图(Graph)这种数据结构能够使用非常简洁的形式来表达复杂的数据。
具体而言,图结构擅长表达形式多样的数据结构,特别是在分析具有明确主谓宾结构的句子时,其能够通过节点表示实体、边表示关系,进而清晰展现实体间的关联关系。
%利用图这种数据结构,它能够很好的表达陈述句中的主谓宾,即以主语和宾语为对象,谓语作为两个对象之间的关联关系,这保证了它对于形式多样的复杂数据有非常强的表达能力。
现实世界中的各类信息往往可以拆解为主谓宾结构,进而构建相应的图模型。更重要的是,大多数实际应用中构建的图模型不仅关注节点之间的关系,还会为边赋予权值,以表示特定的含义。
以金融支付场景为例,“用户A向用户B转账100元”可以建模为一个图结构,其中“用户A”和“用户B”分别表示图中的两个节点,而“转账”则表示它们之间的边,边上的权重值“100”则代表了转账的金额。
通过这种方式,图模型能够清晰地表示对象之间的关联,并为数据分析提供结构化支持,帮助研究人员挖掘潜在的规律和模式。
% 因此,图模型在现实世界中具有广泛的应用,能够清晰展示对象之间的关联关系,并帮助数据分析人员深入理解和挖掘数据背后的潜在规律。
在本文中,我们主要以无向有权图作为主要分析对象,并且所提出的算法也同样适用于有向图,以期为更复杂的图结构分析提供借鉴。

\subsection{子图匹配的概念与应用}
子图匹配问题是图论中的一个经典问题\cite{sm-ullmann-DBLP:journals/jacm/Ullmann76}。给定一个查询图(Query Graph)和一个数据图(Data Graph),子图匹配的任务是寻找数据图$G$中所有能够与查询图$Q$完全匹配的子图。
具体而言,子图匹配涉及以下几个关键概念:
\begin{figure}[h!]
    \centering
    \resizebox{0.8\linewidth}{!}{
        \includegraphics{\figs e_subgraph_matching.pdf}
    }
    \caption{子图匹配示例}
    \label{fig:example_subgraph_matching}
\end{figure}
\begin{itemize}
    \item \textbf{子图同构(Subgraph Isomorphism)}:即给定一个查询图$Q$以及一个数据图$G$,查询图 $Q$ 与数据图 $G$ 中的某个子图 $g$ 是同构的,当且仅当存在一个双向映射,使得查询图中的每个节点都能够与数据图中的一个唯一节点一一对应,并且查询图中的每一条边在数据图中也有对应的边。
    %$Q$子图同构于$G$中的一个子图g,当且仅当存在一个双射函数,对于Q中的每一个节点,在g中都能到找到与之对应的节点。对于Q中的每一条边,在g中都能找到与其对应的边。即查询图Q和子图g之间存在一个一一对应的节点和边的映射关系,换句话说就是查询图的结构能够完全嵌入到数据图中。
    例如,图\ref{fig:example_subgraph_matching}a为查询图$Q$,图\ref{fig:example_subgraph_matching}b为数据图$G$。
    在数据图$G$中,子图$<v_1,v_2,v_3>$与查询图$<u_1,u_2,u_3>$同构,图中虚线表示他们的映射关系。
    \item \textbf{子图匹配(Subgraph Matching)}:子图匹配的目标是在数据图中找到所有能与查询图完全匹配的子图集合。这个集合中的每一个子图都可以与查询图形成子图同构关系。
    \item \textbf{完全匹配(Full Match)}:完全匹配其实就是一次完整的子图同构,查询图中的所有节点和边都在数据图的某个子图中找到一一对应的映射。
    \item \textbf{部分匹配(Partial Match)}:部分匹配是指查询图中的一部分节点和边在数据图中找到了对应的关系,但不要求查询图的所有节点都能够找到映射。
\end{itemize}


子图匹配问题的经典算法主要可分为两类:一类基于深度优先搜索(DFS)加回溯的方式\cite{sm-ullmann-DBLP:journals/jacm/Ullmann76},另一类基于广度优先搜索(BFS)及多路联接(Multi-way Join)的方式\cite{sm-bfs-DBLP:conf/focs/AtseriasGM08}。

基于深度优先搜索和回溯的方式: 该算法首先定义查询图$Q$中节点的匹配顺序,如$<u_1, u_2, u_3>$,然后通过深度优先搜索(DFS)逐步匹配数据图中的相应节点。
如果某一步匹配失败,算法会回溯并尝试其他可能的匹配路径。为了找到所有可能的匹配结果,必须进行完整的回溯搜索。
该算法的优点在于能够有效减少中间结果的存储,因为深度优先搜索仅依赖递归调用栈,无需显式存储大量中间匹配状态,从而降低了内存消耗。
并且,由于回溯仅在搜索路径失效时进行,可以通过路径剪枝技术降低计算量。
然而,该方法也存在一定的局限性:首先DFS 本质上是串行执行的,难以高效并行化,且在回溯过程中会产生较大的递归开销。
其次,在大规模图数据中,可能的匹配路径数量指数级增长,导致计算时间大幅增加,影响算法的实际可用性。

基于广度优先的多路联接方法: 广度优先搜索(BFS)算法在子图匹配中的应用与关系数据库中的Join操作相似。
首先为查询图$Q$中的每个节点计算候选解集(即数据图中所有可能匹配的节点集合),然后通过边的相邻关系执行多路联接操作,最终得到符合条件的子图匹配结果。
与深度优先搜索相比,广度优先搜索的优势在于其具有更好的并行性,因为每个节点的候选解在搜索过程中是独立计算的,且候选解集较为稳定。
因此,BFS特别适合在多核或分布式环境中进行并行计算。
例如,本文所提出的框架中的Worst Case Optimal Join算法\cite{sm-bfs-DBLP:conf/focs/AtseriasGM08}采用了这种广度优先搜索与多路联接结合的策略,能够在理论上保证最坏情况下的最优性能。

子图匹配在多个领域具有广泛的应用价值,尤其在处理复杂结构和大规模图数据时,能够提供重要的分析工具。
在生物信息学领域,子图匹配被广泛应用于分子结构的分析与比较\cite{窦建凯2019单图中的近似频繁子图挖掘算法}。通过识别分子结构中的相似模式,研究人员能够发现潜在的药物靶点或生物标志物,在药物研发过程中发挥重要作用。
此外,在蛋白质结构分析和基因表达数据的处理过程中\cite{wang2023zerobind,biology-proteins-DBLP:journals/nar/XenariosSDHKE03,uniprot2017uniprot},子图匹配有助于揭示生物体内的复杂关系和相互作用。
在社交网络分析中,子图匹配同样至关重要。它能够识别社交网络中的互动模式和社区结构,用于分析社交行为、信息传播路径以及群体演化\cite{wang2023mago}。
同时,通过识别用户交互模式中的特定子图结构,可有效检测定向广告投放异常、垃圾邮件传播路径及欺诈行为网络\cite{boshmaf2011socialbot,jiang2012isolating,wang2010don}。
%在个性化推荐和用户行为预测等应用中,子图匹配可用于快速发现具有相似结构或行为的用户群体,从而提升推荐系统的精准度。
网络安全领域也高度依赖子图匹配算法。例如,在入侵检测系统(IDS)中,子图匹配可用于识别恶意行为特征或攻击模式\cite{yuan2023motif}。
通过与已知攻击图谱进行匹配,系统能够迅速检测潜在的安全威胁。此外,子图匹配还能帮助识别异常行为,提高系统在应对零日攻击和复杂攻击时的防御能力。

尽管子图匹配在诸多领域展现出巨大潜力,但由于其判定问题属于 NP 完全问题,列举所有子图匹配位置的计算复杂度随着图规模的增加而迅速上升,导致在大规模数据处理中面临严峻挑战。
为此,研究人员提出了剪枝优化、索引预处理、并行计算等方法,以提升子图匹配算法的效率,使其更适用于实际应用场景。

首先,对深度优先搜索加回溯的算法而言,由于其逐步深入的搜索方式,回溯过程中可能会产生大量的冗余计算。
为了解决这一问题,剪枝技术被广泛应用。
剪枝技术能够通过提前排除不可能的匹配,从而减少不必要的计算。
已有的一些研究工作如GraphQL\cite{sm-GraphQL-DBLP:series/ads/HeS10},SPath\cite{sm-spath-DBLP:journals/pvldb/ZhaoH10},
都通过构建索引结构和一些剪枝策略对候选集合进行剪枝。
例如,GraphQL\cite{sm-GraphQL-DBLP:series/ads/HeS10}结合标签匹配和拓扑结构,提出基于伪子图同构的过滤框架,实现高效剪枝。
SPath\cite{sm-spath-DBLP:journals/pvldb/ZhaoH10}则是通过路径分解和邻居信息索引来快速计算子图匹配结果。

除了剪枝策略,预处理同样是解决子图匹配效率问题的有效优化手段。 
预处理算法的核心,是在子图匹配算法执行枚举操作前,增设一个预处理环节。
在此环节,通过降低每个查询顶点候选集的规模,获取更为精确的统计数据,进而优化查询匹配的先后顺序。
具体来说,在预处理阶段,会为每个查询顶点构建一个完备的候选顶点集合。后续在进行子图匹配时,算法将从这些候选集合中挑选匹配节点。
已有的一些工作如TurboIso\cite{sm-turbo-iso-DBLP:conf/sigmod/HanLL13},CECI\cite{sm-ceci-DBLP:conf/sigmod/BhattaraiLH19},DP-iso\cite{sm-dp-iso-DBLP:conf/sigmod/HanKGPH19}均采用预处理框架来减少子图匹配过程中的搜索空间。

此外,针对大规模图数据,许多基于并行和分布式计算的优化方法也得到了广泛应用。
由于基于广度优先的子图匹配算法天然具有较好的并行性,研究者们提出了基于多核处理器和GPU加速的子图匹配算法。
现阶段,Stwing\cite{sm-stwing-DBLP:journals/pvldb/SunWWSL12},PSgL\cite{sm-psgl-DBLP:conf/sigmod/ShaoCCMYX14}在应对子图匹配问题时,均运用了并行处理策略。
这些方法通过将搜索任务分配到多个处理单元并行执行,有效地减少了整体计算时间。
例如,研究者们利用图的稀疏性和局部性,设计了基于图分割的并行算法,即将大规模数据图划分为多个子图,在各个子图内独立进行匹配操作,并最终合并匹配结果。
这类方法不仅能够显著提升匹配速度,还能够扩展子图匹配算法的适用范围,使其能够高效处理更大规模的图数据。


\subsection{连续子图匹配的概念与应用}
\label{csm-concept}
连续子图匹配(CSM,Continuous Subgraph Matching)是指在图的结构发生动态变化时,如何有效地持续计算并维护子图匹配的结果\cite{wang2023survey}。
传统的子图匹配主要针对静态图,在图的结构保持不变的情况下,可以通过一次性计算获得匹配结果。
然而,在现实应用场景中,图往往是动态变化的,即节点和边可能不断新增或删除,这使得静态子图匹配方法无法满足实时性需求。
因此,CSM 需要在图结构变化时,高效更新匹配结果,确保其实时性和有效性。
与静态子图匹配不同,连续子图匹配(CSM)主要涉及以下几个核心特点:
\begin{itemize}
   \item \textbf{连续子图匹配}:在动态图中,图的结构随着时间不断变化,节点和边可能会被增加或删除。因此,连续子图匹配需要在这些变化中,维持已有子图匹配结果的有效性,避免每次变化都从头开始重新计算匹配结果。
  % \item \textbf{增量更新}:在动态图中,子图匹配不仅是对静态图进行一次性的匹配操作,而是需要根据图的增量变化实时更新匹配结果。增量更新的核心思想是通过最小化每次更新的计算代价,仅针对图的变化部分进行局部调整,而非重新计算所有匹配。通过增量更新,可以大幅度减少计算开销,提升匹配算法的效率。
   \item \textbf{匹配结果的维护}:连续子图匹配的核心问题之一是如何高效地维护结果。在动态图中,随着节点和边的增加或删除,某个匹配结果会新增,而某些原有的匹配结果可能会失效,导致匹配结果需要重新调整。此时,需要高效的结果维护策略,实时维护当前所有的子图匹配结果,显著提高匹配的响应速度。
\end{itemize}

在现实生活中,大部分数据图并非静态不变,而是随着时间不断演化。因此,连续子图匹配具有广泛的应用前景。
例如,社交网络中的用户关系和行为模式处于持续变化之中。用户的兴趣、活动、互动方式随时间推移不断演变,因此通过连续子图匹配,我们可以实时追踪和分析社交圈子或行为模式的演变,进而识别潜在的用户群体或社交动态\cite{wang2023mago}。
%具体而言,基于用户行为的实时分析可以揭示群体之间的变化趋势,或者帮助检测异常行为的出现,例如通过匹配特定的子图模式,发现潜在的社交关系网络中的突变或异常行为。
在交通网络中,道路连接状态、交通流量、施工信息、事故情况等因素都会导致图结构的动态变化\cite{traffic-graph-matching-DBLP:journals/pvldb/SongGCW14}。
因此,利用连续子图匹配技术,交通管理系统可以实时监控交通状况并预测可能的拥堵或事故区域。
%通过动态匹配交通网络中的子图,系统能够及时识别出哪些道路段可能出现交通堵塞,从而为交通信号的优化提供数据支持,甚至可以根据不同路段的变化及时调整交通管控措施,优化道路通行效率,缓解交通压力,减少拥堵。
在金融网络中,不同金融机构之间的资金流动、账户交易关系不断变化\cite{csm-timing-DBLP:conf/icde/Li0O019}。
%子图匹配技术能够帮助实时检测潜在的异常行为或非法资金流动,利用连续子图匹配可以分析交易模式的变化,及时发现和响应异常行为。
通过匹配历史交易模式与当前交易子图,金融监控系统能够识别出潜在的欺诈行为,并采取相应的应对措施,从而防止金融欺诈的蔓延。
此外,连续子图匹配还广泛用于知识图谱领域。它可以处理随时间推移而演变的知识,以支持连续的问答\cite{barbieri2010c},并且RDF图的推理也基于这种模式监控\cite{dell2014incremental}。
因此,连续子图匹配的研究不仅能够在学术领域提供新的视角,也能够在诸如社交网络分析、交通监控、金融反欺诈等实际应用中发挥巨大的作用。

\section{密度约束下TopK连续子图匹配的概念与应用}
在子图匹配的研究中,如何在大量候选匹配中快速选择最具代表性的子图是一个重要问题。
传统的子图匹配方法大多关注如何找到所有可能的匹配,而在实际应用中,我们往往更关注某些特定的子图,特别是那些密度较高、结构更为紧密的子图。
且在2014年,Manish Gupta等人首次提出了子图匹配中密度优先级的概念\cite{static-topk-Gupta-DBLP:conf/icde/GuptaGYCH14},其基于子图的密度对匹配结果进行排序,选择密度最高的前$k$个匹配。
然而,该研究主要聚焦于静态图中的密度优先级,对于动态图中的应用并未进行深入探讨。并且提出的静态图索引无法有效应用于动态场景。
至今,尚未有研究关注动态场景中,TopK密度优先级机制对子图匹配结果的影响。
因此,在动态场景下引入密度约束并结合TopK匹配策略成为了一个重要的研究方向。

\begin{figure}[h!]
    \def\wscorevone{0.49}
    \centering
        \begin{subfigure}[t]{\wscorevone\linewidth}
            \centering
            \resizebox{\linewidth}{!}
            {
                \includegraphics{\figs e_attack_pattern.pdf}
            }
            \caption{子网攻击模式~\cite{static-topk-Gupta-DBLP:conf/icde/GuptaGYCH14}}
            \label{fig:example_attack_pattern}
        \end{subfigure}
        \hfill
        \begin{subfigure}[t]{\wscorevone\linewidth}
            \centering
            \resizebox{\linewidth}{!}
            {
                \includegraphics{\figs e_traffic_jam.pdf}
            }
            \caption{交通拥堵模式~\cite{traffic-graph-matching-DBLP:journals/pvldb/SongGCW14}}
            \label{fig:example_traffic_jam}
        \end{subfigure}
        \label{fig:definition}
        \caption{CSM-TopK示例}
    \end{figure}

本文提出并研究一个新问题——\textbf{密度约束下TopK连续子图匹配},即在\ref{csm-concept}节中定义的连续子图匹配概念的基础上,引入密度优先级机制(见\ref{ch3:definition}节),旨在有效获取前k个具有最高密度的子图匹配结果。
这一问题在实际应用领域具有广阔的应用空间。

例如,在通信网络中,给定固定的网络攻击模式可能具有不同的分析优先级。
图\ref{fig:example_attack_pattern}展示了通信网络中的攻击模式\cite{static-topk-Gupta-DBLP:conf/icde/GuptaGYCH14},其中高数据传输速率的攻击模式应当具有更高的响应优先级。
快速识别这些高数据传输速率的模式,有助于网络管理员及时找到受损最严重的子网络。
同样,图\ref{fig:example_traffic_jam}展示了一种典型的交通拥堵模式\cite{traffic-graph-matching-DBLP:journals/pvldb/SongGCW14},在道路网络中,交通流量较高的模式通常意味着拥堵程度较高,交管系统需要迅速采取应对措施。
快速识别交通流量较高的前$k$个匹配模式,可以让交管系统优先处理堵塞严重的区域,从而快速缓解交通拥堵情况。
上述两种模式均涉及到动态变化的场景,因此,基于动态图中的密度约束进行TopK连续子图匹配,其应用价值远高于静态场景中的相同问题,也远高于动态场景下的子图匹配问题。

\section{本章小结}
本章介绍了图的基本概念、子图匹配、连续子图匹配(CSM)以及本文的研究问题。
首先,回顾了图的定义与分类,重点讨论了无向有权图,并介绍了图模型在实际应用中的广泛应用。
接着,阐述了子图匹配的基本概念,介绍了基于深度优先搜索的回溯方法以及广度优先的多向联接方法等经典算法以及子图匹配的几种常见优化方式和在现实生活中的应用。
最后,本章探讨了连续子图匹配(CSM)的定义和基本概念,并介绍了其在社交网络、交通网络等领域的应用。
基于这些基础,本章提出了一个新的研究问题——如何实现基于密度约束的TopK连续子图匹配,并指出其实际的应用价值。