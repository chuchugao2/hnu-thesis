\def\itk{KiSD}
\def\pm{PBSM}
\chapter{实验设计与结果分析}
\label{ch5:experiment}
\section{实验设置}
\subsection{实验环境}
本节对我们提出的解决方案进行了评估。
所有方法均采用C++实现,并在配置为64GB内存、Intel(R) 4210R CPU @ 2.40GHz的CentOS服务器上运行。
实验所用的所有代码和数据集可在GitHub上获取~\cite{code-git-csmtok}。
实验所依赖软硬件操作平台配置如表\ref{table:setup}所示。
\begin{table}[H]
    \centering
    \caption{实验环境配置}
    \label{table:setup}
    \begin{tabular}{cc}
        \toprule
        描述   & 型号  \\
        \midrule
        CPU处理器 & Intel(R) 4210R CPU @ 2.40GHz\\
        内存 & 64G \\
        硬盘 & 1T\\
        操作系统 & CentOS Linux 7 (Core)\\
        编程语言 & C++ \\
        \bottomrule
    \end{tabular}
\end{table}

\subsection{数据集和评价标准}
\label{ss-sec:dataset}
参考之前的CSM的研究工作~\cite{csm-survey:DBLP:journals/pvldb/SunSLH22,static-sm:DBLP:conf/sigmod/Sun020},我们在实验中使用了五个数据集,具体数据集包括:
\begin{itemize}
\item Amazon 数据集:一个从亚马逊网站爬取的产品购买网络图。
\item LiveJournal 数据集:包含一个社区网络数据图。
\item Human 数据集:包含一个用于建模蛋白质相互作用网络的大型图。
\item YouTube 数据集:包含一个视频分享网站的社交网络。
\item Orkut 数据集:来自一个免费的在线社交网络,用户在其中互相建立友谊。
\end{itemize}   

具体测试的数据集规模如表\ref{table:dataset}所示,表中列出了每个数据集的顶点数,边数,顶点标签数,边标签数,以及图的平均度。
这些数据集的边最大可超过亿级,能够有效展示该算法在处理大规模数据图时的性能表现。通过实验,我们能够全面评估所提出的最终方案在不同规模的数据图中的时间和空间效率。

\begin{table}[H]
    \centering
    \caption{数据集信息描述}
    \label{table:dataset}
    \begin{tabular}{cccccc}
        \toprule
        数据集   & $|V|$  & $|E|$ & $|L(V)|$ & $ |L(E)|$ & $degr(G)=\dfrac{2|E|}{|V|}$\\
        \midrule
        Amazon    & 403,394 & 1,015,000 & 6 & 1 & 5.03     \\ 
        LiveJournal   & 4,847,571 & 30,005,000 & 30 & 1 & 12.38 \\ 
        Human  & 4,674 & 81,282  & 44 & 1 & 34.78   \\ 
        YouTube  & 1,134,890 & 2,015,000 & 25 & 1 & 3.55  \\ 
        Orkut  & 3,072,441 & 117,185,083 & 10 & 1 & 76.28  \\ 
        \bottomrule
    \end{tabular}
\end{table}

请注意,以上数据集中的边均为无权边。对于每条边 $(v_1, v_2)$,我们为其分配一个权重,计算方式为:

\[
    \frac{(N(v_1) \cap N(v_2)) + 1}{(N(v_1) \cup N(v_2)) + 1}
\]


结果保留六位小数。该方法生成的边权重反映了顶点邻域之间的连接程度。
对于上述数据集,我们通过从每个图中随机抽取 $10,000$ 条边来构建相应的更新流。


\textbf{查询图生成}
\label{ss-sec:querygen}

与之前的研究~\cite{csm-turboflux-DBLP:conf/sigmod/KimSHLHCSJ18,csm-symbi-DBLP:journals/pvldb/MinPPGIH21,csm-survey:DBLP:journals/pvldb/SunSLH22}类似,
我们通过从数据图中随机提取子图来生成查询图。
对于每个数据集,我们设置了六种不同的查询大小 ($|E_Q|$):$6$、$8$、$10$、$12$和$14$。
对于每个查询大小,我们提取了10个查询图。
随后,我们评估了不同的 $k$ 值:$100$、$300$、$500$、$700$和$900$。
报告的时间和空间效率结果是通过对相应生成的查询图执行不同算法,并对得到的结果取平均值得到的。

\textbf{对比方法}
本节实验将对我们所提出的算法在上述真实数据集上进行评估,主要涉及以下算法:

(1) Baseline: 第\label{ch3:base-framework}小节中提出的CSM-TopK问题的基础计算框架,并实现利用第k个子图匹配结果作为剪枝上限的基线方案。

(2)MWstar-global:第\ref{mwstar:global}小节中提出的基于全局MWstar索引的面向CSM-TopK问题的算法。

(3)MWstar-both:第\ref{mwstar:local}小节中提出的同时基于全局MWstar和局部MWstar的索引,利用更紧凑的密度上限,解决CSM-TopK问题的算法。

(4) Our-final: 第\label{mwstar:compact-graph}小结中提出的压缩图技术,将它与局部MWstar相结合,形成我们的最终解决方案,并且作为本章实验中的对比算法。

我们将我们的最终方案Our-final(简称为Ours),与现有的具有TopK密度约束的静态子图匹配工作进行比较,包括 \itk\cite{static-topk-Gupta-DBLP:conf/icde/GuptaGYCH14} 和 \pm\cite{static-topk-Chen-DBLP:journals/ijprai/ChenLCTL18}。
此外,我们还将我们的最终方案与几种最先进的CSM方法进行了对比,包含 Graphflow\cite{csm-graphflow-DBLP:conf/sigmod/KankanamgeSMCS17}、Rapidflow\cite{csm-rapidflow-DBLP:journals/pvldb/SunSHL22} 和 CaLiG~\cite{csm-calig-DBLP:journals/pacmmod/YangZZY23}。
需要注意的是,\itk 在处理 LiveJournal 和 YouTube 数据集时,因内存不足而导致无法执行,相关的实验结果标记为“无穷大”。

\section{实验结果分析}
\label{ch5:overall-compare}
\subsection{插入与删除效率对比}
\label{ch5:insertion-deletion}
本节旨在通过实验评估我们提出的算法与现有对比算法在不同数据集下的插入与删除性能。
在本实验中,我们通过固定参数$k$,测试查询图的大小在$6$,$8$,$10$,$12$时的插入和删除的平均时间,如图\ref{fig:time:insertion:fixQuerySize}和图\ref{fig:time:deletion:fixQuerySize}所示;
同样,我们通过固定查询图的边数$|E(Q)|$,测试参数$k$的大小在$100$,$300$,$500$,$700$,$900$时的插入和删除的平均时间,如图\ref{fig:time:insertion:fixKSize}和图\ref{fig:time:deletion:fixKSize}所示。

\input{\csmnewFig time_insertion_fixKsize}
\input{\csmnewFig time_insertion_fixQuerySize}
\input{\csmnewFig time_deletion_fixKsize}
\input{\csmnewFig time_deletion_fixQuerySize}

在实验中,我们通过改变 $k$ 值和查询图大小 $|E_Q|$ 来评估在不同数据集下的插入和删除的性能,其中横坐标表示实验中变化的参数,纵坐标表示算法的平均执行时间。


如图~\ref{fig:time:insertion:fixQuerySize} 和图~\ref{fig:time:deletion:fixQuerySize}所示,随着查询图大小的增加,所有算法的平均执行时间均有所上升。
这一趋势的产生可以归因于较大查询图的递归搜索深度通常高于较小查询图,从而增加了计算和存储的需求,导致算法的效率有所降低。
与此同时,如图~\ref{fig:time:insertion:fixKSize} 和图~\ref{fig:time:deletion:fixKSize}所示,在$k$值增加时,其他已有的CSM的解决方法的平均执行时间几乎不受影响。
这是因为这些算法均适用于连续子图匹配问题,并且未采用基于密度的剪枝策略,因此它们需要搜索所有新的匹配项以获取前$k$个匹配结果,并需要对所有子图匹配结果进行重排序。
因此,无论$k$值增加的幅度有多大,算法始终维护所有的子图匹配结果,导致其执行时间不随着$k$值影响。
特别地,在Orkut数据集上,Graphflow、RapidFlow和CaLiG的结果非常相似。
而对于\itk 和 \pm,虽然这两种方法采用了基于密度的剪枝策略,但是由于它们的索引重建需要消耗大量时间,因此不同 $k$ 值之间的性能差异不大。


我们还观察到,两个静态TopK方法(\itk 和 \pm)在动态更新场景中表现最差,因为它们的索引结构难以动态更新,每次数据更新都需要重新构建索引,
且其空间占用较高,导致在数据图较大时可能会因为内存空间耗尽而无法执行。


总体而言,从图\ref{fig:time:insertion:fixKSize}-图\ref{fig:time:deletion:fixQuerySize}的实验结果表明,我们的算法在插入和删除方面始终优于其他对比算法。
随着 $k$ 值的增加,我们的算法的执行时间虽然呈现轻微的上升趋势(见图~\ref{fig:time:insertion:fixKSize} 和图~\ref{fig:time:deletion:fixKSize}),但由于我们的算法比其他算法快2到4个数量级,这一上升趋势几乎可以忽略不计。
更新时间的轻微增加主要是由于随着 $k$ 增大,最小密度下界 $den(g_{min})$ 的松弛(见算法~\ref{alg:find-dense-matches}中的第~\ref{code:g-min-filter}行),导致了基于密度的剪枝效果有所弱化。


\subsection{空间效率对比}
\label{ch5:space}
\input{\csmnewFig space_memory}
我们进一步评估了不同算法的空间效率。
图~\ref{fig:exp:space:memory}展示了不同算法在不同查询图大小下的内存使用情况。
请注意,在CSM-TopK中,随着参数 $k$ 的变化,空间开销的差异可以忽略不计。
如图~\ref{fig:exp:space:memory}所示,我们的方法在空间效率上明显优于 \itk、\pm、Rapidflow 和 CaLiG。

在LiveJournal,YouTube,Orkut数据集上,\itk、\pm 这两种静态方法都耗尽内存而停止,这主要是因为它们的索引在大规模数据集上的时空复杂度都是指数级别的。
而GraphFlow由于不维护索引,其空间开销主要来自于维护所有匹配项,因此在空间效率上与我们的算法相似。
我们的算法通过设计轻量级的索引,其索引仅与查询图的一阶邻居有关,从而达到线性的空间复杂度。


\subsection{索引构建时间对比}
\label{ch5:index-construction}
\begin{figure}[h!]
    \centering
    \resizebox{0.7\linewidth}{!}{
    \includegraphics{\csmnewFig time_index_construct_all.pdf}
    }
    \caption{索引构建时间  ($|E_Q|=8, k=100$)}
    \label{fig:exp:time:index}
\end{figure}    
我们对不同算法的索引构建时间进行了对比,实验结果如图~\ref{fig:exp:time:index}所示,
可以看到,我们的算法在索引构建时间上比对比算法快2到5个数量级。

特别地,\pm 和 \itk 在索引构建时消耗的时间较长,因为它们在静态场景下采用离线构建索引,在动态场景下则需要每次更新时重新构建索引。
而Graphflow 由于不维护索引或其他辅助数据结构,因此不在此实验的对比范围内。


\subsection{自身对比}
\label{ch5:couterparts}
\begin{figure*}[h!]
    \def\wscorevone{0.19}
    \centering
        %=========================================amazon
        \begin{subfigure}[t]{\wscorevone\linewidth}
            \centering
            \resizebox{\linewidth}{!}
            {
                \includegraphics{\csmnewFig time_couterparts_fixQuerySize_amazon.pdf}
            }
            \caption{Amazon}
            \label{fig:time:counterpart:fixQuerySize:amazon}
        \end{subfigure}
        %=========================================livejournal
        \begin{subfigure}[t]{\wscorevone\linewidth}
            \centering
            \resizebox{\linewidth}{!}
            {
                \includegraphics{\csmnewFig time_couterparts_fixQuerySize_livejournal.pdf}
            }
            \caption{LiveJournal}
            \label{fig:time:counterpart:fixQuerySize:livejournal}
        \end{subfigure}
         %=========================================human
         \begin{subfigure}[t]{\wscorevone\linewidth}
             \centering
             \resizebox{\linewidth}{!}
             {
                 \includegraphics{\csmnewFig  time_couterparts_fixQuerySize_human.pdf}
             }
             \caption{Human}
             \label{fig:time:counterpart:fixQuerySize:human}
         \end{subfigure}
         %=========================================youtube
         \begin{subfigure}[t]{\wscorevone\linewidth}
             \centering
             \resizebox{\linewidth}{!}
             {
                 \includegraphics{\csmnewFig  time_couterparts_fixQuerySize_youtube.pdf}
             }
             \caption{YouTube}
             \label{fig:time:counterpart:fixQuerySize:youtube}
         \end{subfigure}
            %=========================================orkut
         \begin{subfigure}[t]{\wscorevone\linewidth}
            \centering
            \resizebox{\linewidth}{!}
            {
                \includegraphics{\csmnewFig  time_couterparts_fixQuerySize_orkut.pdf}
            }
            \caption{Orkut}
            \label{fig:time:counterpart:fixQuerySize:orkut}
        \end{subfigure}
    \caption{ 不同$E(Q)$下自身算法效率对比($k=100$)}
    \label{fig:time:counterpart:fixQuerySize}
    \end{figure*}
    
    
\input{./exp/newFIg/time_couterparts_fixKSize}


\label{sec:couterparts}
我们将我们提出的四种优化算法进行了对比,以验证最终优化策略的有效性。
图 \ref{fig:time:counterpart:fixQuerySize} -图 \ref{fig:time:counterpart:fixKSize} 展示了不同$k$值和查询图大小$|E_Q|$下,我们提出的不同算法的时间开销。
从实验结果中可以看到,基于压缩图的最终版本显著优于基于原始图的版本,性能提升了约 $1{\sim}2$ 个数量级。
同时,MWstar-global 和 MWstar-both 的性能明显优于基准方法。
从图 \ref{fig:time:counterpart:fixQuerySize} 到图 \ref{fig:time:counterpart:fixKSize} 还可以看出,MWstar-both 的搜索效率明显高于 MWstar-global。
上述实验结果表明,我们算法的优势在于 MWstar 索引和图压缩策略,将这两者结合形成最终的优化算法,能够将算法的执行效率提升到最高。

\section{案例研究}
\begin{figure}[h!]
    \centering
    \resizebox{0.8\linewidth}{!}{
    \includegraphics{./exp/newFIg/human_casestudy.pdf}
    }
    \caption{基于蛋白质相互作用网络的案例研究}
    \label{fig:human-caseStudy}
    \end{figure}    
图 \ref{fig:human-caseStudy} 展示了一个查询图 $Q_1$ 及其在蛋白质相互作用网络~\cite{dat-protein} 中的查询结果。
    查询图$Q_1$ 表示由一个调节蛋白和3个肽酶组成的关键结构,边的权重表示它们之间的相互促进或抑制的程度。
    我们根据边的权重优先级对子图进行排序,以识别与相互作用度最高的前 $k$ 个子图。
    在匹配序列 ${u_1, u_2, u_3, u_4}$ 下,最密集的答案是 $g_0=${$SIRT1$, $EP300$, $p53$, $MDM2$}。
    值得注意的是,$p53$ 是在人类细胞中广泛研究的肿瘤抑制蛋白,而其他三个蛋白激酶表现出强烈的相互作用,并对 $p53$ 的活性产生显著的促进作用,这可能与肿瘤疾病的状态密切相关。
\section{本章小结}
本章详细介绍了我们提出的算法的实验设计与结果分析。
在实验设计部分,我们首先介绍了实验环境与配置,包括硬件和软件平台的详细信息、五个真实数据集、查询图的生成方式、以及参与测试的算法。
在实验过程中,我们通过从数据图中随机抽取子图生成查询图,并对不同查询图大小和不同的$k$值进行了全面测试。
接着,我们对实验结果进行深入分析,重点对比了不同算法在时间和空间效率方面的表现。实验结果表明,与现有的解决方案相比,我们提出的方案在插入与删除效率上表现出显著优势,性能提升达到2至4个数量级,验证了我们优化方法在处理大规模数据集时的有效性和优越性。

此外,我们对比了我们自己提出的四种不同的算法版本,分别是基线算法、全局MWstar算法、全局与局部MWstar结合的算法以及最终提出的压缩图上的MWstar算法。
实验结果表明,我们的最终算法显著优于基线方法,提升了$1\sim2$个数量级。同时,基于MWstar的全局和局部索引方法也表现出明显的性能优势,特别是在搜索效率方面,进一步证明了我们算法的有效性和优越性。
最后,我们还给出了CSM-TopK问题在蛋白质相互交互网络上的案例研究,验证了此算法的应用研究价值。